\documentclass[12pt]{article}
\usepackage{amsmath}
\usepackage{siunitx}
\usepackage[shortlabels]{enumitem}
%\usepackage[margin=1in]{geometry}

\begin{document} 
    \textbf{Page 186}
    \begin{enumerate}[1)]
        \item Le son d'une guitare provient de la vibration des cordes, provenant d'une action exterieure, telle que celle de gratter les cordes. Le timbre est donné par la composition de la corde, ainsi que la forme et le materiau qui constitue la guitare.
        \item Si la corde est tendue, la fréquance augmente.
        \item Si la longeur de la corde est divisée en deux, alors la fréquance est doublée, la nouvelle fréquance est donc \SI{880}{\Hz}.
        \item La fréquance fondamentale dépend de la longeur de la corde et sa masse linéique.
        \item Dans les instruments à vent, le ton dépend de la longeur de l'instrument. Ceci peut etre facilement modifiée, grace aux trous qui se situent tout au long de l'instrument.
        \item La corde de \SI{50}{\cm} émet une fréquance de \SI{80}{\Hz}.
        \item La première harmonique à une fréquance qui est doublée. \\ Donc, $f_2=\SI{160}{\Hz}$
        \item La longeur de la corde est divisée en deux. \\ Donc, la fréquance doublera. $f_{1;\,\ell=\SI{25}{\cm}}=\SI{160}{\Hz}$
        \item Le plus la corde est longue, le plus sa fréquance fondamentale sera faible.
        \item Il y a deux raisons.
                \begin{itemize}
                    \item Les cordes de la contrebasse sont beaucoup plus longues. Ceci correspond au $\ell$ de la formule de $f_1$.
                    \item Les cordes de la contrebasse sont aussi plus épaisses et donc plus lourdes. Ceci correspond au $\mu$ de la formule de $f_1$, puisque $\mu=\frac{m}{\ell}$.
                \end{itemize}
    \end{enumerate}

\end{document}

