\documentclass[12pt]{article}
\usepackage{amsmath}
\usepackage{amssymb}
\usepackage{siunitx}
\usepackage[shortlabels]{enumitem}
\usepackage[upright]{fourier}
\usepackage[french]{babel}
\usepackage{geometry}
\usepackage[symbol]{footmisc}
\usepackage{longtable}
\geometry{
 a4paper,
 total={170mm,257mm},
 left=20mm,
 top=20mm,
 }

\renewcommand{\thefootnote}{\fnsymbol{footnote}}

\title{Enseignement Scientifique \\ Exercices page 198 à 201}
\author{Diego Van Overberghe}
\date{6 Mai 2020}

\begin{document}
    \maketitle

    \section{La Gamme de Pythagore}
    \begin{enumerate}
        \item Les notes d'une gamme doivent se siter dans l'intervalle d'un octave. Les notes en-dehors de l'octave ont une sonorité semblante.
        \item La quinte est le deuxième intervalle le plus consonnant, après l'octave. La quinte est equivalente à 7 demi-tons.
        \item $f_0=\text{do}$; \quad $f_1=\frac{3}{2}$; \quad $f_2=f_1\times\frac{3}{2}=\frac{3^2}{2^2}\times\left(\frac{1}{2}\right)\footnote[2]{L'intervalle serait supérieur a un octave, donc on le rabaisse d'un octave.}=\frac{3^2}{2^3}=\frac{9}{8}$; \quad $f_3=f_2\times\frac{3}{2}=\frac{3^3}{2^4}=\frac{27}{16}$
        \item $f_\text{sol}=f_\text{do}\times\frac{3}{2}=392{,}4\ \text{Hz}$; \quad $f_\text{ré}=f_\text{do}\times\frac{9}{8}=294{,}3\ \text{Hz}$; \quad $f_\text{la}=f_\text{do}\times\frac{27}{9}=441{,}5\ \text{Hz}$
        \item $f_\text{do cycle}=f_\text{do}\times\frac{3^{12}}{2^{12}}\times\frac{1}{2^6}\approx 530{,}3\ \text{Hz}\neq f_\text{do octave}=2f_\text{do}=523{,}2\ \text{Hz}$ \\ Le comma du pythagoricien vaut donc $\approx 7{,}138\ \text{Hz}$.
        \item $\Delta F_\text{do-ré}=\frac{294{,}3}{261{,}6}=1{,}125$; \quad $\Delta F_\text{ré-mi}=\frac{331{,}1}{294{,}3}\approx1{,}12504$; \quad $\Delta F_\text{mi-fa}=\frac{372{,}5}{331{,}1}\approx 1{,}12504$
        \item Des multiples de 12? 24, 48?
    \end{enumerate}

    \section{Vers un découpage égal de l'octave}
    \begin{enumerate}
        \item La gamme tempéree apparaît pendant l'époque Baroque (c. 1700). Sa particularité est que l'interval entre chaque note est identique.
        \item Le rapport de fréquances d'un demi-ton est de $\sqrt[12]{2}=2^{\frac{1}{12}}$. Le rapport des fréquances d'un ton est donc $\sqrt[12]{2}^2=2^{\frac{2}{12}}$.
        \item Fréquences des notes de la gamme à intervalles égaux
                \begin{longtable}{|c|c|} \hline \endfirsthead
                    \textbf{Note} & \textbf{Fréquance} (en Hz) \\ \hline
                    do            & 261{,}6                    \\ \hline
                    do$^\sharp$   & 277{,}2                    \\ \hline
                    ré            & 293{,}7                    \\ \hline
                    ré$^\sharp$   & 311{,}1                    \\ \hline
                    mi            & 329{,}6                    \\ \hline
                    fa            & 349{,}2                    \\ \hline
                    fa$^\sharp$   & 370{,}0                    \\ \hline
                    sol           &$f_\text{sol}=f_\text{la}\times\sqrt[12]{12}^{-2}\approx 392{,}0$ \\ \hline
                    sol$^\sharp$  & $f_{\text{sol}\sharp}=f_\text{la}\times\sqrt[12]{12}^{-1}\approx 415{,}3$ \\ \hline
                    la            & 440{,}0                    \\ \hline
                    la$^\sharp$   & $f_{\text{la}\sharp}=f_\text{la}\times\sqrt[12]{12}^1\approx 466{,}2$ \\ \hline
                    si            & $f_\text{si}=f_\text{la}\times\sqrt[12]{12}^2\approx 493{,}9$ \\ \hline
                    do            & 523{,}2                    \\ \hline
                \end{longtable}
    \item $\Delta f_\text{do-sol}=\frac{392{,}0}{261{,}6}\approx 1{,}4985$ \\ Le rapport n'est pas exactement égal à $\frac{3}{2}$, la qualité du son est donc légèrement inférieure.
    \item Le do devient un fa.
    \item L'avantage de la gamme à intervalles égaux est la possibilité de transposer les morceaux. Ceci permet à des instruements accordés différemments de jouer ensemble.
    \end{enumerate}
\end{document}