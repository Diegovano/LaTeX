\documentclass[DIV=12]{scrartcl}
\usepackage[french]{babel}
\usepackage{newpxtext}
\usepackage{siunitx}
\usepackage{enumitem}
\selectlanguage{french}
\usepackage[varbb]{newpxmath}
\setlength{\parskip}{1em}
\setlength{\parindent}{0pt}

\sisetup{
    inter-unit-product = \ensuremath{{}\cdot{}}
}

\renewcommand{\thesection}{\Roman{section}}

\title{Pourquoi Certains Écrans Tactiles Fonctionnent-ils avec des Gants et d'Autres~Non~?}
\author{Diego Van Overberghe}
\setitemize{itemsep=1ex}

\begin{document}
    \maketitle
   \section{L'histoire des Écrans Tactiles}
   Deux Types de Technologies :
   \begin{itemize}
       \item Capteurs Tactiles Résistifs, développés en 1971  par Sam Hurst aux États-Unis.
       \item Capteurs Tactiles Capacitifs, développés dans les années 1970 au CERN, iPhone multi-touch en 2007.
   \end{itemize}
   \section{Fonctionnement}
   \begin{itemize}
        \item Les capteurs de type résistifs sont composés de deux plaques conductrices, transparentes, séparées les unes des autres par un isolant tel qu'une couche d'air.
        Lorsqu'un utilisateur met de la pression sur l'écran, les deux plaques se rapprochent, et la distance séparant les conducteurs diminue. La résistance est inversement proportionnelle à la distance donc, la résistance diminue. : \[R\propto\frac{1}{\ell}\qquad\begin{cases}~R\text{ Résistance en }\Omega\\~\ell\text{ La distance qui sépare les plaques en m}\end{cases}\]
        D'après la loi d'Ohm, un changement de résistance provoque soit un changement de tension entre les deux plaques, soit un changement de courant.
       
        Cette modification est captée et un micro-contrôleur localise l'emplacement du rapprochement.
        \pagebreak\item Un condensateur est un composant électronique constitué de deux plaques (des armatures) qui sont séparées par un isolant.
        
        Pour une tension donnée, un condensateur peut stocker une certaine charge. La grandeur qui relie la charge portée par un condensateur pour une tension donnée est nommée capacité, se note : \[q=C\cdot u_C\qquad\begin{cases}~q\text{ charge en C}\\~C\text{ capacité électrique en F}\\~u_C\text{ tension en Volt}\end{cases}\]
        
        Lorsque le doigt de l'utilisateur se rapproche de l'écran, une charge s'accumule sur la surface de l'écran au niveau du doigt. L'écran étant composé d'une grille constituée d'une multitude de points, reliés a des condensateurs de valeur connue, on peut détecter un changement de capacité local.
        
        Un micro-contrôleur détecte ce changement de capacité et repère la position du doigt sur l'écran.
   \end{itemize}
   \section{Réponse à la Question Posée}
   Nous voyons donc que le fonctionnement des écrans à capteurs capacitifs dépend de la détection d'un changement de capacité. Assimilons le doigt et l'écran à deux armatures d'un condensateur plan. Alors, la capacité de ce condensateur vérifie : 
   \[C=\frac{\epsilon_0\epsilon_r S}{d}\qquad\begin{cases}~C~\text{capacité électrique en C}\\~\epsilon_0,~\epsilon_r~\text{permittivités en $\si{\farad.\meter^{-2}}$}\\~S~\text{surface des armatures en $\si{m^{-2}}$}\\~d~\text{écartement des armatures en $\si{m}$}\end{cases}\]
   \section{Applications}
   Les écrans tactiles résistifs sont typiquement utilisés sur les distributeurs automatiques, les caisses automatiques au supermarché ou sur les GPS. \\
   Ce type d'écran permet d'utiliser un stylet, plus précis que le doigt. On peut aussi utiliser ce type d'écran sous l'eau ou dans des environments difficiles, ou avec des gants ! Finalement, ce type d'écran est moins cher à produire.

   Les écrans capacitifs sont plus rigides, réactifs et surtout permettent le multitouch, fonction essentielle pour un téléphone portable moderne.
\end{document}