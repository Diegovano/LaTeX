\documentclass[DIV=12]{scrartcl}
\usepackage[french]{babel}
\usepackage{newpxtext}
\usepackage{siunitx}
\selectlanguage{french}
\usepackage[varbb,frenchmath]{newpxmath}
\setlength{\parskip}{0.5em}
\setlength{\parindent}{0pt}

\renewcommand{\thesection}{\Roman{section}}

\title{Pourquoi Certains Écrans Tactiles Fonctionnent-ils avec des Gants et d'Autres~Non~?}
\author{Diego Van Overberghe}

\begin{document}
    \maketitle
   \section{L'histoire des Écrans Tactiles}
   Deux Types de Technologies :
   \begin{itemize} 
       \item Capteurs Tactiles Résistifs, développés en 1971  par Sam Hurst aux États-Unis.
       \item Capteurs Tactiles Capacitifs, développés dans les années 1970 au CERN, iPhone multi-touch en 2007.
   \end{itemize}
   \section{Fonctionnement}
   \begin{itemize}
        \item Les capteurs de type résistifs sont composés de deux plaques conductrices, transparentes, séparées les unes des autres par un isolant tel qu'une couche d'air.
        Lorsqu'un utilisateur met de la pression sur l'écran, les deux plaques se rapprochent, et la distance séparant les conducteurs diminue. La résistance est inversement proportionnelle à la distance donc, la résistance diminue. : \[R\propto\frac{1}{\ell}\qquad\begin{cases}~R\text{ Résistance en }\Omega\\~\ell\text{ La distance qui sépare les plaques en m}\end{cases}\]
        D'après la loi d'Ohm, un changement de résistance provoque soit un changement de tension entre les deux plaques, soit un changement de courant.
       
        Cette modification est captée et un micro-contrôleur localise l'emplacement du rapprochement.
        \item Un condensateur est un composant électronique constitué de deux plaques (des armatures) qui sont séparées par un isolant.
        
        Pour une tension donnée, un condensateur peut stocker une certaine charge. La grandeur qui relie la charge portée par un condensateur pour une tension donnée est nommée capacité, se note : \[q=C\cdot u\qquad\begin{cases}~q\text{ charge en C}\\~C\text{ capacité électrique en F}\\~u\text{ tension en Volt}\end{cases}\]
        
        Lorsque le doigt de l'utilisateur se rapproche de l'écran, une charge s'accumule sur la surface de l'écran au niveau du doigt.
        
        Un microcontrolleur détecte cette accumulation de charge afin de déduire la position du doigt.
   \end{itemize}
\end{document}