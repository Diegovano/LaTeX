
\documentclass[12pt, a4paper]{article}
\usepackage{exercise}
\usepackage{amsmath}
\usepackage{amsfonts}
\usepackage{siunitx}
\usepackage[shortlabels]{enumitem}
\usepackage[margin=2.5cm]{geometry}
%\usepackage{graphicx}
\usepackage{lmodern}
\usepackage[french]{babel}
\usepackage[upright]{fourier}
%\usepackage[justification=centering]{caption}

%\graphicspath{ {./rsc/} }

\sisetup{inter-unit-product=\ensuremath{{}\cdot{}}}
\renewcommand{\ExerciseHeader}
{
  \par\noindent
  \textbf{\large \ExerciseName\ \ExerciseHeaderNB\ExerciseHeaderTitle\ExerciseHeaderOrigin}%
  \par\nopagebreak\medskip
}

\renewcommand{\ExerciseName}{Exercice}

\newenvironment{conditions}
  {\par\vspace{\abovedisplayskip}\noindent\begin{tabular}{>{$}l<{$} @{${\quad}={\quad}$} l}}
  {\end{tabular}\par\vspace{\belowdisplayskip}}

\begin{document}
    \title{Exercices du Chapitre Modèles Ondulatoire et Particulaire de la Lumière \\ \Large{Pages 349-356}}
    \author{Diego Van Overberghe}
    \date{9 Juin 2020}
    \maketitle

    \begin{Exercise}[number={9}]
        \begin{enumerate}[A.]
            \item	Il s'agit des rayons-X: $10^{-12}\ \si{m}<\lambda<10^{-8}\ \si{m}$
            \item   Il s'agit des ondes hertziennes: $10^0\ \si{m}<\lambda<+\infty\ \si{m}$
            \item   Il s'agit des infrarouge: $10^{-6}\ \si{m}<\lambda<10^{-3}\ \si{m}$
            \item   Il s'agit des ultraviolet: $10^{-8}\ \si{m}<\lambda<10^{-7}\ \si{m}$
            \item   Il s'agit des infrarouge: $10^{-6}\ \si{m}<\lambda<10^{-3}\ \si{m}$
        \end{enumerate}
    \end{Exercise}

    \begin{Exercise}[number={11}]
        \begin{enumerate}[1.]
            \item	\begin{enumerate}[a.]
                        \item	L'échelle { \large\textcircled{\small{B}} } est l'échelle de fréquence
                        \item   L'échelle { \large\textcircled{\small{A}} } est l'échelle de la longeur d'onde dans le vide.
                    \end{enumerate}
            \item   D'après la formule 
                    \begin{equation*}
                        \nu=\frac{c}{\lambda}
                    \end{equation*} On voit que la fréquence est inversement proportionnelle à la longeur d'onde, ($c$) étant constante.
        \end{enumerate}
    \end{Exercise}

    \begin{Exercise}[number={15}]
        \begin{enumerate}[1.]
            \item	L'ordre de grandeur des ondes FM est de $10^8\ \si{Hz}$
            \item   Il s'agit d'ondes hertziennes.
            \item   $\lambda\times\nu=c\iff\lambda=\frac{c}{\nu}$ \\ $\lambda_1=\frac{3{,}00\times 10^8}{108\times 10^6}=2{,}8\ \si{m}$ \\ $\lambda_2=\frac{3{,}00\times 10^8}{87\times 10^6}=3{,}4\ \si{m}$
        \end{enumerate}
    \end{Exercise}

    \begin{Exercise}[number={18}]
        \begin{enumerate}[1.]
            \item	\begin{enumerate}[a.]
                        \item	$E=\frac{hc}{\lambda}$ \quad $E=\frac{6{,}63\times 10^{-34}\times 3{,}00\times 10^8}{200\times 10^{-6}}=9{,}95\times 10^{-22}\ \si{J}$
                        \item   $E=\frac{6{,}63\times 10^{-34}\times 3{,}00\times 10^8}{580\times 10^{-9}}=3{,}43\times 10^{-19}\ \si{J}$
                        \item   $E=\frac{6{,}63\times 10^{-34}\times 3{,}00\times 10^8}{0{,}100\times 10^3}=1{,}99\times 10^{-21}\ \si{J}$
                    \end{enumerate}
            \item   \begin{enumerate}[a.]
                        \item	$E=h\nu$ \quad $E=6{,}63\times 10^{-34}\times 3{,}90\times 10^{14}=2{,}59\times 10^{-19}\ \si{J}$
                        \item   $E=6{,}63\times 10^{-34}\times 5{,}00\times 10^{9}=3{,}32\times 10^{-24}\ \si{J}$
                        \item   $E=6{,}63\times 10^{-34}\times 1{,}50\times 10^{6}=9{,}95\times 10^{-28}\ \si{J}$
                    \end{enumerate}
            \item   \begin{enumerate}[a.]
                        \item	Plus la longeur d'onde augmente, et plus l'énergie des photons diminue.
                        \item   Plus la fréquence augmente, et plus l'énergie des photons augmente.
                    \end{enumerate}
        \end{enumerate}
    \end{Exercise}

    \begin{Exercise}[number={19}]
        \begin{enumerate}[1.]
            \item	$E=\frac{hc}{\lambda}=\frac{6{,}63\times 10^{-34}\times 3{,}00\times 10^{8}}{405\times 10^{-9}}=4{,}91\times 10^{-19}\ \si{J}$
            \item   L'énergie d'un photon du laser rouge aura une énergie inféieure parce que la longeur d'onde est plus importante.
            \item   \begin{enumerate}[a.]
                        \item	$n_\text{photons bleus}=\frac{50\times 10^{-3}}{4{,}19\times 10^{-19}}=2{,}39\times 10^{15}\ \text{photons}$
                        \item   $E=\frac{6{,}63\times 10^{-34}\times 3{,}00\times 10^{8}}{650\times 10^{-9}}=3{,}06\times 10^{-19}\ \si{J}$ \\ $n_\text{photons rouges}=\frac{50\times 10^{-3}}{2{,}84\times 10^{-9}}=3{,}52\times 10^{15}\ \text{photons}$
                    \end{enumerate}
        \end{enumerate}
    \end{Exercise}

    \begin{Exercise}[number={21}]
        \begin{enumerate}[1.]
            \item	Les traits horizontaux représentent les différents niveaux d'énergie de l'atome.
            \item   \begin{enumerate}[a.]
                        \item	Le premier schéma explique l'absorption d'un électron parce que l'atome absorbe l'énergie du photon pour exciter les électrons.
                        \item   Le deuxième schéma explique l'emission d'un électron parce que l'emmision de l'électron est le resulat de la transformation de l'énergie perdue lorsque les électrons perdent un niveau d'énergie.
                    \end{enumerate}
            \item   \begin{enumerate}[a.]
                        \item	L'atome gangne de l'énergie en absorbant un photon.
                        \item   L'atome perd de l'énergie en émmetant un photon.
                    \end{enumerate}
        \end{enumerate}
    \end{Exercise}
    
    \begin{Exercise}[number={23}]
        \begin{enumerate}[1.]
            \item	Il s'agit d'un spectre d'émission.
            \item   $E=\frac{hc}{\lambda}=\frac{6{,}63\times 10^{-34}\times 3{,}00\times 10^{8}}{589{,}0\times 10^{-9}}=3{,}38\times 10^{-19}\ \si{J}=2{,}11\ \si{eV}$ \\ $E=\frac{6{,}63\times 10^{-34}\times 3{,}00\times 10^{8}}{589{,}6\times 10^{-9}}=3{,}37\times 10^{-19}\ \si{J}=2{,}11\ \si{eV}$
            \item   L'énergie des deux photons absorbés est quasimment égale. C'est $2{,}11\ \si{eV}$.
        \end{enumerate}
    \end{Exercise}

    \begin{Exercise}[number={29}]
        \begin{enumerate}[1.]
            \item	La propriété
            \item   Le fait qu'il y ait plus de raies sombres s'explique par l'existance de plus d'élements dans l'atmosphère de l'étoile. Ceci veut dire que l'étoile est plus chaude et donc sans doute aussi plus massive aussi.
            \item   \begin{enumerate}[a.]
                        \item	La flèche bleue représente l'augmentation de niveau d'énergie par absorption d'un photon.
                        \item   $\Delta E=\lvert -1{,}51-(-3{,}40) \rvert=1{,}89\ \si{eV}=3{,}02\times 10^{-19}\ \si{J}$ \\ On calcule maintenant la fréquence $\nu$ d'un photon portant cette charge. \\ $\lambda=\frac{hc}{\Delta E}=\frac{6{,}63\times 10^{-34}\times 3{,}00\times 10^{8}}{3{,}02\times 10^{-19}}=659\times 10^{-9}\ \si{nm}$ \quad Ceci correspond à la raie G.
                    \end{enumerate}
        \end{enumerate}
    \end{Exercise}

    \begin{Exercise}[number={31}]
        \begin{enumerate}[1.]
            \item	Les traits horizontaux corréspondent aux différents niveaux d'énergie de l'atome
            \item   { \large\textcircled{\small{1}} }$\rightarrow$ absorption d'un photon \\ { \large\textcircled{\small{2}} }$\rightarrow$ émission par phosphoréscence. \\ { \large\textcircled{\small{3}} }$\rightarrow$ émission par fluoréscence
            \item   \begin{enumerate}[a.]
                        \item	Photon { \large\textcircled{\small{1}} }$\rightarrow$ ultraviolet. \\ Photon { \large\textcircled{\small{2}} } et { \large\textcircled{\small{3}} }$\rightarrow$ visible
                        \item   L'energie des deux photons émis est inférieure à l'énergie du photon absorbé.
                    \end{enumerate}
        \end{enumerate}
    \end{Exercise}

    \begin{Exercise}[number={32}]
        \begin{enumerate}[1.]
            \item	Les Modèles ondulatoire et particulaire sont deux modèles qui expliquent chaquns différentes observations experimentales.
            \item   $c=\lambda\nu\iff\nu=\frac{c}{\lambda}$ \quad $\nu=\frac{3{,}00\times 10^{8}}{450\times 10^{-9}}=6{,}67\times 10^{14}\ \si{Hz}$ \smallbreak On peut donc observer l'effet photoélectrique dans Césium, le Potassium et le Baryum.
            \item   Le métal qui nécessite le photon le plus énergétique est le Zinc, celui qui nécessite le photon le moins énergétique est le Césium.
        \end{enumerate}
    \end{Exercise}

    \begin{Exercise}[number={34}]
        \begin{enumerate}[1.]
            \item	$c=\lambda\nu$, $c$ est constante donc doubler la fréquence implique diviser la longeur d'onde par deux.
            \item   \begin{enumerate}[a.]
                        \item	$\lambda=532\ \si{nm}$
                        \item   On peut voir que cette longeur d'onde correspond à peu près à la couleur verte.
                        \item   $E=\frac{hc}{\lambda}=\frac{6{,}63\times 10^{-34}\times 3{,}00\times 10^{8}}{532\times 10^{-9}}=3{,}74\times 10^{-19}\ \si{J}$ \\ $n_\text{photons}=\frac{200\times 10^{-3}}{3{,}74\times 10^{-19}}=5{,}35\times 10^{17}\ \text{photons}$
                    \end{enumerate}
        \end{enumerate}
    \end{Exercise}

    \begin{Exercise}[number={36}]
        \begin{itemize}
            \item[] On calcule l'énergie d'un photon émis par le néon. \smallbreak $E=\frac{hc}{\lambda}=\frac{6{,}63\times 10^{-34}\times 3{,}00\times 10^{8}}{621{,}5\times 10^{-9}}=3{,}20\times 10^{-19}\ \si{J}=2{,}00\ \si{eV}$ \smallbreak Ceci correspond donc à une transition du niveau 4 au niveau 3.
        \end{itemize}
    \end{Exercise}
\end{document}