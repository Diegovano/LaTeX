\documentclass{cours}

\title{Opérations sur les Variables Aléatoires}

\begin{document}
    \maketitle{15}

    \begin{Gpartie}{Rappels} 
        Soit $\Omega$ l'univers d'une experience aléatoire.

        Une variable aléatoire $X$ est une fonction qui à tout élément de $\Omega$ associe un nombre réel.
        \begin{Spartie}{Exemple} 
            On tire deux fois une pièce. $\Omega=\big\{~PP~,~PF~,~FP~,~FF~\big\}$

            Soit $X$ la variable aléatoire définie sur $\Omega$ qui donne le nombre de \og pile \fg obtenus.

            Soit $G$ la variable aléatoire égale au gain suivant :
            \begin{itemize}
                \item Deux faces identiques ( PP et FF ) font gagner $\pounds~2$.
                \item Deux faces différentes ( PF et FP ) font perdre $\pounds~2$.
            \end{itemize}

            Les lois de probabilités des variables aléatoires $X$ et $G$ sont :

            \begin{center}\begin{tabular}{ | p{0.12\linewidth}||*{3}{>{\centering\arraybackslash}m{0.05\linewidth} | }} \hline
                $x_i$     & $0$             & $1$             & $2$             \\ \hline
                $P(X=x_i)$& $\frac{1}{4}$   & $\frac{1}{2}$   & $\frac{1}{4}$   \\ \hline
            \end{tabular}\hspace{4ex}\begin{tabular}{ | p{0.12\linewidth}||*{2}{>{\centering\arraybackslash}m{0.05\linewidth} | }} \hline
                $g_i$       & $-2$          & $2$           \\ \hline
                $P(G=g_i)$  & $\frac{1}{2}$ & $\frac{1}{2}$ \\ \hline
            \end{tabular}\end{center}
            \parbox{\linewidth}{\captionof{figure}{\centering Lois de Probabilité de X et G}}
        \end{Spartie}
        \begin{Spartie}{Définition} 
            L'espérance de la variable aléatoire $X$ est le réel : \[E(X)=\sum_{i=1}^nx_iP\left(X=x_i\right)\]
            L'espérance est une moyenne, les probabilités jouent le role de fréquences.

            Dans notre exemple, $E(X)=1$ et $E(G)=0$
        \end{Spartie}
        \begin{Spartie}{Définition} 
            La variance de la variable aléatoire $X$ est le réel : \[V(X)=\sum_{i=1}^n\left(x_i-E(X)\right)^2P(X=x_i)\]
            C'est la moyenne des écarts au carré
        \end{Spartie}
        \begin{Spartie}{Définition} 
            L'écart-type de la variable aléatoire $X$ est le réel : \[\sigma=\sqrt{V(X)}\]
        \end{Spartie}
        \begin{Spartie}{Théorème} 
            \[V(X)=\sum_{i=1}^nx_i^2P(X=x_i)-\left(E(X)\right)^2=E\left(X^2\right)-\left(E(X)\right)^2\]
            La variance est aussi la moyenne des valeurs au carré moins la moyenne au carré.
            \begin{SSpartie}{Démonstration} 
                $\begin{aligned}[t]
                    V(X)&=\sum_{i=1}^n\left(x_i-E(X)^2\right)p_i\qquad\left(P(X=x_i)=p_i\right) \\
                    &=\sum_{i=1}^n\left(x_i^2-2x_iE(X)+\left(E(X)\right)^2\right)p_i \\
                    &=\sum_{i=1}^n\left(x_i^2p_i\right)-2E(X)\sum_{i=1}^n x_i p_i+\left(E(X)\right)^2\sum_{i=1}^np_i \\
                    &=E\left(X^2\right)-2\left(E(X)\times E(X)\right)+\left(E(X)\right)^2\times 1 \\
                    &=E\left(X^2\right)-\left(E(X)\right)^2\quad\square
                \end{aligned}$
            \end{SSpartie}
        \end{Spartie}
    \end{Gpartie}
    \pagebreak
    \begin{Gpartie}{Opérations sur les Variables Aléatoires} 
        \begin{Spartie}{Changement Affine} 
            \begin{SSpartie}{Théorème} 
                Si $X$ est une variable aléatoire et $a$ et $b$ sont deux réels : \[(1)\quad E(aX+b)=aE(X)+b\] \[(2)\quad V(aX+b)=a^2V(X)\]
                \begin{SSSpartie}{Démonstrations}
                    On note $p_i$ la probabilité $P(X=x_i)$ pour alléger les notations.
                    \begin{enumerate}[(1),leftmargin=4ex]
                        \item $\begin{aligned}[t]
                            E(aX+b)&=\sum_{i=1}^n\left(a x_i+b\right)p_i \\
                            &=a\sum_{i=1}^n x_i p_i+b\sum_{i=1}^np_i \\
                            &=aE(X)+b\quad\square
                        \end{aligned}$
                        \item $\begin{aligned}[t]
                            V(aX+b)&=E\left((aX+b)^2\right)-\left(E(aX+b)\right)^2 \\
                            &=E\left(a^2X^2+2aXb+b^2\right)-\left(E(aX+b)\right)^2 \\
                            &=a^2E\left(X^2\right)+2abE(X)+b^2-a^2\left(E(X)\right)^2-2abE(X)-b^2 \\
                            &=a^2\left(E\left(X^2\right)-\left(E(X)\right)^2\right) \\
                            &=a^2V(X)\quad\square
                        \end{aligned}$
                    \end{enumerate}
                \end{SSSpartie}
            \end{SSpartie}
            \begin{SSpartie}{Théorème} 
                Si $X$ est une variable aléatoire et $a$ et $b$ sont deux réels : \[\sigma(aX+b)=\lvert a\rvert \sigma(X)\]
                \begin{SSSpartie}{Démonstration} 
                    $\begin{aligned}[t]
                        \sigma(aX+b)&=\sqrt{V(aX+b)} \\
                        &=\sqrt{a^2V(X)} \\ 
                        &=\lvert a\rvert\sigma(X)\quad\square
                    \end{aligned}$
                \end{SSSpartie}
            \end{SSpartie}
            \begin{SSpartie}{Exemple} 
                Dans le jeu précédent (deux tirages de pièce), si $X'$ est la variable aléatoire qui donne le triple de nombre de \og pile \fg , $X'=3X$ et donc $E\left(X'\right)=3E(X)=3$
            \end{SSpartie}
        \end{Spartie}
        \begin{Spartie}{Somme de Variables Aléatoires} 
            Dans cette sous-partie, $X$ et $Y$ sont deux variables aléatoires prenant respectivement les valeurs $\big\{~a_1~,~a_2~,~\dotsc~,~a_n~\big\}$ et $\big\{~b_1~,~b_2~,~\dotsc~,~b_m~\big\}$, où $n$ et $m$ sont des entiers naturels non nuls.
            \begin{SSpartie}{Définition} 
                La variable aléatoire $X+Y$ prend tous les valeurs $a_i+b_j$ possibles (où $1\leq i\leq n$ et $q\leq j\leq m$)

                Pour toute valeur de $w$, $P(X+Y=w)=\sum_{a_i+b_j=w}P\left(\{X=a_i\}\cap\{Y=b_j\}\right)$ c'est-à-dire la somme des probabilités $P\left(\{X=a_i\}\cap\{Y=b_j\}\right)$ telles que $a_i+b_j=w$ (toutes les sommes égales à $w$).
            \end{SSpartie}
            \begin{SSpartie}{Théorème (admis)} 
                \[E(X+Y)=E(X)+E(Y)\]
            \end{SSpartie}
            \begin{SSpartie}{Rappel} 
                $A$ et $B$ sont deux événements indépendants si et seulement si $P\left(A\cap B\right)=P(A)\times P(B)$.
            \end{SSpartie}
            \begin{SSpartie}{Définition} 
                Deux variables aléatoires sont indépendantes si elles donnent des résultats de deux expériences aléatoires indépendantes.
            \end{SSpartie}
            \begin{SSpartie}{Propriété} 
                Si $X$ et $Y$ indépendantes : \[P\left(\{X=a_i\}\cap\{Y=b_j\}\right)=P(X=a_i)\times P(Y=b_j)\]
            \end{SSpartie}
            \begin{SSpartie}{Théorème (admis)} 
                Si $X$ et $Y$ sont deux variables aléatoires indépendantes définies sur un meme univers : \[V(X+Y)=V(X)+V(Y)\]
            \end{SSpartie}
        \end{Spartie}
    \end{Gpartie}
    \begin{Gpartie}{Applications} 
        \begin{Spartie}{Loi Binomiale} 
            \begin{SSpartie}{Propriété} 
                Si $X$ est une variable aléatoire qui suit la loi binomiale $\mathcal{B}~\left(~n~;~p~\right)$ : \[E(X)=np\qquad V(X)=np\left(1-p\right)\qquad\sigma(X)=\sqrt{np\left(1-p\right)}\]
                \begin{SSSpartie}{Démonstration} 
                    $X$ est la somme de $n$ variables aléatoires indépendantes $X_1~,~X_2~,\dotsc,~X_n~$, suivant la loi de Bernoulli.

                    Pour tout $i$, $1\leq i\leq n,~E\left(X_i\right)=p$ et $V\left(X_i\right)=p\left(1-p\right)$.

                    Ainsi, $E(X)=E\left(X_1\right)+E\left(X_2\right)+\dotsb+E\left(X_n\right)=np$ et idem pour $V(X)$. \\
                    Enfin, $\sigma(X)=\sqrt{V(X)}$
                \end{SSSpartie}
            \end{SSpartie}
        \end{Spartie}
        \begin{Spartie}{Somme et Moyenne d'un Échantillon (généralisation)} 
            \begin{SSpartie}{Définition} 
                Un échantillon de taille $n$ d'une loi de probabilité est une liste $\big\{~X_1~,~X_2~,~\dotsc~,~X_n~\big\}$ de $n$ variables aléatoires identiques et indépendantes qui suivent toutes cette loi.
            \end{SSpartie}
            \begin{SSpartie}{Propriété} 
                Si on note $X$ une variable aléatoire suivant la loi de probabilité de cet échantillon, la variable aléatoire \og somme \fg{} de cet échantillon $S_n=X_1+X_2+\dotsb+X_n$ vérifie : \[(1)~E\left(S_n\right)=nE(X)\qquad(2)~V\left(S_n\right)=nV(X)\qquad(3)~\sigma\left(S_n\right)=\sqrt{n}\sigma(X)\]
                \begin{SSSpartie}{Démonstration} 
                    \begin{enumerate}[(1)]
                        \item $\begin{aligned}[t]
                            E\left(S_n\right)&=E\left(X_1+X_2+\dotsb+X_n\right) \\
                            &=E(X_1)+E(X_2)+\dotsb+E(X_n) \\
                            &=nE(X)\quad\text{Car les variables sont identiques}
                        \end{aligned}$
                        \item $\begin{aligned}[t]
                            V\left(S_n\right)&=V(X_1+X_2+\dotsb+X_n) \\
                            &=V(X_1)+V(X_2)+\dotsb+V(X_n) \\
                            &=nV(X)
                        \end{aligned}$
                        \item $\begin{aligned}[t]
                            \sigma(S_n)&=\sqrt{V(S_n)} \\
                            &=\sqrt{nV(X)} \\
                            &=\sqrt{n}\sigma(X)
                        \end{aligned}$
                    \end{enumerate}
                \end{SSSpartie}
            \end{SSpartie}
            \begin{SSpartie}{Propriété} 
                Si $X$ est une variable aléatoire suivant la loi de probabilité de cet échantillon, la variable aléatoire \og moyenne \fg{} de cet échantillon $M_n=\frac{X_1+X_2+\dotsb+X_n}{n}=\frac{S_n}{n}$ vérifie : \[(1)~E\left(M_n\right)=E(X)\qquad(2)~V(M_n)=\frac{V(X)}{n}\qquad(3)~\sigma(M_n)=\frac{\sigma(X)}{\sqrt{n}}\]
                \begin{SSSpartie}{Démonstration} 
                    \begin{enumerate}[(1)]
                        \item $\begin{aligned}[t]
                            E(M_n)&=E\left(\frac{S_n}{n}\right) \\
                            &=nE\left(\frac{X}{n}\right) \\
                            &=\frac{1}{n}\times nE(X)\qquad\text{changement affine de $\tfrac{1}{n}$}\\
                            &=E(X)
                        \end{aligned}$
                        \item $\begin{aligned}[t]
                            V(M_n)&=V\left(\frac{S_n}{n}\right) \\
                            &=nV\left(\frac{X}{n}\right) \\
                            &=\frac{1}{n^2}\times nV(X) \\
                            &=\frac{V(X)}{n}
                        \end{aligned}$
                        \item $\begin{aligned}[t]
                            \sigma(M_n)&=\sqrt{V(M_n)} \\
                            &=\frac{\sigma(X)}{\sqrt{n}}
                        \end{aligned}$
                    \end{enumerate}
                \end{SSSpartie}
            \end{SSpartie}
        \end{Spartie}
    \end{Gpartie}
\end{document}