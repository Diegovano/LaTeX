\documentclass{cours}
\usepackage{array}
\usepackage{diagbox}

\setlength{\arraycolsep}{1cm}

\title{Limites de Suites}

\begin{document}
    \maketitle{7}

    \begin{Gpartie}{Suites Majorées, Minorées et Bornées} 
        \begin{Spartie}{Définitions} 
            \begin{itemize}
                \item Une suite ($u_n$) est majorée s'il existe un réel $M$ tel que : $\forall n\in\mathbb{N},\ u_n\leq M$
                \item Une suite ($u_n$) est minorée s'il existe un réel $m$ tel que : $\forall n\in\mathbb{N},\ u_n\geq m$
                \item Une suite est bornée si elle est majorée et bornée
            \end{itemize}
        \end{Spartie}
        \begin{Spartie}{Exemple} 
            La suite $\left(\frac{1}{n}\right)_{n\geq 1}$ est minorée par 0, mais aussi par tout nombre négatif. 
            
            Elle est majorée par 1 (qui est aussi son maximum) et par tout nombre supérieur à 1. Elle est donc bornée.
        \end{Spartie}
    \end{Gpartie}
    \begin{Gpartie}{Définitions} 
        \begin{Spartie}{Limite Infinie}
            \begin{SSpartie}{Définition} 
                Une suite $(u_n)$ a pour limite $+\infty$ si, quel que soit le réel $M$, l'intervalle $\big]M\,;+\infty\big[$ contient tous les termes de la suite à partir d'un certain rang.

                Autrement dit, pour tout réel $M$, on peut trouver un rang $n$ tel que : \[\forall n\geq N,\quad u_n>M\]
                (A partir du rang $N$, tous les termes sont supérieurs à $M$)

                On note $\lim\limits_{n\to +\infty} u_n=+\infty$,\quad On peut dire que la suite diverge vers $+\infty$

                H.P. : $\forall M\in\mathbb{R},\ \exists N\in\mathbb{N},\ \forall n\in\mathbb{N},\ (n\geq N\implies u_n>M)$
            \end{SSpartie}
            \begin{SSpartie}{Définition} 
                Une suite ($u_n$) a pour limite $-\infty$, si, quel que soit le réel $m$, l'intervalle $\big]-\infty\,;m\big[$ contient tous les termes de le suite à partir d'un certain rang.

                Autrement dit, pour tout réel $m$, on peut trouver un rang $N$ tel que : \[\forall n\geq N,\quad u_n<m\]
                (A partir du rang $N$, tous les termes sont inférieurs à $m$)

                On note $\lim\limits_{n\to +\infty} u_n=-\infty$,\quad On peut dire que la suite diverge vers $-\infty$

                H.P. : $\forall m\in\mathbb{R},\ \exists N\in\mathbb{N},\ \forall n\in\mathbb{N},\ (n\geq N\implies u_n<m)$
            \end{SSpartie}
            \begin{SSpartie}{Théorème}
                \begin{center}$\begin{array}{cc}
                    \lim\limits_{n\to +\infty} n^2=+\infty & \lim\limits_{n\to +\infty} \sqrt{n}=+\infty \\
    
                    \lim\limits_{n\to +\infty} \ln(n)=+\infty & p\in\mathbb{N^*},\ \lim\limits_{n\to +\infty} n^p=+\infty \\
                    \lim\limits+{n\to +\infty} -n^2=-\infty & \lim\limits_{n\to +\infty} \ln\left(\frac{1}{n}\right)=-\infty
                \end{array}$\end{center}
            \end{SSpartie}
            \begin{SSpartie}{Rappels} 
                $\forall n\in\mathbb{N},\ u_{n+1}-u_n=\dotsc$ signe

                $\forall n\in\mathbb{N},\ u_n>0\quad\frac{u_{n+1}}{u_n}$ on compare à 1

                Cas où $u_n=f(n)$, par exemple : $u_n=\sqrt{n^2+n-3}$, on étudie $f$.
                \begin{SSSpartie}{Arithmétique} 
                    $u_{n+1}=u_n+r$\quad alors\quad $u_n=u_0+nr$\quad\big($u_n=u_p+(n-p)r$\big)

                    $S=u_0+u_1+\dotsb+u_n=\frac{u_0+u_n}{2}\times(n+1)$
                \end{SSSpartie}
                \begin{SSSpartie}{Géométrique} 
                    $u_{n+1}=qu_n$\quad alors\quad $u_n=u_0\times q^n$\quad\big($u_n=u_p\times q^{n-p}$\big)

                    $S=u_0+u_1+\dotsb+u_n=u_0\times\frac{1-q^{n+1}}{1-q}$
                \end{SSSpartie}
            \end{SSpartie}
        \end{Spartie}
        \pagebreak
        \begin{Spartie}{Limites Finies / Suites Convergentes} 
            \begin{SSpartie}{Définition} 
                Une suite ($u_n$) converge vers un réel $l$ si tout intervalle ouvert contenant $l$ contient tous les termes de la suite à partir d'un certain rang.

                Autrement dit, on peut trouver un rang $N$ à partir duquel tous les termes de la suite sont aussi près que l'on veut de $l$.

                On dit que $l$ est la limite de la suite ($u_n$) et que la suite est convergente.

                On note $\lim\limits_{n\to +\infty} u_n=l$

                H.P. : $\lim\limits_{n\to +\infty} u_n=l\iff\forall\epsilon >0,\ \exists N\in\mathbb{N},\ \forall n\in\mathbb{N}, \Big(n\geq N\implies u_n\in\big]l-\epsilon\,;l+\epsilon\big[\Big)$
            \end{SSpartie}
            \begin{SSpartie}{Théorème} 
                \begin{center}$\begin{array}{cc}
                    \lim\limits_{n\to +\infty}\frac{1}{n}=0 & p\in\mathbb{N^*},\ \lim\limits_{n\to +\infty}\frac{1}{n^p}=0 \\
                    \lim\limits_{n\to +\infty}e^{-n}=0 & \lim\limits_{n\to +\infty}\frac{1}{\sqrt{n}}=0
                \end{array}$\end{center}
            \end{SSpartie}
            \begin{SSpartie}{Définition} 
                Une suite qui n'est pas convergente est divergente
            \end{SSpartie}
            \begin{SSpartie}{Exemple} 
                La suite ($u_n$) définie par $u_n=(-1)^n$ est divergente.
            \end{SSpartie}
        \end{Spartie}
    \end{Gpartie}
    \pagebreak
    \begin{Gpartie}{Propriétés sur les Limites} 
        \begin{Spartie}{Théorèmes de Comparaison} 
            \begin{SSpartie}{Théorème} 
                Soient $(u_n)$ et $(v_n)$ deux suites telles qu'à partir d'un certain rang $u_n\leq v_n$
                \begin{itemize}
                    \setlength\itemsep{0.5em}
                    \item Si $\lim\limits_{n\to +\infty}u_n=+\infty$\quad alors\quad $\lim\limits_{n\to +\infty}v_n=+\infty$
                    \item Si $\lim\limits_{n\to +\infty}v_n=-\infty$\quad alors\quad $\lim\limits_{n\to +\infty}u_n=-\infty$
                \end{itemize}
            \end{SSpartie}
            \begin{SSpartie}{Démonstration} 
                On suppose que $\lim\limits_{n\to +\infty}u_n=+\infty$

                Soit $M>0$, il existe un rang $N$, à partir duquel si $n\geq N_1$, alors $u_n\geq M$

                Or, il existe un rang $N_2$, à partir duquel $n>N_2,\ v_n\geq u_n$

                Donc, il existe un rang $N=\max(N_1\,;N_2)$ à partir duquel $v_n\geq u_n>M$ donc: \[\lim\limits_{n\to +\infty}v_n=+\infty\quad\square\]
            \end{SSpartie}
            \begin{SSpartie}{Théorème dit des Gendarmes (Théorème d'Encadrement)}
                Soient $(u_n),\ (v_n),$ et $(w_n)$ trois suites telles qu'à partir d'un certain rang $u_n\leq v_n\leq w_n$

                Si $\lim\limits_{n\to +\infty}u_n=l$ et $\lim\limits_{n\to +\infty}w_n=l$ où $l$ est un réel alors :
                \[\lim\limits_{n\to +\infty}v_n=l\]
            \end{SSpartie}
            \begin{SSpartie}{Exemple} 
                Soit $(u_n)$ définie par $u_n=\frac{(-1)^n}{n}$

                $\begin{aligned}[t]
                    \forall n\in\mathbb{N},&\quad-1\leq (-1)^n\leq 1 \\
                    \iff&\quad\frac{-1}{n}\leq\frac{(-1)^n}{n}\leq\frac{1}{n}
                \end{aligned}$ \\[2ex]
                Or, $\lim\limits_{n\to +\infty}\frac{-1}{n}=0,\ \lim\limits_{n\to +\infty}\frac{1}{n}=0$

                D'après le Théorème des Gendarmes, $\lim\limits_{n\to +\infty}u_n=0$
            \end{SSpartie}
        \end{Spartie}
        \pagebreak
        \begin{Spartie}{Convergence Monotone} 
            \begin{SSpartie}{Théorème Admis} 
                Toute suite croissante et majorée converge vers une limite finie.

                Toute suite décroissante et minorée converge vers une limite finie.
            \end{SSpartie}
            \begin{SSpartie}{Théorème} 
                Soit une suite $(u_n)$ croissante et qui converge vers un réel $l$, alors $(u_n)$ est majorée par $l$.
            \end{SSpartie}
            \begin{SSpartie}{Démonstration par l'Absurde} 
                On suppose que $(u_n)$ est croissante.
                \begin{SSSpartie}{Lemme} 
                    Si $(u_n)$ est croissante, si $p$ et $n$ sont deux entiers naturels tels que $p\leq n$

                    Alors, $u_p\leq u_n$
                \end{SSSpartie}
                \begin{SSSpartie}{Démonstration par Récurrence}
                    \begin{SSSSpartie}{Initialisation}
                        On fixe $p$ donc $u_p\leq u_{p+1}$
                    \end{SSSSpartie}
                    \begin{SSSSpartie}{Hérédité}
                        On suppose que $k\geq 1,\ u_p\leq u_{p+k}$
                        $u_p\leq u_{p+k}\implies u_p\leq u_{p+k}\leq u_{p+k+1}\quad\square$
                    \end{SSSSpartie}    
                \end{SSSpartie}
                On suppose $\exists n_0\in\mathbb{N},\ u_{n_0}>l$

                Or tout intervalle ouvert contenant $l$ contient tous les termes de la suite à partir d'un certain rang.

                Prenons l'intervalle ouvert $\big]a\,;b\big[$ tel que $l<b<u_{n_0}$

                Il existe un indice $p>n_0$ tel que $u_p\in\big]a\,;b\big[$ \\ (Ils y sont tous à partir d'un certain rang !)

                Donc, $u_p<b<u_{n_0}$, ce qui impossible car $(u_n)$ est croissante et d'après la lemme, $p>n_0\implies u_p\geq u_{n_0}$

                C'est absurde, donc, $\forall n\in\mathbb{N},\ u_n\leq l\quad\square$
            \end{SSpartie}
            \begin{SSpartie}{Théorème} 
                Toute suite croissante et non-majorée diverge vers $+\infty$

                Toute suite décroissante et non-minorée diverge vers $-\infty$
            \end{SSpartie}
        \end{Spartie}
        \begin{Spartie}{Rappel : Limite de $\left(q^n\right)$ où $q\in\mathbb{R}$} 
            \begin{SSpartie}{Théorème} 
                \begin{itemize}
                    \setlength\itemsep{0.5em}
                    \item Si $q>1,\ \lim\limits_{n\to +\infty}q^n=+\infty$
                    \item Si $q=1,\ \lim\limits_{n\to +\infty}q^n=1$
                    \item Si $-1<q<1,\ \lim\limits_{n\to +\infty}q^n=0$
                    \item Si $q<-1$, la suite diverge.
                \end{itemize}
            \end{SSpartie}
        \end{Spartie}
    \end{Gpartie}
    \begin{Gpartie}{Opérations sur les Limites} 
        On considère les suites $(u_n)_{n\in\mathbb{N}}$ et $(v_n)_{n\in\mathbb{N}}$ admettant des limites finies ou infinies.
        \begin{Spartie}{Somme}
            \begin{center}\begin{tabular}{ |p{0.15\textwidth}||w{c}{0.1\textwidth}|w{c}{0.1\textwidth}|w{c}{0.1\textwidth}| } \hline
                \diagbox[innerwidth=0.15\textwidth]{$v_n$}{$u_n$}   & $l$ un réel   & $+\infty$ & $-\infty$ \\ \hline \hline
                $l'$ un réel                                        & $l+l'$        & $+\infty$ & $-\infty$ \\ \hline
                $+\infty$                                           & $+\infty$     & $+\infty$ & F.I.\footnotemark[1] \\ \hline
                $-\infty$                                           & $-\infty$     & F.I.\footnotemark[1] & $-\infty$ \\ \hline
            \end{tabular}\end{center}
            \parbox{\linewidth}{\captionof{figure}{Tableau des Limites des Sommes de Suites, de Limites Données}} \\[2ex]
            F.I. : Forme Indéterminée, il faut faire un calcul pour lever l'indétermination
            \pagebreak
            \begin{SSpartie}{Exemple} 
                $(u_n)_{n\in\mathbb{N}}$ définie par $\begin{aligned}[t]u_n&=n^2-n\leftarrow\text{F.I.} \\ &=n(n-1)\leftarrow\text{On a levé l'indétermination}\end{aligned}$

                $\underbrace{\lim\limits_{n\to +\infty}n^2=+\infty\quad\lim\limits_{n\to +\infty}-n=-\infty}_{F.I.}$
                \qquad$\lim\limits_{n\to +\infty}n=+\infty\quad\lim\limits_{n\to +\infty}n-1=+\infty$

                Donc, par produit, $\lim\limits_{n\to +\infty}n(n+1)=+\infty$ \\ et donc, $\lim\limits_{n\to +\infty}u_n=+\infty$
            \end{SSpartie}
        \end{Spartie}
        \begin{Spartie}{Produit}
            \begin{center}\begin{tabular}{ |p{0.15\textwidth}||w{c}{0.15\textwidth}|w{c}{0.15\textwidth}|w{c}{0.15\textwidth}|w{c}{0.15\textwidth}| } \hline
                \diagbox[innerwidth=0.15\textwidth]{$v_n$}{$u_n$}   & $l\neq 0$ & 0 & $+\infty$ & $-\infty$ \\ \hline\hline
                $l'\neq 0$                                          & $l\times l'$ & 0 & $\pm\infty\text{\tiny{ selon signe }}l'$ & $\pm\infty\text{\tiny{ selon signe }}l'$ \\ \hline
                0                                                   & 0 & 0 & F.I. & F.I. \\ \hline
                $+\infty$                                           & $\pm\infty\text{\tiny{ selon signe }}l$ & F.I. & $+\infty$ & $-\infty$ \\ \hline
                $-\infty$                                           & $\pm\infty\text{\tiny{ selon signe }}l$ & F.I. & $-\infty$ & $-\infty$ \\ \hline
            \end{tabular}\end{center}
            \parbox{\linewidth}{\captionof{figure}{Tableau des Limites des Produits de Suites, de Limites Données}}
            \begin{SSpartie}{Exemple} 
                Soit $(u_n)_{n\in\mathbb{N}}$, définie par $u_n=n\quad\lim\limits_{n\to +\infty}u_n=+\infty$

                Soit $(v_n)_{n\in\mathbb{N}}$, définie par $v_n=\frac{1}{\sqrt{n}}\quad\lim\limits_{n\to +\infty}v_n=0$

                $u_n\times v_n=n\times\frac{1}{\sqrt{n}}=\sqrt{n}\quad\text{or}\quad\lim\limits_{n\to +\infty}\sqrt{n}=+\infty$

                Donc, $\lim\limits_{n\to +\infty}(u_n\times v_n)=+\infty$
            \end{SSpartie}
            \begin{SSpartie}{Exemple 2} 
                $(u_n)_{n\in\mathbb{N}}$, définie par $u_n=n^2\quad\lim\limits_{n\to +\infty}u_n=+\infty$

                $(v_n)_{n\in\mathbb{N}}$, définie par $v_n=\frac{1}{n}-4\quad\lim\limits_{n\to +\infty}v_n=-4$ (Somme)

                Donc, $\lim\limits_{n\to +\infty}(u_n\times v_n)=-\infty$
            \end{SSpartie}
        \end{Spartie}
        \pagebreak
        \begin{Spartie}{Quotient}
            \begin{center}\begin{tabular}{ |p{0.15\textwidth}||w{c}{0.15\textwidth}|w{c}{0.15\textwidth}|w{c}{0.15\textwidth}|w{c}{0.15\textwidth}| } \hline
                \diagbox[innerwidth=0.15\textwidth]{$v_n$}{$u_n$}   & $l\neq 0$ & 0 & $+\infty$ & $-\infty$ \\ \hline\hline
                $l'\neq 0$                                          & $\frac{l}{l'}$ & 0 & $\pm\infty\text{\tiny{ selon signe }}l'$ & $\pm\infty\text{\tiny{ selon signe }}l'$ \\ \hline
                0                                                   & $\pm\infty\text{\tiny{ selon signe }}l\text{\tiny{ et }}0$ & F.I. & $\pm\infty\text{\tiny{ selon signe }}0$ & $\pm\infty\text{\tiny{ selon signe }}0$ \\ \hline
                $+\infty$                                           & 0 & 0 & F.I. & F.I. \\ \hline
                $-\infty$                                           & 0 & 0 & F.I. & F.I. \\ \hline
            \end{tabular}\end{center}
            \parbox{\linewidth}{\captionof{figure}{Tableau des Limites des Quotients de Suites, de Limites Données}} 
            \\[2ex]
            \begin{SSpartie}{Exemple} 
                $(u_n)_{n\in\mathbb{N}}$, définie par $u_n=\frac{1}{n^2-3}\quad\lim\limits_{n\to +\infty}u_n=+\infty$ (Somme)

                Donc, par quotient, $\lim\limits_{n\to +\infty}u_n=0$
            \end{SSpartie}
        \end{Spartie}
    \end{Gpartie}
    \begin{Gpartie}{Limite de Suite et Continuité} 
        \begin{Spartie}{Théorème} 
            Soit $f$ une fonction continue sur un intervalle $I$ et $(u_n)_{n\in\ I}$, une suite qui converge vers un réel $l$, tel que $\forall n\in\mathbb{N},\ u_n\leq I,\ l\in I$, alors $f(u_n)$ converge vers $f(l)$.
            \begin{SSpartie}{Exemple} 
                Soit le suite $(u_n)$, définie par $u_n=\frac{4n}{n+1}$. Alors $\lim\limits_{n\to +\infty}u_n=4$
                
                Donc, la suite $(v_n)$, définie par $v_n=\sqrt{u_n}$ converge vers $\sqrt{4}=2$.
            \end{SSpartie}
        \end{Spartie}
        \begin{Spartie}{Théorème}
            Soit $f$ une fonction continue sur un intervalle $I$, telle que $f(I)\subset I$ et $(u_n)_{n\in\mathbb{N}}$ une suite définie par $u_{n+1}=f(u_n)$ et $u_0\in I$
            
            Si la suite $(u_n)$ converge vers un réel $l$, alors $l$ est solution de l'équation $f(x)=x$ (On peut dire que $l$ est un point fixe de $f$).
            \begin{SSpartie}{Démonstration} 
                $\lim\limits_{n\to +\infty}u_{n+1}=\lim\limits_{n\to +\infty}u_n=f(l)$
            \end{SSpartie}
        \end{Spartie}
    \end{Gpartie}
\end{document}