\documentclass{cours}

\title{Dénombrement et Combinatoire}
\author{Diego Van Overberghe}

\begin{document}
    \maketitle{3}

    \begin{Gpartie}{Parties d'un Ensemble}
        \begin{Spartie}{Définition}
            $E$ étant un ensemble, la notation $\subset$ signifie ``Inclus Dans''.
            \[A\subset E\]
            C'est-à-dire que tout élément de $A$ appartient à $E$. \\
            On dit alors que $A$ est une partie, ou sous-ensemble de $E$. \\
            L'Ensemble vide, noté $\varnothing$ est une partie de tout ensemble. \\
            L'Ensemble des parties de $E$ est noté $\mathcal{P}(E)$.
            \begin{SSpartie}{Exemple}
                Si $E=\big\{x\,;y\big\}$ est un ensemble de deux éléments :
                \[\mathcal{P}(E)=\Big\{\varnothing\,,\big\{x\big\}\,,\big\{y\big\}\,,\big\{x\,;y\big\}\Big\}\]
            \end{SSpartie}
        \end{Spartie}
        \begin{Spartie}{Définition}
            Un ensemble fini est un ensemble dont le nombre d'éléments est fini.
        \end{Spartie}
        \begin{Spartie}{Définition}
            On appelle cardinal, noté ``Card'', le nombre d'éléments d'un ensemble fini ou d'une partie / sous-ensemble.
            \begin{SSpartie}{Exemple}
                Si un ensemble $E$ possède $n$ éléments, alors on peut noter $\text{Card}(E)=n$. \\
                Pour toute partie $A\subset E$, $\text{Card}(A)\leq\text{Card}(E)$.
            \end{SSpartie}
        \end{Spartie}
        \begin{Spartie}{Propriété (Principe Additif)}
            Si $A$ et $B$ sont deux parties quelconques d'un ensemble fini, $E$, alors :
            \[\text{Card}(A\cup B)=\text{Card}(A)+\text{Card}(B)-\text{Card}(A\cap B)\]
            De plus, si $A_1, A_2,\dotsc, A_p$ sont $p$ parties deux à deux disjointes d'un ensemble fini, alors :
            \[\text{Card}(A_1\cup A_2\cup\dotsb\cup A_p)=\text{Card}(A_1)+\text{Card}(A_2)+\dotsb+\text{Card}(A_p)\]
        \end{Spartie}
        \begin{Spartie}{Propriété}
            Soit $n\in\mathbb{N}$, et $E$ un ensemble tel que $\text{Card}(E)=n$. Alors, $E$ possède $2^n$ parties. Autrement dit :
            \[\text{Card}(\mathcal{P}(E))=2^n\]
            \begin{SSpartie}{Démonstration par Récurrence}
                \begin{SSSpartie}{Initialisation}
                    Pour $n=0$, $E=\varnothing$, donc la seule partie de $E$ est $\big\{\varnothing\big\}$ et $1=2^0$.
                \end{SSSpartie}
                \begin{SSSpartie}{Hérédité}
                    Supposons que tout ensemble à $k$ élément, où $k$ est un certain entier naturel, admet $2^k$ parties.

                    Alors, soit $E$, un ensemble à $k+1$ éléments.

                    Soit $x$, un élément de $E$.

                    Alors, il y a deux familles de parties de $E$, celles qui contiennent $x$ et celles qui ne le contiennent pas.

                    Or $E\setminus\big\{x\big\}$ est un ensemble à $k$ éléments, il y a donc $2^k$ parties de $E$ qui ne contiennent pas $x$.

                    En adjoignant $x$ à toutes ces parties, on obtient toutes les parties qui contiennent $x$, donc il y en a également $2^k$.
                    
                    Ainsi, le nombre de parties de $E$ est de $2^k+2^k=2^{k+1}\quad\square$
                \end{SSSpartie}
            \end{SSpartie}
        \end{Spartie}
    \end{Gpartie}
    \pagebreak
    \begin{Gpartie}{Produit Cartésien d'Ensembles}
        \begin{Spartie}{Définition}
            \begin{itemize}
                \item $E$ et $F$ étant deux ensembles, le produit cartésien $E\times F$ est l'ensemble de couples $(a\,;b)$, où $a\in E$ et $b\in F$.
                \item $E$, $F$ et $G$ étant trois ensembles, le produit cartésien $E\times F\times G$ est l'ensemble des triplets $(a\,;b\,;c)$ où $a\in E$, $b\in F$ et $c\in G$.
            \end{itemize}
            \begin{SSpartie}{Cas Général}
                \begin{itemize}
                    \item Le produit cartésien $E_1\times E_2\times\dotsb\times E_n$ des ensembles $E_1,\,E_2,\,\dotsc,\,E_n$ est l'ensemble des n-uplets $(a_1\,;a_2\,;\dotsc\,;a_n)$ où $a_1\in E_1,\,a_2\in E_2,\,\dotsc,\,a_n\in E_n$.
                \end{itemize}
            \end{SSpartie}
        \end{Spartie}
        \begin{Spartie}{Notations}
            $E\times E$ se note $E^2$, $\overbrace{E\times E\times\dotsb\times E}^\text{$k$ fois}$ se note $E^k$.
        \end{Spartie}
        \begin{Spartie}{Exemples}
            Soit $E=\big\{a\,;b\,;c\big\}$ et $F=\big\{1\,;2\big\}$
            \begin{itemize}
                \item $E\times F=\big\{(a\,;1)\,;(a\,;2)\,;(b\,;1)\,;(b\,;2)\,;(c\,;1)\,;(c\,;2)\big\}$
                \item $F\times F=\big\{(1\,;1)\,;(1\,;2)\,;(2\,;1)\,;(2\,;2)\big\}$
                \item $(a\,;b\,;b\,;a\,;c)$ est un 5-uplet d'élément de $E$, il appartient à $E^5$
            \end{itemize}
        \end{Spartie}
        \begin{Spartie}{Propriété}
            Si $E_1,\,E_2,\,\dotsc,\,E_n$ sont $n$ ensembles finis :
            \[\text{Card}(E_1\times E_2\times\dotsb\times E_n)=\text{Card}(E_1)\times\text{Card}(E_2)\times\dotsb\times\text{Card}(E_n)\]
            \begin{SSpartie}{Cas Particulier}
                Si $E$ est un ensemble fini, pour tout $k\in\mathbb{N^*}$ :
                \[\text{Card}(E^k)=\big(\text{Card}(E)\big)^k\]
            \end{SSpartie}
            \pagebreak
            \begin{SSpartie}{Exemples}
                Dans les exemples précédents :
                \begin{itemize}
                    \item $\text{Card}(E\times F)=6=\text{Card}(E)\times\text{Card}(F)$
                    \item $\text{Card}(F^2)=4=2^2=\big(\text{Card}(F)\big)^2$
                \end{itemize}
            \end{SSpartie}
        \end{Spartie}
    \end{Gpartie}
    \pagebreak[3]
    \begin{Gpartie}{Permutations}
        \begin{Spartie}{Définition}
            Soit $E$, un ensemble à $n$ éléments, une permutation est un n-uplet d'éléments distincts de $E$.

            Autrement dit, une permutation est une façon d'ordonner les $n$ éléments de $E$.
            \begin{SSpartie}{Exemple}
                On considère l'ensemble $G=\big\{a\,;b\,;c\big\}$ \\ Ses permutations sont :
                \[(a\,;b\,;c),\,(a\,;c\,;b),\,(b\,;a\,;c),\,(b\,;c\,;a),\,(c\,;a\,;b),\,(c\,;b\,;a)\]
                $G$ admet donc 6 permutations.
            \end{SSpartie}
        \end{Spartie}
        \begin{Spartie}{Propriété}
            Le nombre de permutations d'un ensemble à $n$ éléments est $n!$
            \begin{SSpartie}{Remarque}
                $n! =n\times(n-1)\times\dotsb\times 2\times 1$ \\
                $n!$ est le produit de tous les entiers de $1$ à $n$. \\
                $n!$ se lit ``factorielle de $n$''.
            \end{SSpartie}
            \begin{SSpartie}{Explication}
                On peut considérer que faire une permutation c'est faire un tirage sans remise des $n$ éléments de $E$. Il y $n$ choix pour le premier élément, $n-1$ pour le deuxième et ainsi de suite.
            \end{SSpartie}
            \pagebreak
            \begin{SSpartie}{Démonstration par Récurrence}
                \begin{SSSpartie}{Initialisation}
                    Un ensemble à un élément admet une permutation, et $1! =1$.
                \end{SSSpartie}
                \begin{SSSpartie}{Hérédité}
                    Supposons que tout ensemble à $n$ élément ($n$ fixé) admette $n!$ permutations.

                    Soit $E$, un ensemble à $n+1$ éléments. \\
                    On choisit un élément $x$ de $E$. \\
                    Dans chacune des permutations des $n$ éléments restants, il y $n+1$ positions où insérer $x$.

                    Ainsi, le nombre de permutations de $E$ est $(n+1)\times n! =(n+1)!\quad\square$
                \end{SSSpartie}
            \end{SSpartie}
        \end{Spartie}
    \end{Gpartie}
    \pagebreak[3]
    \begin{Gpartie}{Combinaisons}
        Dans tout ce sous-chapitre, $E$ est un ensemble à $n$ éléments et p est un entier naturel tel que $p\leq n$.
        \begin{Spartie}{Définition}
            Une combinaison de $p$ éléments de $E$ est une partie de $E$ possédant $p$ éléments.
            \begin{SSpartie}{Remarque}
                L'Ordre des éléments n'a pas d'importance, les éléments sont distincts.
            \end{SSpartie}
        \end{Spartie}
        \begin{Spartie}{Propriété}
            Le nombre de combinaisons à $p$ éléments de $E$ est égal à $\binom{n}{p}$, où :
            \[\binom{n}{p}=\dfrac{n!}{p!(n-p)!}\]
            \[\binom{n}{p}=\dfrac{n(n-1)\times\dotsb\times(n-p+1)}{p!}\]
            $\dbinom{n}{p}$ est appelé coefficient binomial et il se lit ``$p$ parmi $n$''.
            \begin{SSpartie}{Explication}
                Lorsqu'on choisit $p$ éléments dans un ensemble à $n$ éléments, on a $n$ choix pour le premier, $n-1$ choix pour le deuxième, etc.\ mais ainsi, les $p$ éléments sont ordonnés.

                On divise donc par le nombre de permutations de $p$ éléments, c'est-à-dire $p!$
            \end{SSpartie}
            \begin{SSpartie}{Cas Particuliers}
                $\dbinom{n}{0}=1\quad$ La seule partie de $E$ à $0$ élément est $\varnothing$.

                $\dbinom{n}{n}=1\quad$ La seule partie de $E$ à $n$ éléments est $E$.

                $\dbinom{n}{1}=n\quad$ Il y à $n$ parties de $E$ à $1$ élément.
            \end{SSpartie}
        \end{Spartie}
        \begin{Spartie}{Propriété : Symétrie}
            Choisir $p$, c'est ne pas choisir $n-p$ :
            \[\binom{n}{n-p}=\binom{n}{p}\quad\]
            Démonstration Alternative : 
            \[\binom{n}{n-p}=\dfrac{n!}{(n-p)!(n-(n-p))!}=\dfrac{n!}{(n-p)!p!}=\binom{n}{p}\]
        \end{Spartie}
        \begin{Spartie}{Propriété : Relation de Pascal}
            \[\binom{n+1}{p+1}=\binom{n}{p}+\binom{n}{p+1}\]
            \begin{SSpartie}{Démonstration}
                Soit $E$, un ensemble à $n+1$ éléments (on va compter le nombre de parties de E à $p+1$ éléments).

                Soit $x$ un élément de $E$.

                Alors il y a deux ``familles'' de parties : celles qui contiennent $x$ et celles qui ne le contiennent pas.

                Or $E\setminus\big\{x\big\}$ contient $n$ éléments.

                Donc il y a $\binom{n}{p}$ à $p$ éléments de $E\setminus\big\{x\big\}$.

                En leur adjoignant $x$, on obtient toutes les parties à $p+1$ éléments qui contiennent $x$.

                Il y a $\binom{n}{p+1}$ parties de $E$ qui ne contiennent pas $x$. (On choisit $p+1$ éléments dans $E\setminus\{x\}$ qui contient $n$ éléments).
            \end{SSpartie}
        \end{Spartie}
        \begin{Spartie}{Triangle de Pascal}
            \begin{center}
                % \begin{tikzpicture}
                    \includegraphics{example-image}
                    \parbox{\linewidth}{\captionof{figure}{Représentation du Triangle de Pascal}}
                % \end{tikzpicture}
            \end{center}
        \end{Spartie}
        \begin{Spartie}{Propriété}
            \[\sum_{p=0}^{n}\binom{n}{p}=\binom{n}{0}+\binom{n}{1}+\dotsb+\binom{n}{n}=2^n\]
        \end{Spartie}
    \end{Gpartie}
\end{document}