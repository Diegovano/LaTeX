\documentclass{cours}

\title{Orthogonalité dans l'Espace}

\begin{document}
    \maketitle{12}

    \begin{Gpartie}{Différentes Expressions du Produit Scalaire} 
        \begin{Spartie}{Définitions} 
            $\vec{u}$ et $\vec{v}$ étant deux vecteurs de l'espace, soit un point $A$ de l'espace, et les points $B$ et $C$ tels que $\overrightarrow{AB}=\vec{u}$ et $\overrightarrow{AC}=\vec{v}$.

            Alors, $\vec{u}\cdot\vec{v}=\overrightarrow{AB}\cdot\overrightarrow{AC}$\quad(Produit Scalaire de deux vecteurs dans le plan (ABC))

            Ainsi, toutes les définitions et propriétés du produit scalaire dans le plan sont conservées dans l'espace.

            Les trois définitions suivantes sont équivalentes :
            \begin{SSpartie}{Définition avec le Projeté Orthogonal} 
                $\vec{u}$ et $\vec{v}$ étant deux vecteurs de l'espace, soit un point $A$ de l'espace, et les points $B$ et $C$ tels que $\overrightarrow{AB}=\vec{u}$ et $\overrightarrow{AC}=\vec{v}$.

                Soit $H$ le projeté orthogonal de $C$ sur $(AB)$,

                $\overrightarrow{AB}\cdot\overrightarrow{AC}=AB\times AH$ si $\overrightarrow{AB}$ et $\overrightarrow{AH}$ sont de même sens. \\
                $\overrightarrow{AB}\cdot\overrightarrow{AC}=-AB\times AH$ si $\overrightarrow{AB}$ et $\overrightarrow{AH}$ sont de sens contraire.
                \begin{center}
                    % \begin{tikzpicture}
                        \includegraphics[width=10cm]{example-image}
                    % \end{tikzpicture}
                    \parbox{\linewidth}{\captionof{figure}{\centering Illustration de la Définition}}
                \end{center}
            \end{SSpartie}
            \begin{SSpartie}{Définitions avec les Normes} 
                $\vec{u}\cdot\vec{v}=\dfrac{1}{2}\left(\lvert\lvert\vec{u}+\vec{v}\rvert\rvert^2-\lvert\lvert\vec{u}\rvert\rvert^2-\lvert\lvert\vec{v}\rvert\rvert^2\right)=\dfrac{1}{2}\left(\lvert\lvert\vec{u}\rvert\rvert^2+\lvert\lvert\vec{v}\rvert\rvert^2-\lvert\lvert\vec{u}-\vec{v}\rvert\rvert^2\right)$

                Ainsi, $\overrightarrow{AB}\cdot\overrightarrow{AC}=\dfrac{1}{2}\left(AB^2+AC^2-CB^2\right)$
            \end{SSpartie}
            \begin{SSpartie}{Définition avec l'Angle} 
                $\vec{u}\cdot\vec{v}=\lvert\lvert\vec{u}\rvert\rvert\times\lvert\lvert\vec{v}\rvert\rvert\times\cos\left(\vec{u}~,\vec{v}\right)$
            \end{SSpartie}
        \end{Spartie}
        \begin{Spartie}{Propriétés} 
            Pour tous vecteurs $\vec{u}$, $\vec{v}$ et $\vec{w}$, et pour tout réel $\lambda$ : \[\vec{u}\cdot\vec{v}=\vec{v}\cdot\vec{u}\qquad\text{(symmétrie du produit scalaire)}\] \[\begin{drcases}\left(\lambda\vec{u}\right)\cdot\vec{v}=\lambda\times\left(\vec{u}\cdot\vec{v}\right)=\vec{u}\cdot\left(\lambda\vec{v}\right) \\ \vec{u}\cdot\left(\vec{v}+\vec{w}\right)=\vec{u}\cdot\vec{v}+\vec{u}\cdot\vec{w}\end{drcases}\text{(bilinéarité du produit scalaire)}\]
        \end{Spartie}
        \begin{Spartie}{Expressions Analytique du Produit Scalaire} 
            Dans un repère orthonormé de l'espace, $\vec{u}~(x~;y~;z)$ et $\vec{v}~(x'~;y'~;~z')$ étant deux vecteurs, alors : \[\vec{u}\cdot\vec{v}=xx'+yy'+zz'\] 
            \begin{SSpartie}{Démonstration} 
                On rappelle que $\lvert\lvert\vec{u}\rvert\rvert=\sqrt{x^2+y^2+z^2}$

                $\begin{aligned}[t]
                    \vec{u}\cdot\vec{v}&=\dfrac{1}{2}\left(\lvert\lvert\vec{u}+\vec{v}\rvert\rvert^2-\lvert\lvert\vec{u}\rvert\rvert^2-\lvert\lvert\vec{v}\rvert\rvert^2\right) \\
                    &=\frac{1}{2}\left(\left(x+x'\right)^2+\left(y+y'\right)^2+\left(z+z'\right)^2-\left(x^2+y^2+z^2\right)-\left(x'^2+y'^2+z'^2\right)\right) \\
                    &=\frac{1}{2}\left(2xx'+2yy'+2zz'\right) \\
                    &=xx'+y y'+z z'
                \end{aligned}$
            \end{SSpartie}
        \end{Spartie}
    \end{Gpartie}
\end{document}