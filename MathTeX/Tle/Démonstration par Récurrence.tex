\documentclass[12pt, a4paper]{article}
\usepackage{amsmath}
\usepackage{amsfonts}
%\usepackage[upright]{fourier}
\usepackage[french]{babel}
\usepackage[T1]{fontenc}
\usepackage{fancyhdr}

\makeatletter
\renewcommand\maketitle{
{\raggedright
{\large \bfseries Chapitre 1}\\[4ex]
{\Large \bfseries \@title}\\[4ex]
\@date \\[8ex]}}
\makeatother
\setlength{\headheight}{15pt}


\pagestyle{fancy}
\fancyhf{}
\fancyhead[L]{12 Septembre 2020}
\fancyhead[R]{Spé Mathématiques}
\fancyfoot[C]{\thepage}

\begin{document}
    \title{\underline{Suites: La Démonstration par Récurrence}}
    \date{ 12 Septembre 2020}
    \maketitle

    \textbf{\underline{Exemple:}}\\[2ex]
    \hspace*{2\parindent}{\parbox[t]{\dimexpr\linewidth-2\parindent}{
        On considère la suite $\left(u_n\right)$, définie pour tout $n\in\mathbb{N}$, par $u_n=4^n-1$.
        \begin{center}
            \vspace{\baselineskip}
            $\left(u_n\right)=\left\{0\,;3\,;15\,;63\,;255\,;\dots\right\}$
            \vspace{\baselineskip}
        \end{center}
        On remarque que tous ces nombres sont des multiples de 3. On se demande si $\forall n\in\mathbb{N}$, $4^n-1$ est un multiple de 3.
    }} \\[4ex]
    \indent \textbf{\underline{Axiome de la récurrence:}}\\[2ex]
    \hspace*{2\parindent}{\parbox[t]{\dimexpr\linewidth-2\parindent}{
        Soit $P(n)$, une propriété dépendante de l'entier naturel $n$. \\[2ex]
        Si:
        $\begin{cases}
            P(0)\text{ est vraie (Initialisation)} \\ \forall k\in\mathbb{N}, P(k)\text{ vraie}\implies P(k+1)\text{ vraie (hérédité)}
        \end{cases}$ \\[2ex]

        Alors: $P(n)$ est vraie pour tout entier naturel
    }}\\[4ex]
    {\indent\parbox[t]{\dimexpr\linewidth-\parindent}{
        Reprenons notre exemple. La propriété: ``$4^n-1$ est un multiple de 3'', est vraie pour $n=0$. La propriété est donc initialisée.\\[2ex]
        Hérédité: Supposons que la propriété est vraie à un certain rang $k$, (fixe). C'est l'hypothèse de la récurrence, et montrons qu'alors, elle est vraie au rang $k+1$.\\[2ex]
        L'Hypothèse de récurrence est donc: ``$4^n-1$ est un multiple de 3''. C'est-à-dire qu'il existe un entier $p$ tel que $4^n-1=3p$ et donc $4^n=3p+1$ \pagebreak
    }}

    {\begin{center}
    $\begin{aligned}[t]
        4^{k+1}-1&=4\times 4^k-1 &\\
        &= 4\times\left(3p+1\right)-1 \quad \text{(Hypothèse de récurrence)} &\\
        &=12p+4-1 &\\
        &=12p+3 &\\
        &=3\left(4p+1\right) \quad \text{(Il s'agit bien d'un multiple de 3)}
    \end{aligned}$
    \end{center}} \vspace{\baselineskip}
    {\indent\parbox[t]{\dimexpr\linewidth-\parindent}{
    Donc, la propriété est héréditaire.

    On a démontré par récurrence que $\forall n\in\mathbb{N},4^n-1\implies$ multiple de 3.
    }} 
\end{document}}