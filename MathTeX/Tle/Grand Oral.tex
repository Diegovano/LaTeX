\documentclass[DIV=12]{scrartcl}
\usepackage[french]{babel}
\usepackage[T1]{fontenc}
\usepackage{newpxtext}
% \usepackage[scaled=.92]{helvet}
\usepackage[varbb]{newpxmath}
\setlength{\parskip}{0.5em}
\setlength{\parindent}{0pt}

\renewcommand{\thesection}{\Roman{section}}

\title{Comment Approcher Toute Fonction par un~Polynôme~?}
\author{Diego Van Overberghe}

\begin{document}
    \maketitle
    \section{Pourquoi avons nous besoin de faire l'approximation de fonctions par des polynômes ?}
    L'exemple en physique : approximation des fonctions trigonométriques sinus et tangente, par le polynôme $x$.
    \begin{center}
        \og Si $\theta<15^{\circ}$ On peut dire que $\tan\,(\theta)\approx\theta$ \fg{}\quad -- p. 467 Manuel Physique-Chimie
    \end{center}
    L'effet de remplacer les $\tan\,(\theta)$ par des $\theta$ permet d'alléger grandement la notation, et de plus facilite l'analyse, telle que le calcul de fonction dérivée, par exemple. 

    $\mathrm{N}\mathrm{B}$ : Il ne s'agit pas de simplement remplacer les fonctions entières par des simplifications polynomiales, car l'approximation n'est pas valable sur un intervalle large. On  utilise donc ce rapprochement uniquement si l'on analyse l'évolution d'une variable sur un domaine~restreint.

    Ceci mène naturellement à la question de comment est-ce qu'on développe des polynômes qui se rapprochent d'une fonction donnée, et pourquoi cette approximation n'est-elle que valable sur un petit intervalle ?

    \section{La recherche d'une approximation pour la fonction sinus}
    On cherche a faire l'approximation de la fonction sinus \emph{proche de zéro}.

    La fonction sinus est une fonction paire, sa représentation graphique est sinusoïdale (vague). 
    
    Commençons en cherchant quelques polynômes qui dont l'image en zéro est égal à l'image de zéro par la fonction sinus.
    \[\sin\,(0)=0\quad\text{ donc, on cherche }\quad f:0\mapsto 0\]
    Plusieurs candidats existent : $g:x\mapsto 0$ (la fonction constante) ou $h:x\mapsto x$, voire même $j:x\mapsto x^2+x$
    
    Pour sélectionner la meilleure fonction, on évalue la dérivée des candidats. Le nombre dérivé donne des informations sur la rapidité de l'évolution d'une fonction en un point.
    
    On cherche donc un polynôme dont le nombre dérivé en zéro est égal au nombre dérivé de la fonction sinus en zéro $\sin'(0)=1$. $g'(0)=0$ :
    
    \[h'(0)=1\quad\text{et}\quad j'(0)=1\]

    Nous avons donc éliminé la fonction constante. On va répéter ce processus, en cherchant une fonction dont la dérivée seconde en 0 $\sin''(0)=-\sin\,(0)=0$ :
    
    \[h''(0)=0\quad\text{et}\quad j''(0)=2\]

    Nous voyons donc qu'ici que $g:x\mapsto x$ semble être une bonne approximation de~$\sin\,(x)$ près de zéro. Cette fonction est paire, tout comme la fonction sinus.

    Si l'on voulait augmenter la précision de notre approximation, on prend le polynôme $x+\frac{-1}{6}x^3$.

    \section{Généralisation : La série de Taylor}
    On peut généraliser la technique utilisée auparavant pour approcher toute fonction. L'opération consiste en trouvant un polynôme dont les nombres dérivés succéssifs sont identiques à ceux de notre fonction cible. Par conséquence, la fonction que l'on souhaite approcher doit être \emph{infiniment dérivable}.

    La dérivation annule les termes constants d'une expression. On utilise cette propriété de la dérivation pour construire une fonction à partir de nombres dérivés succéssifs.

    Le polynôme résultant se nomme \og développement limité \fg{}.

    Soit $g$ une approximation de $f$, et $f^n$ la n-ième dérivée de $f$ : \[g(x)=\underbrace{f(a)+f'(a)x}_\text{tangente}+\frac{f''(a)x^2}{2}+\dotsb\] Ou, avec la notation sigma : \[g(x)=\sum_{n=0}^{\infty}\frac{f^n(a)x^n}{n!}\]
    Le $n!$ s'explique par les règles de dérivation des fonctions de type $x^n$ car la dérivée est $nx^{n-1}$. Si l'on dérive à nouveau on a $n(n-1)x^{n-2}$, d'où le factorielle pour compenser ces facteurs.

    Cette année, on a utilisé certaines méthodes telles que la méthode des rectangles ou la méthode de Monte-Carlo afin d'approcher l'aire sous une courbe. On peut utiliser le développement limité pour par exemple approcher des fonctions dont on ne peut pas exprimer de primitive avec des fonctions classiques, comme par exemple $x\mapsto e^{x^2}$. \\ Le développement limité d'ordre 2 serait : \[1+x^2\quad\text{d'où une primitive : }\quad x+\frac{x^3}{3}\]
\end{document}