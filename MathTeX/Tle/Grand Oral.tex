\documentclass[DIV=12]{scrartcl}
\usepackage[french]{babel}
\usepackage{newpxtext}
\usepackage[varbb]{newpxmath}
\setlength{\parskip}{0.5em}
\setlength{\parindent}{0pt}

\title{Comment Approcher Toute Fonction par un~Polynôme~?}
\author{Diego Van Overberghe}

\begin{document}
    \maketitle
    \section*{Pourquoi avons nous besoin de faire l'approximation de fonctions par des polynômes ?}
    L'exemple en physique : approximation des fonctions trigonométriques $\sin$ et $\tan$, par le polynôme $x$.
    \begin{center}
        \og Si $\uptheta<15^{\circ}$ On peut dire que $\tan\,(\uptheta)\approx\uptheta$ \fg{}\quad -- p. 467 Manuel Physique-Chimie
    \end{center}
    L'effet de remplacer les $\tan\,(\uptheta)$ par des $\uptheta$ permet d'alléger grandement la notation, et de plus facilite l'analyse, telle que le calcul de fonction dérivée, par exemple. 

    $\mathrm{N}\mathrm{B}$ : Il ne s'agit pas de simplement remplacer les fonctions entières par des simplifications polynomiales, car l'approximation n'est pas valable sur un intervalle large. On  utilise donc ces rapprochement uniquement si l'on analyse l'évolution d'une variable sur un domaine restreint.

    Ceci mène naturellement à la question de pourquoi et comment développe-t-on des polynômes qui se rapprochent d'une fonction donnée, et pourquoi cette approximation n'est elle valable que sur un petit intervalle ?

    \section*{La recherche d'une approximation pour la fonction sinus}
    On cherche a faire l'approximation de la fonction sinus proche de zéro.
    
    Commençons en cherchant quelques polynômes qui dont l'image en zéro est égal à l'image de zéro par la fonction sinus. 
    \[\sin\,(0)=0\quad\text{ donc, on cherche }\quad f:0\mapsto 0\]
    Plusieurs candidats existent : $g:x\mapsto 0$ (la fonction constante) ou $h:x\mapsto x$, voire même $j:x\mapsto x^2+x$
    
    Pour sélectionner la meilleure fonction, on évalue la dérivée des candidats. Le nombre dérivé donne des informations sur la rapidité de l'évolution d'une fonction en un point.
    
    On cherche donc un polynôme dont le nombre dérivé en zéro est égal au nombre dérivé de la fonction sinus en zéro $\sin'(0)=1$. $g'(0)=0$ :
    
    \[h'(0)=1\quad\text{et}\quad j'(0)=1\]

    Nous avons donc éliminé la fonction constante. On va répéter ce processus, en cherchant une fonction dont la dérivée seconde en 0 $\sin''(0)=-\sin\,(0)=0$ :
    
    \[h''(0)=0\quad\text{et}\quad j''(0)=2\]

    Nous voyons donc qu'ici que $g:x\mapsto x$ semble être une bonne approximation de~$\sin\,(x)$ près de zéro.
\end{document}