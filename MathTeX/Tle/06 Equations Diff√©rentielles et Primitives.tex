\documentclass{cours}

\title{Équations Différentielles et Primitives}

\begin{document}
    \maketitle{6}

    \begin{Gpartie}{Notion d'Équation Différentielle} 
        \begin{Spartie}{Définition} 
            Une équation liant une fonction et ses dérivées est appelée \emph{équation différentielle}. En général, on note $y$ la fonction, $y'$ sa dérivée, $y''$ sa dérivée seconde.
        \end{Spartie}
        \begin{Spartie}{Exemples} 
            \begin{itemize}
                \item $y'=2x$ est une équation différentielle définie sur $\mathbb{R}$ dont \textit{une} solution est~$y=x^2$.
                \item $y'=y$ est une équation différentielle définie sur $\mathbb{R}$ dont une solution est $\exp$.
            \end{itemize}
            \begin{SSpartie}{Remarque} 
                Ce ne sont pas les seules solutions !
            \end{SSpartie}
        \end{Spartie}
        \begin{Spartie}{Remarques} 
            \begin{enumerate}[(1)]
                \item L'équation est dite de \emph{premier ordre} ou \emph{d'ordre 1} lorsque seule intervient la dérivée première d'une fonction (et éventuellement la fonction). \\ Par exemple : $y'=3y-5$.
                \item Plutôt que d'écrire l'équation $f'(x)=3f(x)-5$, on note $f(x)$ à l'aide de la variable $y$, qui joue le rôle d'inconnue, ou plutôt de \og fonction inconnue \fg{}, car un point $(x\,; y)$ appartient à la courbe représentative de $f$ si et seulement si $y=f(x)$. \\ $y$ étant la variable utilisée pour les ordonnées et les images, il est cohérent de l'utiliser pour symboliser une fonction.
            \end{enumerate}
        \end{Spartie}
    \end{Gpartie}
    \begin{Gpartie}{Primitives d'une Fonction Continue \big(Solutions de $y'=f(x)$\big)} 
        \vspace{-5ex}
        \begin{Spartie}{Définition} 
            Soit $f$ une fonction continue sur un intervalle $I$. Une primitive de $f$ sur $I$ est une fonction $F$ telle que :
            \[F'=f\]
        \end{Spartie}
        \vspace{-5ex}
        \begin{Spartie}{Exemple} 
            La fonction $F:x\mapsto\frac{1}{x}$ est une primitive de $f:x\mapsto\frac{-1}{x^2}$, sur $\mathbb{R^{+*}}$.
        \end{Spartie}
        \vspace{-3ex}
        \begin{Spartie}{Théorème (Démontré dans le Chapitre sur l'Intégration)} 
            Toute fonction continue sur un intervalle $I$ admet une primitive sur $I$.
        \end{Spartie}
        \vspace{-3ex}
        \begin{Spartie}{Remarque} 
            Il arrive qu'on ne puisse pas exprimer cette primitive avec les fonctions classiques.
            \begin{SSpartie}{Exemple} 
                $f:x\mapsto \me^{x^2}$
            \end{SSpartie}
        \end{Spartie}
        \vspace{-5ex}
        \begin{Spartie}{Théorème} 
            Soit $f$ une fonction continue sur un intervalle $I$ et $F$ est une primitive de $f$ sur $I$.

            Alors, $f$ admet une infinité de primitives sur $I$\textsuperscript{\raisebox{.5pt}{\textcircled{\raisebox{-.9pt}{1}}}} et toute primitive $G$ de $f$ sur $I$ est définie par $G(x)=F(x)+k$\textsuperscript{\raisebox{.5pt}{\textcircled{\raisebox{-.9pt}{2}}}} où $k$ est une constante réelle.

            \begin{SSpartie}{Démonstration} 
                \begin{enumerate}\vspace*{-2ex}
                    \item Si $F$ est une primitive de $f$, quel que soit le réel $k$, la fonction $G:x\mapsto F(x)+k$ est une primitive de $f$. En effet, $G'(x)=F'(x)=f(x)$, donc, il y a une infinité de primitives.
                    \item Si $F$ et $G$ sont deux primitives d'une même fonction $f$, alors :
                    \[(F-G)'(x)=F'(x)-G'(x)=f(x)-f(x)=0\]
    
                    La dérivée de $F-G$ est nulle, donc $F-G$ est une constante.
    
                    Donc, 
                    $\begin{aligned}[t]
                        \forall x\in I,\ F(x)-G(x)&=k \\
                        F(x)&=G(x)+k\quad\square
                    \end{aligned}$
    
                    On a montré que deux primitives d'une même fonction sont différents d'une constante.
                \end{enumerate}
            \end{SSpartie}
        \end{Spartie}
    \end{Gpartie}
    \begin{Gpartie}{Calcul des Primitives} 
        \vspace{-5ex}
        \begin{Spartie}{Primitives des Fonctions usuelles} 
            \begin{table}[H] \centering \captionabove{Tableau des Primitives des Fonctions Usuelles}
                \begin{tabular}[c]{ @{}*{3}{C}@{} } \toprule
                    Fonction $f:x\mapsto$                       & Primitive $F:x\mapsto$                                            & Intervalle  \\ \midrule
                    $k$ (constante)                             & $kx$                                                              & $\mathbb{R}$ \\ 
                    $x^n$ ($n\in\mathbb{N}$)                    & $\frac{x^{n+1}}{n+1}$                                             & $\mathbb{R}$ \\ 
                    $\frac{1}{x^n}=x^{-n}$ ($n\in\mathbb{N}$)   & $-\frac{1}{n-1}\times\frac{1}{x^{n-1}}$ ou $\frac{x^{-n+1}}{-n+1}$& $\big]-\infty\,; 0\big[$ ou $\big]0\,; +\infty\big[$ \\ 
                    $\frac{1}{x}$                               & $\ln\,(x)$                                                          & $\mathbb{R^{+*}}$ \\ 
                    $\frac{1}{\sqrt{x}}$                        & $2\sqrt{x}$                                                       & $\mathbb{R^{+*}}$ \\ 
                    $\me^x$                                       & $\me^x$                                                             & $\mathbb{R}$ \\ 
                    $\sin\,(x)$                                   & $-\cos\,(x)$                                                        & $\mathbb{R}$ \\ 
                    $\cos\,(x)$                                   & $\sin\,(x)$                                                         & $\mathbb{R}$ \\ \bottomrule
                \end{tabular}
            \end{table}
        \end{Spartie}
        \begin{Spartie}{Primitives de Fonctions Composées}
            Les primitives se déduisent des formules de dérivation. $u$ désigne une fonction continue sur $I$ 
            \begin{table}[H] \centering \captionabove{Tableau des Primitives de Fonctions Composées}
                \begin{tabular}[c]{ @{}*{3}{C}@{} } \toprule
                    Fonction $f$ du type                                    & Primitive $F$                                                             & Conditions  \\ \midrule
                    $u'u^n$ ($n\in\mathbb{N}$)                              & $\frac{1}{n+1}\times u^{n+1}$                                             & \---\\
                    $\frac{u'}{u^n}=u'u^{-n}$ ($n\geq2, n\in\mathbb{N}$)   & $-\frac{1}{n-1}\times\frac{1}{u^{n-1}}$ ou $\frac{u^{-n+1}}{-n+1}$        & $\forall x\in I,~u(x)\neq0$ \\
                    $\frac{u'}{u}$                                          & $\ln\,(u)$                                                                  & $\forall x\in I,~u(x)>0$ \\
                    $\frac{u'}{\sqrt{u}}$                                   & $2\sqrt{u}$                                                               & $\forall x\in I,~u(x)>0$ \\
                    $u'\me^u$                                                 & $\me^u$                                                                     & \---\\
                    $u'\sin\,(u)$                                             & $-\cos\,(u)$                                                                & \---\\
                    $u'\cos\,(u)$                                             & $\sin\,(u)$                                                                 & \---\\ \bottomrule
                \end{tabular}
            \end{table}
        \end{Spartie}
        \pagebreak
        \begin{Spartie}{Primitives et Opérations sur les Fonctions} 
            \begin{SSpartie}{Théorème} 
                Si $F$ et $G$ sont des primitives respectivement des fonctions $f$ et $g$ et si $k$ est une constante réelle, $F+G$ est une primitive de $f+g$ et $kF$ est une primitive de $kf$.
            \end{SSpartie}
            \begin{SSpartie}{Propriété} 
                Si $f$ est une fonction continue sur un intervalle tel que $ax+b\in I$, et $F$ est une primitive de $f$, alors : $g:x\mapsto g(x)=f(ax+b)$ admet une primitive :
                \[G(x)=\frac{1}{a}F(ax+b)\]
                \begin{SSSpartie}{Exemple} 
                    $f(x)=(3x-2)^4\qquad
                    \begin{aligned}[t]
                        F(x)&=\tfrac{1}{3}\times\tfrac{1}{5}(3x-2)^5 \\
                        &=\tfrac{1}{15}(3x-2)^5
                    \end{aligned}$
                \end{SSSpartie}
            \end{SSpartie}
        \end{Spartie}
    \end{Gpartie}
    \begin{Gpartie}{Équation Différentielle $y'-ay=0$} 
        \begin{Spartie}{Théorème} 
            Posons $(E) : y'=ay$.

            Soit $a\in\mathbb{R}$, les solutions sur $\mathbb{R}$ de l'équation différentielle $(E)$ sont les fonctions définies sur $\mathbb{R}$ par $f(x)=C\me^{ax}$, où $C$ est une constante réelle :
            \[y'-ay=0\qquad\mathcal{S}=\big\{\,x\mapsto C\me^{ax},~C\in\mathbb{R}\,\big\}\]

            \begin{SSpartie}{Démonstration} 
                \begin{itemize}
                    \item Toute fonction $f:x\mapsto f(x)=C\me^{ax}$ vérifie $f'(x)=aC\me^{ax}=a\times f(x)$. \\ $f$ est donc solution~de $(E)$.
                    \item Montrons que toute solution de $(E)$ est sous la forme $x\mapsto C\me^{ax}$ \\
                    Soit $g$, une solution de $(E)$. On a $g'=ag$, or $f:x\mapsto f(x)=\me^{ax}$ est solution~de $(E)$. \\
                    $f$ ne s'annulant pas, on peut définir $h:x\mapsto \frac{g(x)}{f(x)}=\frac{g(x)}{\me^{ax}}$. \\
                    On peut écrire $h(x)=g(x)\me^{-ax}$ :
                    \[\begin{aligned}[t]
                        h'(x)&=g'(x)\me^{-ax}-a\me^{-ax}g(x) \\
                        &=\me^{-ax}\big(g'(x)-ag(x)\big)\qquad\text{or}\qquad g'(x)=ag(x) \\
                        &=\me^{-ax}\times 0 \\
                        &=0
                    \end{aligned}\]

                    Donc, $h$ est une fonction constante, $\exists C,~h(x)=C=\frac{g(x)}{\me^{ax}}$.
                    \[g(x)=C\me^{ax},~C\in\mathbb{R}\quad\square\]
                \end{itemize}
            \end{SSpartie}
            \begin{Spartie}{Exemple} 
                Soit l'équation $y'+5y=0\iff y'=-5y$.

                Les solutions de l'équation sont les fonctions $f$ définies par $f(x)=C\me^{-5x},\ C\in\mathbb{R}$.
            \end{Spartie}
        \end{Spartie}
    \end{Gpartie}
    \begin{Gpartie}{Équations Différentielles $y'-ay=k(x)$, $k$ étant Continue} 
        \begin{Spartie}{Méthode} 
            Supposons qu'on a deux fonctions $f$ et $g$, solutions de l'équation $(E): y'-ay=k(x)$.

            Alors la fonction $h(x)$, définie par $h(x)=g(x)-f(x)$ est solution de l'équation $y'-ay=0$.
            \begin{SSpartie}{Vérification} 
                \[\begin{aligned}[t]
                    h'(x)-ah(x)&=g'(x)-f'(x)-a\big(g(x)-f(x)\big) \\
                    &=g'(x)-ag(x)-\big(f'(x)-a f(x)\big) \\
                    &=k(x)-k(x) \\
                    &=0
                \end{aligned}\]
                Ainsi, pour tout $x$ (sur un ensemble de définition que l'on n’a pas étudié) :
                \[h(x)=C\me^{ax},~C\in\mathbb{R}\quad\text{(d'après \textsection IV)}\]
                Donc, si l'on trouve une solution particulière $f$ de l'équation $(E)$ toute solution s'écrit sous la forme :
                \[g(x)=f(x)+C\me^{ax},~C\in\mathbb{R}\]
                \[\text{car}\quad g(x)-f(x)=h(x)=C\me^{ax}\] 
            \end{SSpartie}
        \end{Spartie}
        \begin{Spartie}{Exemple} 
            Soit l'équation différentielle $y'+y=\frac{x-1}{x^2}$.$\quad(E)$
            \begin{enumerate}
                \item Vérifier que la fonction inverse est solution de $(E)$.
                \item En déduire toutes les solutions de $(E)$.
            \end{enumerate}

            \begin{enumerate}
                \item Soit $f_0:x\mapsto\frac{1}{x}\qquad\text{alors}\qquad f_0'=\frac{-1}{x^2}$. \\
                $f_0'+f_0=\frac{-1}{x^2}+\frac{1}{x}=\frac{x-1}{x^2}$, la fonction inverse est solution de $(E)$ sur $\mathbb{R^{+*}}$.
                \item D'après la démonstration précédente, toute solution $f$ de $(E)$ est de la forme $f:x\mapsto f_0(x)+C\me^{ax},\ C\in\mathbb{R}$. \\
                On résout $y'+u=0$ : ici $a=-1$. \\
                Ainsi, les solutions sont :
                \[\mathcal{S}=\Big\{\,x\mapsto\tfrac{1}{x}+C\me^{ax},\ C\in\mathbb{R}\,\Big\}\]
            \end{enumerate}
        \end{Spartie}
        \begin{Spartie}{Exemple} 
            Soit l'équation $y'-2y=5$.$\quad(E)$
            \begin{enumerate}
                \item Trouvons une fonction constante solution $(E)$. \\
                La fonction $x\mapsto\frac{-5}{2}$ convient.
                \item On résout $y'-2y=0$ :
                \[\Big\{\,x\mapsto C\me^{2x},\ C\in\mathbb{R}\,\Big\}\]
            \end{enumerate}

            Les solutions de $(E)$ sont les fonctions du type :
            \[\mathcal{S}=\Big\{\,x\mapsto\tfrac{-5}{2}+C\me^{2x},\ C\in\mathbb{R}\,\Big\}\]
        \end{Spartie}
    \end{Gpartie}
\end{document}