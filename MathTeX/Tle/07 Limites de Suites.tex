\documentclass{cours}

\setlength{\arraycolsep}{1cm}

\title{Limites de Suites}

\begin{document}
    \maketitle{7}

    \begin{Gpartie}{Suites Majorées, Minorées et Bornées} 
        \begin{Spartie}{Définitions} 
            \begin{itemize}
                \item Une suite ($u_n$) est \emph{majorée} s'il existe un réel $M$ tel que : $\forall n\in\mathbb{N},~u_n\leq M$
                \item Une suite ($u_n$) est \emph{minorée} s'il existe un réel $m$ tel que : $\forall n\in\mathbb{N},~u_n\geq m$
                \item Une suite est \emph{bornée} si elle est majorée \emph{et} bornée
            \end{itemize}
        \end{Spartie}
        \begin{Spartie}{Exemple} 
            La suite $\left(\frac{1}{n}\right)_{n\geq 1}$ est minorée par 0, mais aussi par tout nombre négatif. 
            
            Elle est majorée par 1 (qui est aussi son maximum) et par tout nombre supérieur~à~1. Elle est donc bornée.
        \end{Spartie}
    \end{Gpartie}
    \begin{Gpartie}{Définitions} 
        \begin{Spartie}{Limite Infinie}
            \vspace*{-2ex}
            \begin{SSpartie}{Définition} 
                Une suite $(u_n)$ a pour limite $+\infty$ si, quel que soit le réel $M$, l'intervalle $\big]\,M~;~+\infty\,\big[$ contient tous les termes de la suite à partir d'un certain rang.

                Autrement dit, pour tout réel $M$, on peut trouver un rang $n$ tel que : \[\forall n\geq N,\quad u_n>M\]
                A partir du rang $N$, tous les termes sont supérieurs à $M$.

                On note : \[\boxed{\lim\limits_{n\to +\infty} u_n=+\infty}\] On peut dire que la suite diverge vers $+\infty$.

                H.P. : $\forall M\in\mathbb{R},~\exists N\in\mathbb{N},~\forall n\in\mathbb{N},~(n\geq N\implies u_n>M)$
            \end{SSpartie}
            \begin{SSpartie}{Définition} 
                Une suite ($u_n$) a pour limite $-\infty$, si, quel que soit le réel $m$, l'intervalle $\big]\,-\infty~;~m\,\big[$ contient tous les termes de le suite à partir d'un certain rang.

                Autrement dit, pour tout réel $m$, on peut trouver un rang $N$ tel que : \[\forall n\geq N,\quad u_n<m\]
                A partir du rang $N$, tous les termes sont inférieurs à $m$.

                On note : \[\boxed{\lim\limits_{n\to +\infty} u_n=-\infty}\] On peut dire que la suite diverge vers $-\infty$.

                H.P. : $\forall m\in\mathbb{R},~\exists N\in\mathbb{N},~\forall n\in\mathbb{N},~(n\geq N\implies u_n<m)$
            \end{SSpartie}
            \begin{SSpartie}{Théorème}
                \begin{center}$\begin{array}{cc}
                    \lim\limits_{n\to +\infty} n^2=+\infty & \lim\limits_{n\to +\infty} \sqrt{n}=+\infty \\
    
                    \lim\limits_{n\to +\infty} \ln\,(n)=+\infty & p\in\mathbb{N^*},\ \lim\limits_{n\to +\infty} n^p=+\infty \\
                    \lim\limits+{n\to +\infty} -n^2=-\infty & \lim\limits_{n\to +\infty} \ln\left(\frac{1}{n}\right)=-\infty
                \end{array}$\end{center}
            \end{SSpartie}
            \begin{SSpartie}{Rappels} 
                $\forall n\in\mathbb{N},\ u_{n+1}-u_n=\dotsc$\quad signe

                $\forall n\in\mathbb{N},\ u_n>0\quad\frac{u_{n+1}}{u_n}$\quad on compare à 1

                Cas où $u_n=f(n)$, par exemple : $u_n=\sqrt{n^2+n-3}$\quad on étudie $f$.
                \begin{SSSpartie}{Arithmétique} 
                    $u_{n+1}=u_n+r$\quad alors\quad $u_n=u_0+nr$\quad\big($u_n=u_p+(n-p)r$\big)

                    $S=u_0+u_1+\dotsb+u_n=\frac{u_0+u_n}{2}\times(n+1)$
                \end{SSSpartie}
                \begin{SSSpartie}{Géométrique} 
                    $u_{n+1}=qu_n$\quad alors\quad $u_n=u_0\times q^n$\quad\big($u_n=u_p\times q^{n-p}$\big)

                    $S=u_0+u_1+\dotsb+u_n=u_0\times\frac{1-q^{n+1}}{1-q}$
                \end{SSSpartie}
            \end{SSpartie}
        \end{Spartie}
        \pagebreak
        \begin{Spartie}{Limites Finies / Suites Convergentes} 
            \begin{SSpartie}{Définition} 
                Une suite ($u_n$) \emph{converge} vers un réel $\ell$ si tout intervalle ouvert contenant $\ell$ contient tous les termes de la suite à partir d'un certain rang.

                Autrement dit, on peut trouver un rang $N$ à partir duquel tous les termes de la suite sont aussi près que l'on veut de $l$.

                On dit que $l$ est la \emph{limite} de la suite ($u_n$) et que la suite est \emph{convergente}.

                On note : \[\boxed{\lim\limits_{n\to +\infty} u_n=\ell}\]

                H.P. : $\forall\epsilon >0,~\exists N\in\mathbb{N},~\forall n\in\mathbb{N}, \Big(n\geq N\implies u_n\in\big]\,\ell-\epsilon~;~\ell+\epsilon\,\big[\Big)$
            \end{SSpartie}
            \begin{SSpartie}{Théorème} 
                \begin{center}$\begin{array}{cc}
                    \lim\limits_{n\to +\infty}\frac{1}{n}=0 & p\in\mathbb{N^*},\ \lim\limits_{n\to +\infty}\frac{1}{n^p}=0 \\
                    \lim\limits_{n\to +\infty}\me^{-n}=0 & \lim\limits_{n\to +\infty}\frac{1}{\sqrt{n}}=0
                \end{array}$\end{center}
            \end{SSpartie}
            \begin{SSpartie}{Définition} 
                Une suite qui n'est pas convergente est \emph{divergente}.
            \end{SSpartie}
            \begin{SSpartie}{Exemple} 
                La suite ($u_n$) définie par $u_n=(-1)^n$ est divergente.
            \end{SSpartie}
        \end{Spartie}
    \end{Gpartie}
    \pagebreak
    \begin{Gpartie}{Propriétés sur les Limites} 
        \begin{Spartie}{Théorèmes de Comparaison} 
            \begin{SSpartie}{Théorème} 
                Soient $(u_n)$ et $(v_n)$ deux suites telles qu'à partir d'un certain rang $u_n\leq v_n$ :
                \begin{itemize}
                    \setlength\itemsep{0.5em}
                    \item Si $\lim\limits_{n\to +\infty}u_n=+\infty$\quad alors\quad $\lim\limits_{n\to +\infty}v_n=+\infty$
                    \item Si $\lim\limits_{n\to +\infty}v_n=-\infty$\quad alors\quad $\lim\limits_{n\to +\infty}u_n=-\infty$
                \end{itemize}
            \end{SSpartie}
            \begin{SSpartie}{Démonstration} 
                On suppose que $\lim\limits_{n\to +\infty}u_n=+\infty$.

                Soit $M>0$, il existe un rang $N$, à partir duquel si $n\geq N_1$, alors $u_n\geq M$.

                Or, il existe un rang $N_2$, à partir duquel $n>N_2,\ v_n\geq u_n$.

                Donc, il existe un rang $N=\max\,(\,N_1~;~N_2\,)$ à partir duquel $v_n\geq u_n>M$ donc: \[\lim\limits_{n\to +\infty}v_n=+\infty\quad\square\]
            \end{SSpartie}
            \begin{SSpartie}{Théorème dit \og des Gendarmes \fg{} (Théorème d'Encadrement)}
                Soient $(u_n),\ (v_n),$ et $(w_n)$ trois suites telles qu'à partir d'un certain rang~$u_n\leq v_n\leq w_n$ :

                Si $\lim\limits_{n\to +\infty}u_n=\ell$ et $\lim\limits_{n\to +\infty}w_n=\ell$ où $\ell$ est un réel alors :
                \[\lim\limits_{n\to +\infty}v_n=\ell\]
            \end{SSpartie}
            \begin{SSpartie}{Exemple} 
                Soit $(u_n)$ définie par $u_n=\frac{(-1)^n}{n}$

                $\begin{aligned}[t]
                    \forall n\in\mathbb{N},&\quad-1\leq (-1)^n\leq 1 \\
                    \iff&\quad\frac{-1}{n}\leq\frac{(-1)^n}{n}\leq\frac{1}{n}
                \end{aligned}$ \\[2ex]
                Or, $\lim\limits_{n\to +\infty}\frac{-1}{n}=0,\ \lim\limits_{n\to +\infty}\frac{1}{n}=0$

                D'après le Théorème des Gendarmes, $\lim\limits_{n\to +\infty}u_n=0$
            \end{SSpartie}
        \end{Spartie}
        \pagebreak
        \begin{Spartie}{Convergence Monotone} 
            \begin{SSpartie}{Théorème Admis} 
                Toute suite croissante et majorée converge vers une limite finie.

                Toute suite décroissante et minorée converge vers une limite finie.
            \end{SSpartie}
            \begin{SSpartie}{Théorème} 
                Soit une suite $(u_n)$ croissante et qui converge vers un réel $\ell$, alors $(u_n)$ est majorée par $\ell$.
                \begin{SSSpartie}{Démonstration par l'Absurde} 
                    On suppose que $(u_n)$ est croissante.
                    \begin{SSSSpartie}{Lemme} 
                        Si $(u_n)$ est croissante, si $p$ et $n$ sont deux entiers naturels tels que $p\leq n$.

                        Alors, $u_p\leq u_n$.
                    \end{SSSSpartie}
                    \begin{SSSSpartie}{Démonstration de la Lemme par Récurrence}
                        \vspace{-1em}
                        \begin{itemize}
                            \item Initialisation : On fixe $p$ donc $u_p\leq u_{p+1}$.
                            \item Hérédité : On suppose que $k\geq 1,\ u_p\leq u_{p+k}$. \\
                            \phantom{Hérédité : }$u_p\leq u_{p+k}\implies u_p\leq u_{p+k}\leq u_{p+k+1}\quad\square$
                        \end{itemize}   
                    \end{SSSSpartie}
                    On suppose $\exists n_0\in\mathbb{N},\ u_{n_0}>\ell$.

                    Or tout intervalle ouvert contenant $\ell$ contient tous les termes de la suite à partir d'un certain rang.

                    Prenons l'intervalle ouvert $\big]\,a~;~b\,\big[$ tel que $\ell<b<u_{n_0}$.

                    Il existe un indice $p>n_0$ tel que $u_p\in\big]\,a~;~b\,\big[$ \\ (Ils y sont tous à partir d'un certain rang !)

                    Donc, $u_p<b<u_{n_0}$, ce qui impossible car $(u_n)$ est croissante et d'après la~lemme,~$p>n_0\implies u_p\geq u_{n_0}$.

                    C'est absurde, donc, $\forall n\in\mathbb{N},~u_n\leq\ell$.$\quad\square$
                \end{SSSpartie}
            \end{SSpartie}
            \pagebreak
            \begin{SSpartie}{Théorème} 
                Toute suite croissante et non-majorée diverge vers $+\infty$

                Toute suite décroissante et non-minorée diverge vers $-\infty$

                \begin{SSSpartie}{Démonstration} 
                    Soit $M$ un réel, et $(u_n)_{n\in\mathbb{N}}$, une suite croissante, non-majorée.

                    Il existe un rang $N$ tel que $u_N>M$

                    Donc, $(u_n)$ diverge vers $+\infty$
                    \begin{itemize}
                        \item Si $(u_n)$ admettait une limite finie, d'après le théorème de croissance, 
                        \[\lim\limits_{n\to +\infty}u_n=\ell\implies\forall n\in\mathbb{N},\ u_n\leq\ell\] \\ $(u_n)$ est majorée par $\ell$, c'est une contradiction.$\quad\square$
                        \\[2ex]
                        \item Soit $M\in\mathbb{R}$ et $(u_n)_{n\in\mathbb{N}}$, une suite non-majorée, donc, $\exists n_0,\ u_{n_0}>M$ \\ D'après la lemme : $(u_n)$ croissante, $p,\ n\in\mathbb{N},\ p\leq n\implies u_p\leq u_n$ \\ $\forall n\geq n_0,\ u_n\geq u_{n_0}>M$ \\ Donc, tous les termes à partir de $n_0$ sont supérieurs à $M$. D'où : \[\lim\limits_{n\to +\infty}u_n=+\infty\]
                    \end{itemize}
                \end{SSSpartie}
            \end{SSpartie}
        \end{Spartie}
        \begin{Spartie}{Rappel : Limite de $\left(q^n\right)$ où $q\in\mathbb{R}$} 
            \begin{SSpartie}{Théorème} 
                \begin{itemize}
                    \setlength\itemsep{0.5em}
                    \item Si $q>1,~\lim\limits_{n\to+\infty}q^n=+\infty$
                    \item Si $q=1,~\lim\limits_{n\to+\infty}q^n=1$
                    \item Si $-1<q<1,~\lim\limits_{n\to+\infty}q^n=0$
                    \item Si $q<-1$, la suite diverge.
                \end{itemize}
            \end{SSpartie}
        \end{Spartie}
    \end{Gpartie}
    \pagebreak
    \begin{Gpartie}{Opérations sur les Limites} 
        On considère les suites $(u_n)_{n\in\mathbb{N}}$ et $(v_n)_{n\in\mathbb{N}}$ admettant des limites finies ou infinies. \\ F.I. : Forme Indéterminée, il faut faire un calcul pour lever l'indétermination
        \begin{Spartie}{Somme}
            \begin{table}[H] \centering \captionabove{Tableau des Limites de Sommes de Suites} 
                    %  tableau alternatif, avec barres verticales
                    % \begin{tabular}{ |p{0.15\textwidth}||w{c}{0.1\textwidth}|w{c}{0.1\textwidth}|w{c}{0.1\textwidth}| } \hline
                    %     \diagbox[innerwidth=0.15\textwidth]{$\lim v_n$}{$\lim u_n$} & $\ell$ un réel& $+\infty$ & $-\infty$ \\ \hline \hline
                    %     $\ell'$ un réel                                             & $\ell+\ell'$  & $+\infty$ & $-\infty$ \\ \hline
                    %     $+\infty$                                                   & $+\infty$     & $+\infty$ & F.I.\footnotemark[1] \\ \hline
                    %     $-\infty$                                                   & $-\infty$     & F.I.\footnotemark[1] & $-\infty$ \\ \hline
                    % \end{tabular}
                \resizebox{0.75\linewidth}{!}{
                    \begin{tabular}{ p{0.15\linewidth} *{3}{M{0.15\linewidth} }} \toprule
                        {}  & \multicolumn{3}{c}{$\lim u_n$}                \\ \cmidrule(r l){2-4}
                        $\lim v_n$  & $\ell'$       & $+\infty$ & $-\infty$ \\ \midrule
                        $\ell'$     & $\ell+\ell'$  & $+\infty$ & $-\infty$ \\ 
                        $+\infty$   & $+\infty$     & $+\infty$ & F.I.      \\
                        $-\infty$   & $-\infty$     & F.I.      & $-\infty$ \\ \bottomrule
                    \end{tabular}
                }
            \end{table}
            \begin{SSpartie}{Exemple} 
                $(u_n)_{n\in\mathbb{N}}$ définie par $\begin{aligned}[t]u_n&=n^2-n\quad\text{F.I.} \\ &=n(n-1)\quad\text{On a levé l'indétermination}\end{aligned}$

                $\underbrace{\lim\limits_{n\to +\infty}n^2=+\infty\quad\lim\limits_{n\to +\infty}-n=-\infty}_{F.I.}$
                \qquad$\lim\limits_{n\to +\infty}n=+\infty\quad\lim\limits_{n\to +\infty}n-1=+\infty$

                Donc, par produit, $\lim\limits_{n\to +\infty}n(n+1)=+\infty$ \\ et donc, $\lim\limits_{n\to +\infty}u_n=+\infty$
            \end{SSpartie}
        \end{Spartie}
        \begin{Spartie}{Produit}
            \begin{table}[H] \centering \captionabove{Tableau des Limites de Produits de Suites}
                % tableau alternatif, avec barres verticales
                % \begin{tabular}{ |p{0.15\textwidth}||w{c}{0.15\textwidth}|w{c}{0.15\textwidth}|w{c}{0.15\textwidth}|w{c}{0.15\textwidth}| } \hline
                %     \diagbox[innerwidth=0.15\textwidth]{$\lim v_n$}{$\lim u_n$} & $\ell\neq 0$ & 0 & $+\infty$ & $-\infty$ \\ \hline\hline
                %     $\ell'\neq 0$                                               & $\ell\times l'$ & 0 & $\pm\infty\text{\tiny{ selon signe }}\ell'$ & $\pm\infty\text{\tiny{ selon signe }}\ell'$ \\ \hline
                % 0                                                               & 0 & 0 & F.I. & F.I. \\ \hline
                % $+\infty$                                                       & $\pm\infty\text{\tiny{ selon signe }}\ell$ & F.I. & $+\infty$ & $-\infty$ \\ \hline
                %     $-\infty$                                                   & $\pm\infty\text{\tiny{ selon signe }}\ell$ & F.I. & $-\infty$ & $-\infty$ \\ \hline
                % \end{tabular}
                \resizebox{0.85\linewidth}{!}{
                    \begin{tabular}{ p{0.15\linewidth} *{4}{M{0.17\linewidth} }} \toprule
                        {} & \multicolumn{4}{c}{$\lim u_n$} \\ \cmidrule(r l){2-5}
                        $\lim v_n$  & $\ell\neq0$                                   & $0$ & $+\infty$                                   & $-\infty$                                     \\ \midrule
                        $\ell'\neq0$& $\ell\times\ell'$                             & $0$ & $\pm\infty~\text{\tiny{selon signe}}~\ell'$ & $\pm\infty~\text{\tiny{selon signe}}~\ell'$   \\
                        $0$         & $0$                                           & $0$ & F.I.                                        & F.I.                                          \\
                        $+\infty$   & $\pm\infty~\text{\tiny{selon signe}}~\ell$    & F.I.& $+\infty$                                   & $-\infty$                                     \\
                        $-\infty$   & $\pm\infty~\text{\tiny{selon signe}}~\ell$    & F.I.& $-\infty$                                   & $+\infty$                                     \\ \bottomrule
                    \end{tabular}
                }
            \end{table}
            \begin{SSpartie}{Exemple} 
                Soit $(u_n)_{n\in\mathbb{N}}$, définie par $u_n=n\quad\lim\limits_{n\to +\infty}u_n=+\infty$

                Soit $(v_n)_{n\in\mathbb{N}}$, définie par $v_n=\frac{1}{\sqrt{n}}\quad\lim\limits_{n\to +\infty}v_n=0$

                $u_n\times v_n=n\times\frac{1}{\sqrt{n}}=\sqrt{n}\quad\text{or}\quad\lim\limits_{n\to +\infty}\sqrt{n}=+\infty$

                Donc, $\lim\limits_{n\to +\infty}(u_n\times v_n)=+\infty$
            \end{SSpartie}
            \begin{SSpartie}{Exemple 2} 
                $(u_n)_{n\in\mathbb{N}}$, définie par $u_n=n^2\quad\lim\limits_{n\to +\infty}u_n=+\infty$

                $(v_n)_{n\in\mathbb{N}}$, définie par $v_n=\frac{1}{n}-4\quad\lim\limits_{n\to +\infty}v_n=-4$ (Somme)

                Donc, $\lim\limits_{n\to +\infty}(u_n\times v_n)=-\infty$
            \end{SSpartie}
        \end{Spartie}
        \begin{Spartie}{Quotient}
            \begin{table}[H]% en commentaire : tableau alternatif, avec barres verticales.
                \centering \captionabove{Tableau des Limites de Quotients de Suites}
                % \begin{tabular}{ |p{0.15\textwidth}||w{c}{0.15\textwidth}|w{c}{0.15\textwidth}|w{c}{0.15\textwidth}|w{c}{0.15\textwidth}| } \hline
                %     \diagbox[innerwidth=0.15\textwidth]{$\lim v_n$}{$\lim u_n$} & $\ell\neq 0$          & 0 & $+\infty$ & $-\infty$ \\ \hline\hline
                %     $\ell'\neq 0$                                               & $\frac{\ell}{\ell'}$  & 0 & $\pm\infty\text{\tiny{ selon signe }}\ell'$ & $\pm\infty\text{\tiny{ selon signe }}\ell'$ \\ \hline
                %     0                                                           & $\pm\infty\text{\tiny{ selon signe }}\ell\text{\tiny{ et }}0$ & F.I. & $\pm\infty\text{\tiny{ selon signe }}0$ & $\pm\infty\text{\tiny{ selon signe }}0$ \\ \hline
                %     $+\infty$                                                   & 0 & 0 & F.I. & F.I. \\ \hline
                %     $-\infty$                                                   & 0 & 0 & F.I. & F.I. \\ \hline
                % \end{tabular}
                \resizebox{0.85\linewidth}{!}{
                    \begin{tabular}{ p{0.15\linewidth} *{4}{M{0.17\linewidth} }} \toprule
                        {}              & \multicolumn{4}{c}{$\lim u_n$} \\ \cmidrule(l r){2-5}
                        $\lim v_n$      & $\ell\neq0$                                                   & $0$ & $+\infty$                                   & $-\infty$                                     \\ \midrule
                        $\ell'\neq 0$   & $\frac{\ell}{\ell'}$                                          & $0$ & $\pm\infty\text{\tiny{ selon signe }}\ell'$ & $\pm\infty\text{\tiny{ selon signe }}\ell'$   \\ 
                        $0$             & $\pm\infty\text{\tiny{ selon signe }}\ell\text{\tiny{ et }}0$ & F.I.& $\pm\infty\text{\tiny{ selon signe }}0$     & $\pm\infty\text{\tiny{ selon signe }}0$       \\
                        $+\infty$       & $0$                                                           & $0$ & F.I.                                        & F.I.                                          \\ 
                        $-\infty$       & $0$                                                           & $0$ & F.I.                                        & F.I.                                          \\ \bottomrule
                    \end{tabular}
                }
            \end{table}
            \begin{SSpartie}{Exemple} 
                $(u_n)_{n\in\mathbb{N}}$, définie par $u_n=\frac{1}{n^2-3}\quad\lim\limits_{n\to +\infty}u_n=+\infty$ (Somme)

                Donc, par quotient, $\lim\limits_{n\to +\infty}u_n=0$
            \end{SSpartie}
        \end{Spartie}
    \end{Gpartie}
    \pagebreak
    \begin{Gpartie}{Limite de Suite et Continuité} 
        \begin{Spartie}{Théorème} 
            Soit $f$ une fonction \emph{continue} sur un intervalle $I$ et $(u_n)_{n\in\ I}$, une suite qui converge vers un réel $\ell$, telle que $\forall n\in\mathbb{N},\ u_n\in I,~\ell\in I$, alors, la suite $\big(f(u_n)\big)$ converge~vers~$f(\ell)$.
            \begin{SSpartie}{Exemple} 
                Soit le suite $(u_n)$, définie par $u_n=\frac{4n}{n+1}$. Alors $\lim\limits_{n\to +\infty}u_n=4$
                
                Donc, la suite $(v_n)$, définie par $v_n=\sqrt{u_n}$ converge vers $\sqrt{4}=2$.
            \end{SSpartie}
        \end{Spartie}
        \begin{Spartie}{Théorème}
            Soit $f$ une fonction \emph{continue} sur un intervalle $I$, telle que $f(I)\subset I$ et $(u_n)_{n\in\mathbb{N}}$ une suite définie par $u_{n+1}=f(u_n)$ et $u_0\in I$
            
            Si la suite $(u_n)$ \emph{converge vers un réel} $\ell$, alors $l$ est solution de l'équation $f(x)=x$ \\ On peut dire que $\ell$ est un point fixe de $f$.
            \begin{SSpartie}{Démonstration} 
                $\lim\limits_{n\to +\infty}u_{n+1}=\lim\limits_{n\to +\infty}u_n=f(\ell)$
            \end{SSpartie}
            \begin{SSpartie}{Remarque} 
                Graphiquement, les termes de la suite se rapprochent du point d'intersection $\mathcal{C}_f$ et la droite d'équation $y=x$. Uniquement si la suite converge !
                \begin{center}
                    \begin{tikzpicture}[scale=0.9]
                        \begin{axis}[
                            xmin=-4, xmax=8,
                            ymin=-3, ymax=10,
                            domain=-4.2:8.2,
                            xtick=\empty,
                            ytick=\empty
                        ]
                            \addplot[red, thick]{x};
                            \addplot[blue, very thick, smooth]{(x^2)/4};

                            \draw 
                                (4,4) node [dot, label={[label distance=-2pt]below right:$\scriptscriptstyle  f(x)=x$}] {}
                                (0,0) node [dot, label={[label distance=-2pt]below right:$\scriptscriptstyle  f(x)=x$}] {}
                                (-1,-1) node [label={[label distance=-2pt]left:$\scriptscriptstyle y=x$}] {}
                                (-1.5,0.57) node [label={left:$\scriptscriptstyle y=f(x)$}] {};

                            \draw[-{stealth[scale=0.5]}, black!60!green] 
                                (3.1,3.1) edge (3.1,2.3)
                                (3.1,2.3) edge (2.3,2.3) 
                                (2.3,2.3) edge (2.3,1.4)
                                (2.3,1.4) edge (1.4,1.4)
                                (1.4,1.4) edge (1.4,0.5)
                                (1.4,0.5) -- (0.5,0.5);

                            \draw[dashed, black!60!green]
                                (0,3.1) node [label={[label distance=-6pt]180:$\scriptscriptstyle u_0$}] {} -- (3.1,3.1)
                                (0,2.3) node [label={[label distance=-6pt]180:$\scriptscriptstyle u_1$}] {} -- (2.3,2.3)
                                (0,1.4) node [label={[label distance=-6pt]180:$\scriptscriptstyle u_2$}] {} -- (1.4,1.4)
                                (0,0.5) node [label={[label distance=-6pt]180:$\scriptscriptstyle u_3$}] {} -- (0.5,0.5);

                            \draw[-{stealth[scale=0.5]}, orange]
                                (4.5,4.5) edge (4.5,5.1)
                                (4.5,5.1) edge (5.1,5.1)
                                (5.1,5.1) edge (5.1,6.4)
                                (5.1,6.4) edge (6.4,6.4)
                                (6.4,6.4) -- (6.4,10.2);

                            \draw[dashed, orange]
                                (0,4.5) node [label={[label distance=-6pt]180:$\scriptscriptstyle u_0$}] {} -- (4.5,4.5)
                                (0,5.1) node [label={[label distance=-6pt]180:$\scriptscriptstyle u_1$}] {} -- (5.1,5.1)
                                (0,6.4) node [label={[label distance=-6pt]180:$\scriptscriptstyle u_2$}] {} -- (6.4,6.4);
                        \end{axis}
                    \end{tikzpicture}
                    \parbox{\linewidth}{\captionof{figure}{\centering Représentation Graphique de la Convergence d'une Suite}}
                \end{center}
                
                Ici, lorsque $u_0$ est supérieur au deuxième point fixe de $f$, le suite diverge.
            \end{SSpartie}
        \end{Spartie}
    \end{Gpartie}
\end{document}