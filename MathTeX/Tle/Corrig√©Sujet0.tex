\documentclass{scrartcl}
\usepackage[T1]{fontenc}
\usepackage{amsmath,amssymb}
\usepackage[french]{babel}
\usepackage{changepage}
\usepackage{newpxtext}
\usepackage{newpxmath}
\usepackage{enumitem}
\usepackage{mathtools}
\usepackage{tikz}
\usepackage{tkz-tab}
\usepackage{array}
\usetikzlibrary{trees}

\setenumerate{itemsep=1em}

% \DeclareUnicodeCharacter{1D49}{ᵉ}

\renewcommand{\arraystretch}{1.5}

\title{Correction Bac 2021 Sujet 0}
\author{N. Sibert}
\begin{document}
    \maketitle

    \section*{Exercice 1}
    \begin{adjustwidth}{2em}{0pt}
        \begin{enumerate}
            \item \fbox{Réponse b)}\par 
            On a $u_n\leq w_n\leq v_n$,\quad$\lim\limits_{n\to +\infty}u_n=1$\quad et \quad$\lim\limits_{n\to +\infty}v_n=1$\quad car $0<\frac{1}{4}<1$ \\ Donc, d'après le théorème des gendarmes, $w_n$ converge vers $1$.            
            \item \fbox{Réponse c)}\par
            $f'(x)=e^{x^2}+x\times 2xe^{x^2}=\left(1+2x^2\right)e^{x^2}$
            \item \fbox{Réponse c)}\par
            $\dfrac{x^2-1}{2x^2-2x+1}=\dfrac{1-\frac{1}{x^2}}{2-\frac{2}{x}+\frac{1}{x^2}}$\quad la limite vaut donc $\frac{1}{2}$
            \item \fbox{Réponse c)}\par
            C'est le théorème des valeurs intermédiaires (cas général, on n'a pas d'information sur la monotonie de $h$)
            \item \fbox{Réponse c)}\par
            $g'$ est croissante sur $\big[1\,;2\big]$ donc $g$ convexe sur $\big[1\,;2\big]$
        \end{enumerate}
    \end{adjustwidth}
    
\pagebreak    
    
    \section*{Exercice 2}
    \begin{adjustwidth}{2em}{0pt}
        \begin{enumerate}
            \item \begin{enumerate}[label=\alph*)]
                \item   $\boxed{I\left(\tfrac{1}{2}\,;0\,;1\right)}\quad$et$\quad \boxed{J\left(2\,;0\,;1\right)}$
                \item   $D\left(0\,;1\,;1\right)\quad$donc$\quad\boxed{\overrightarrow{DJ}\begin{psmallmatrix}2 \\ -1 \\ 1\end{psmallmatrix}}\qquad B\left(1\,;0\,;0\right)\quad\boxed{\overrightarrow{BI}\begin{psmallmatrix}-\frac{1}{2} \\ 0 \\ 1\end{psmallmatrix}}\quad$et$\quad\boxed{\overrightarrow{BG}\begin{psmallmatrix}0 \\ 1 \\ 1\end{psmallmatrix}}$
                \item   $\overrightarrow{BI}$ et $\overrightarrow{BG}$ sont deux vecteurs \textbf{non colinéaires} de $(BGI)$ \par
                 $\overrightarrow{DJ}\cdot\overrightarrow{BI}=2\times\left(-\frac{1}{2}\right)+\left(-1\right)\times 0+1\times 1=0\quad$donc$\quad\overrightarrow{DJ}\perp\overrightarrow{BI}$ \par
                 $\overrightarrow{DJ}\cdot\overrightarrow{BG}=2\times 0+(-1)\times 1+1\times 1=0\quad$donc$\quad \overrightarrow{DJ}\perp\overrightarrow{BG}$ \par
                 \hspace{2em} donc, $\boxed{\overrightarrow{DJ}\perp(BGI)},\ \overrightarrow{DJ}$ vecteur normal à $(BGI)$
                \item   $\overrightarrow{DJ}$ étant un vecteur normal de $(BGI)$, tout point $M(x\,;y\,;z)\in(BGI)$ vérifie \linebreak $\overrightarrow{BM}\cdot\overrightarrow{DJ}=0\quad$où$\quad\overrightarrow{BM}\begin{psmallmatrix}x-1 \\ y \\ z\end{psmallmatrix}\quad$donc$\quad 2(x-1)-y+z=0$ \par
                $\boxed{(BGI):2x-y+z-2=0}$ 
            \end{enumerate}
            \item \begin{enumerate}[label=\alph*)]
                \item   $d$ passe par $F(1\,;0\,;1)$ et admet $\overrightarrow{DJ}$ comme vecteur directeur. Donc : \[\boxed{d:\begin{cases}
                    x=1+2t \\ y=-t \\ z=1+t
                \end{cases} (t\in{\mathbb{R}})}\]
                \item On peut vérifier que $L\in d$ et $L\in(BGI)$, donc trouver $L$ dont les coordonnées vérifient 1d) et 2a), on peut trouver $t$ tel que : \[2(1+2t)-(-t)+1+t-2=0\iff t=-\frac{1}{6}\]  \[\text{et :}\begin{cases}
                    x=1-\frac{2}{6}=\frac{2}{3} \\ y=\frac{1}{6} \\ z=1-\frac{1}{6}=\frac{5}{6}
                \end{cases}\] \\
                On retrouve $\boxed{L\left(\frac{2}{3}\,;\frac{1}{6}\,;\frac{5}{6}\right)}$ c'est bien $d\cap(BGI)$
            \end{enumerate}
            \item \begin{enumerate}[label=\alph*)]
                \item On prend pour base $FBG$ et pour hauteur $FI$. \[V=\frac{1}{3}\times\frac{1\times 1}{2}\times\frac{1}{2}\qquad\boxed{V=\frac{1}{12}}\]
                \item Si on prend pour base $BGI$, la hauteur est alors $FL$ car $L$ est le projeté orthogonal de $F$ sur $(BGI)$ et on a : \[FL=\sqrt{\left(\frac{2}{3}-1\right)^2+\left(\frac{1}{6}-0\right)^2+\left(\frac{5}{6}-1\right)^2}=\sqrt{\frac{1}{9}+\frac{1}{36}+\frac{1}{36}}=\frac{\sqrt{6}}{6}\] \[\text{Donc, }\frac{1}{12}=V=\frac{1}{3}\times\mathcal{A}_{BGI}\times\frac{\sqrt{6}}{6},\quad\text{d'où}\quad\boxed{\mathcal{A}_{BGI}=\frac{\sqrt{6}}{4}}\]
            \end{enumerate}
        \end{enumerate}
    \end{adjustwidth}
    \section*{Exercice 3}
    \begin{adjustwidth}{2em}{0pt}
        \begin{enumerate}
            \item   % Set the overall layout of the tree
                    \tikzstyle{level 1}=[level distance=3.5 cm, sibling distance=3.5 cm]
                    \tikzstyle{level 2}=[level distance=3.5 cm, sibling distance=1.5 cm]
            
                    % Define styles for bags and leafs
                    \tikzstyle{bag} = [text width=4em, text centered]
                    \tikzstyle{end} = [circle, minimum width=3pt,fill, inner sep=0pt]
                    \ %single space for enum label
                    \begin{center}\begin{tikzpicture}[grow=right]
                        \coordinate child {
                            node[bag] {$A$}        
                            child {
                                node[end, label=right:{$R_3$}] {}
                                edge from parent
                                node[below]  {$\frac{2}{9}$}
                            }
                            child {
                                node[end, label=right:{$R_2$}] {}
                                edge from parent
                                node[above] {$\frac{1}{3}$}
                            }
                            child {
                                node[end, label=right:{$R_1$}] {}
                                edge from parent
                                node[above] {$\frac{4}{9}$}
                            }
                            edge from parent 
                            node[below]  {$\frac{3}{4}$}
                        }
                        child {
                            node[bag] {$\overline{A}$}        
                            child {
                                node[end, label=right:{$R_2$}] {}
                                edge from parent
                                node[below] {$\frac{1}{3}$}
                            }
                            child {
                                node[end, label=right:{$R_1$}] {}
                                edge from parent
                                node[above]  {$\frac{2}{3}$}
                            }
                            edge from parent         
                            node[above] {$\frac{1}{4}$}
                        };
                    \end{tikzpicture}\end{center}
                    \parbox{\linewidth-2em}{\captionof{figure}{Arbre de Probabilité Modélisant la Situation}}
    
            \item   \begin{enumerate}[label=\alph*)]
                        \item	$P(A\cap R_2)=\frac{1}{4}\times\frac{1}{3}=\boxed{\frac{1}{12}}$
                        \item   D'après la loi des probabilités totales, $A$ et $\overline{A}$ étant une partition de l'univers : 
                                \[\begin{aligned}[t]
                                    P(R_2)&=P(A\cap R_2)+P(\overline{A}\cap R_2) \\
                                    &=\frac{1}{12}+\frac{3}{4}\times\frac{1}{3}=\frac{4}{12}=\boxed{\frac{1}{3}}
                                \end{aligned}\]
                        \item   $P_{R_2}(A)=\dfrac{P(A\cap R_2)}{P(R_2)}=\boxed{\frac{1}{4}}$
                    \end{enumerate}
                    
            \item   \begin{enumerate}[label=\alph*)]
                        \item  \hfill\begin{tabular}[t]{ | l || *{8}{c| } } \firsthline
                                    $x_i$                     & 1             & 2             & 3               \\ \hline
                                    $P(X=x_i)$ \hspace{0.5cm} & $\frac{1}{2}$ & $\frac{1}{3}$ & $\frac{1}{6}$   \\ \hline
                                \end{tabular}\hfill\mbox{}
                                \parbox{\linewidth}{\captionof{figure}{Tableau présentant la loi de probabilité de X}}  
                        \item   $E(X)=1\times\frac{1}{2}+2\times\frac{1}{3}+3\times\frac{1}{6}=\frac{5}{3}\qquad\boxed{E(X)\approx 1{,}7}$ \par
                                En moyenne, il faut donc $1{,}7$ essais pour réussir son permis.
                    \end{enumerate}
            \item   \begin{enumerate}[label=\alph*)]
                        \item   On note $Y$ la variable aléatoire qui, à toute personne choisie au hasard dans le groupe, associe le nombre de personnes ayant réussi avant le 3\textsuperscript{e} essai.
                        \[1-\left(\frac{5}{6}\right)^n=1-P(Y=0)=\boxed{P(Y\geq 1)}\] 
                        Donc, c'est la probabilité qu'au moins une personne parmi les $n$ a eu son permis à la 3\textsuperscript{e} tentative. Ici, $Y\sim\mathcal{B}\left(n\,;\frac{1}{6}\right)$ (tirage avec remise)

                        \item   \[\begin{aligned}[t]
                                    1-\left(\frac{5}{6}\right)^n>0{,}9&\iff\left(\frac{5}{6}\right)^n<0{,}1 \\
                                    &\iff n>\frac{\ln\left(0{,}1\right)}{\ln\left(\tfrac{5}{6}\right)}\approx 12{,}6
                                \end{aligned}\]
                                \fbox{L'algorithme renvoie 13} \par
                                Donc, à partir du moment où la taille de l'échantillon est de 13 personnes, la probabilité qu'au moins une soit passée 3 fois est supérieur à $90\%$.
                    \end{enumerate}
        \end{enumerate}
    \end{adjustwidth}
    \section*{Exercice A}
    \begin{adjustwidth}{2em}{0pt}
        \subsection*{Partie I}
        \begin{enumerate}
            \item	$f'(\frac{1}{e})=0$ (tangente horizontale en $A$) \\ $f'(1)=-1$ (coéfficient directeur de la tangente $B$)
            \item   $T_B:y=-x+p$ où $p$ est un réel et $T_B$ passe par $(0\,;3)$ \\ D'où : \[\boxed{T_B:y=-x+3} \qquad \text{on peut aussi faire }y=f'(1)(x-1)+f(1)\]
        \end{enumerate}
        \subsection*{Partie II}
        \begin{enumerate}
            \item	$f\left(\frac{1}{e}\right)=\dfrac{2+\ln\left(\frac{1}{e}\right)}{\frac{1}{e}}=2e+e\ln\left(\frac{1}{e}\right)=e(2-1)=e$ \qquad \fbox{$C_f$ passe par $A\ \left(\frac{1}{e}\,;e\right)$} \par
                    $f(1)=\dfrac{2+\ln(1)}{1}=2$ \qquad \fbox{$C_f$ passe par $B\ \left(1\,;2\right)$} \par
                    \vspace{1em}
                    $\begin{aligned}[t]
                        f(x)=0&\iff \frac{2+\ln(x)}{x}=0 \\
                        &\iff 2+\ln(x)=0\quad\text{(et $x\neq 0$)} \\
                        &\iff \ln(x)=-2 \\
                        &\iff x=e^{-2}
                    \end{aligned}$ \par\vspace{1em}
                    \fbox{$C_f$ coupe l'axe des abscisses en $\left(e^{-2}\,;0\right)$} (point unique)
            \item   Etudions la limite lorsque $x$ tend vers $0$ par valeurs supérieures : \vspace{1em}
                    \[\begin{drcases}\lim\limits_{\substack{x\to 0\\x>0}}2+\ln(x)=-\infty\quad \\ \lim\limits_{\substack{x\to 0\\x<0}}x=0^+\end{drcases}\quad\text{par quotient,}\quad\boxed{\lim\limits_{\substack{x\to 0\\x>0}}f(x)=-\infty}\] \\ \vspace{1em}
                    Etudions la limite lorsque $x$ tend vers $+\infty$ :
                    \[f(x)=\frac{2}{x}+\frac{\ln(x)}{x}\quad\text{et}\quad\lim\limits_{x\to +\infty}\frac{2}{x}=0\quad\text{et}\quad\lim\limits_{x\to +\infty}\frac{\ln(x)}{x}=0\qquad\text{(théorème)}\] \vspace{1em}
                    Donc, par somme, $\lim\limits_{x\to +\infty}f(x)=0$.

            \item   $\forall x\in\big]0\,;+\infty\big[,\ f'(x)=\dfrac{\tfrac{1}{x}\times x-\big(2+\ln(x)\big)}{x^2}=\dfrac{1-2-\ln(x)}{x^2}$ \[\boxed{f'(x)=\frac{-1-ln(x)}{x^2}}\]
                    \pagebreak
            \item   \[\begin{aligned}[t]
                        -1-\ln(x)\geq 0&\iff-1\geq\ln(x) \\
                        &\iff e^{-1}\geq x\quad\text{et}\quad x^2>0\quad\text{sur}\quad\big]0\,;+\infty\big[
                    \end{aligned}\]\vspace{1em}
                    \begin{center}\begin{tikzpicture}
                        \tkzTabInit{$x$ / 1, $f'(x)$ / 1, $f$ / 2}{$0$, $\frac{1}{2}$, $+\infty$}
                        \tkzTabLine{ ,$+$,z,$-$, }
                        \tkzTabVar{-/$-\infty$, +/$e$, -/$0$}
                    \end{tikzpicture}\end{center}
                    \parbox{\linewidth}{\captionof{figure}{Tableau de Signes $f'(x)$ et de Variations de $f$}}

            \item   $f$ est convexe si et seulement si $f''(x)\geq 0$ \\ $1+2\ln(x)\geq 0\iff x\geq e^{-\frac{1}{2}}$ \par \fbox{$f$ convexe sur $\big[e^{-\frac{1}{2}}\,;+\infty\big[$}
        \end{enumerate}
    \end{adjustwidth}
    \section*{Exercice B}
\end{document}