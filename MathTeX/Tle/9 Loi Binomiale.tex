\documentclass{cours}

\title{Succession d'Épreuves Indépendantes \\ Loi Binomiale}

\begin{document}
    \maketitle{9}

    \begin{Gpartie}{Succession d'Épreuves Indépendantes} 
        \begin{Spartie}{Rappel} 
            Deux épreuves successives sont indépendantes lorsque le résultat de la première n'influe pas sur le résultat de la deuxième.

            \begin{center}
                % \begin{tikzpicture}
                    \includegraphics[width=10cm]{example-image}
                % \end{tikzpicture}
                \parbox{\linewidth}{\captionof{figure}{Arbre de Probabilité qui Présente l'Indépendance des Épreuves}} \\[2ex]
            \end{center}
            Ainsi, $A$ et $B$ sont deux événements indépendants si et seulement si :
            \begin{itemize}
                \item $P_A(B)=P_{\overline{A}}(B)=P(B)$
                \item $P(A\cap B)=P(A)\times P(B)$
            \end{itemize}
        \end{Spartie}
    \end{Gpartie}
\end{document}