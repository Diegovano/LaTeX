\documentclass[12pt, a4paper]{book}
\usepackage{amsmath}
\usepackage{amsfonts}
%\usepackage[upright]{fourier}
\usepackage[french]{babel}
\usepackage[T1]{fontenc}
\usepackage{lmodern}
\usepackage{fancyhdr}
\usepackage[]{titlesec}
\usepackage{soulutf8}
\usepackage{tikz}
\usetikzlibrary{arrows.meta}

\renewcommand\thesection{\Roman{section}}
\renewcommand\thesubsection{\thesection.\Alph{subsection}}

\makeatletter
\renewcommand\maketitle{
{\raggedright
{\large \bfseries Chapitre 2}\\[4ex]
{\Large \bfseries \@title}\\[4ex]}}
\makeatother
\setlength{\headheight}{15pt}

\setcounter{chapter}{2}

\titleformat{\chapter}{\large\bfseries}
\titleformat{\title}{\Large\bfseries}
\titleformat{\section}{\bfseries}{\ul{\thesection.\enspace}}{-0.15em}{\ul}
\titleformat{\subsection}{\bfseries}{\ul{\thesubsection.\enspace}}{-0.15em}{\ul}

\pagestyle{fancy}
\fancyhf{}
\fancyhead[LO,RE]{20 Septembre 2020}
\fancyhead[RO,LE]{Spé Mathématiques}
\fancyfoot[C]{\thepage}

\begin{document}
    \title{Continuité des Fonctions d'une Variable Réele}
    \maketitle
    \section{Définition}
        \hspace*{1\parindent}{\parbox[t]{\dimexpr\linewidth-1\parindent}{
        Soit $f$, définie sur un intervalle $I$, et soit $a$, un réel de $I$.
        La fonction $f$ est continue si et seulement si: \[\lim_{x \to a} f(x)=f(a)\]
        $f$ est continue sur l'intervalle $I$, si et seulement si, quel que soit le réel $x\in I$, $f$ est continue en $x$.
        \subsection{Exemple}
        }
        \hspace*{3\parindent}{\parbox[t]{\dimexpr\linewidth-3\parindent}{
            La fonction inverse est continue sur $]-\infty\,;0[$, et sur $]0\,;+\infty[$.\\
            La fonction ``Partie Enitère'' est définie sur $\mathbb{R}$, mais pas continue sur $\mathbb{R}$.
            \begin{center}\begin{tikzpicture}
                \draw[gray,very thin] (-.75,-.75) grid (3.75,3.75);
                \draw[thick, -{Latex[length=3mm]}] (0,0) -- (3.75,0) node[anchor=north east] {$x$};
                \draw[thick, -{Latex[length=3mm]}] (0,0) -- (0,3.75) node[anchor=north east] {$y$};
            \end{tikzpicture}\end{center}

        }
        
\end{document}