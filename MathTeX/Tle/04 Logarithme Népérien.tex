\documentclass{cours}

\title{Fonction Logarithme Népérien}
\author{Diego Van Overberghe}

\begin{document}
    \maketitle{4}

    \begin{Gpartie}{Définition de la Fonction ln}
        \begin{Spartie}{Théorème-Définition}
            Pour tout réel $x$ strictement positif, il existe un unique réel tel que :
            \[\me^a=x\]
            Le nombre $a$ est appelé logarithme népérien de $x$.
            
            La fonction qui à $x$ associe $a$ est appelée \og fonction logarithme népérien \fg{}, et se~note~$\ln$.

            Ainsi :
            \[\boxed{\me^a=x\iff a=\ln\,(x)\quad\text{pour $x>0$}}\]
        \end{Spartie}
        \begin{Spartie}{Démonstration}
            La fonction exponentielle est continue et strictement croissante sur $\mathbb{R}$.

            De plus :
            \[\lim_{x\to -\infty} \me^x=0\quad\text{ et }\quad\lim_{x\to +\infty} \me^x=+\infty\]

            D'après le corollaire du Théorème des Valeurs Intermédiaires, pour tout nombre $k\in\big]\,0~;\,+\infty~\big]$, il existe un unique réel $a\in\mathbb{R}$ tel que $\me^a=k$.$\quad\square$
        \end{Spartie}
        \begin{Spartie}{Remarques}
            La fonction $\ln$ est la fonction réciproque de $\exp$.

            On déduit que : \[\boxed{\ln\,(1)=0\quad\text{et}\quad\ln\,(\me)=1}\]
        \end{Spartie}
        \begin{Spartie}{Propriétés}
            \[\forall x\in\big]\,0~;\,+\infty~\big[,~\boxed{\me^{\ln\,(x)}=x}\]
            \[\forall x\in\mathbb{R},~\boxed{\ln\,\left(\me^x\right)=x}\]
        \end{Spartie}
    \end{Gpartie}
    \pagebreak
    \begin{Gpartie}{Étude de la Fonction ln}
        \begin{Spartie}{Continuité, Dérivabilité}
            \begin{SSpartie}{Propriété}
                \begin{enumerate}[(1)]
                    \item La fonction $\ln$ est \emph{continue} sur $\big]\,0~;\,+\infty~\big]$
                    \item La fonction $\ln$ est \emph{dérivable} sur $\big]\,0~;\,+\infty~\big]$ et $\forall x\in\big]\,0~;\,+\infty~\big[$, $\boxed{\ln'(x)=\frac{1}{x}}$
                    \item La fonction $\ln$ est \emph{strictement croissante} sur $\big]\,0~;\,+\infty~\big[$
                \end{enumerate}
                \begin{SSSpartie}{Démonstration (2) (Facile)}
                    Admettre que si $f$ et $u$ sont dérivables, $f\big(u(x)\big)'=u'(x)\times f(u(x))$.
                    \[\big(\me^{\ln\,(x)}\big)'=\ln'(x)\times \me^{\ln\,(x)}=\ln'(x)\times x\]
                    \[(x)'=1\]
                    \[\ln'(x)\times x=1\iff\ln'(x)=\frac{1}{x}\quad\square\]
                    
                \end{SSSpartie}
                \begin{SSSpartie}{Démonstration (2) (Complète)} 
                    On étudie le taux d'accroissement, en admettant que $\ln$ est continue :
                    \[\frac{\ln\,(x+h)-\ln\,(x)}{h}\]

                    Posons :
                    \[\begin{cases}
                        u=\ln\,(x+h)\qquad\text{donc}\qquad x+h=\me^u \\ v=\ln\,(x)\phantom{+h)}\text{\qquad donc\qquad} x=\me^v
                    \end{cases}\qquad\text{et}\qquad k=u-v\]

                    \[\begin{aligned}[t]
                        \frac{\ln\,(x+h)-\ln\,(x)}{h}&=\frac{k}{\me^u-\me^v}\qquad\text{or}\qquad\lim_{h\to0}k=0 \\
                        &=\frac{1}{\frac{\me^{v+h}-\me^v}{k}}
                    \end{aligned}\]

                    Or, $\lim\limits_{h\to0}\frac{\me^{v+h}-\me^v}{h}$ est l'expression de la dérivée de $\me$. La dérivée de $\me^v$ est $\me^v$ :
                    \[\frac{1}{\me^v}=\frac{1}{\me^{\ln\,(x)}}=\frac{1}{x}\]
                    \[\lim\limits_{h\to 0}\frac{\ln\,(x+h)-\ln\,(x)}{h}=\frac{1}{x}\quad\square\]

                \end{SSSpartie}
            \end{SSpartie}
            \begin{SSpartie}{Propriété}
                Soit $u$, une fonction dérivable et strictement positive sur $I$.

                Alors, la fonction $f:x\mapsto\ln\big(u(x)\big)$ et $\boxed{f':x\mapsto\frac{u'}{u}}$.
                \begin{SSSpartie}{Exemple}
                    Soit $f$ définie sur $\mathbb{R}$, par $f(x)=\ln\,(x^2+1)$.

                    $f$ est dérivable en $\mathbb{R}$ et $f'(x)=\frac{2x}{x^2+1}$.
                \end{SSSpartie}
            \end{SSpartie}
        \end{Spartie}
        \begin{Spartie}{Limites}
            \[\boxed{\lim_{x\to+\infty}\ln\,(x)=+\infty\quad\text{et}\quad\lim_{x\to 0}\ln\,(x)=-\infty}\]
            \begin{SSpartie}{Démonstration} 
                Soit $M>0$, en posant $A=\me^M,~\exists A>0$ tel que :
                \[\forall x\in\big]\,0~;\,+\infty~\big[,~x>A\implies\ln\,(x)>M\]
                En effet, $x>A\iff x>\me^M$. \\ Comme $\ln$ est strictement croissante :
                \[\begin{aligned}[t]
                    \ln\,(x)&>\ln\,\left(\me^M\right) \\
                    \ln\,(x)&>M\quad\square
                \end{aligned}\]
            \end{SSpartie}
        \end{Spartie}
        \begin{Spartie}{Représentation Graphique}
            \begin{center}
                \begin{tikzpicture}
                    \begin{axis}[
                        xmin=-0.5,  xmax=8.5,
                        ymin=-2.5,  ymax=2.5,
                        domain=-0.1:11,
                        xtick={1,2,3,4,5,6,7,8},
                        ytick={-2,-1,1,2}
                    ]
                        \addplot[color=blue, very thick, samples=100]{ln(x)};
                        \draw[red, dashed, thick]
                        (0,1) -- (e,1) node [dot] {} -- (e,0) node [label={[label distance=-1.5pt]270:$\me$}] {};
                    \end{axis}
                \end{tikzpicture}
                \parbox{\linewidth}{\captionof{figure}{Représentation Graphique de la Fonction ln}}
            \end{center}
        \end{Spartie}
    \end{Gpartie}
    \pagebreak
    \begin{Gpartie}{Propriétés Algébriques de la Fonction ln}
        \begin{Spartie}{Propriété Fondamentale}
            Quels que soient les réels $a$ et $b$, strictement positifs, 
            \[\boxed{\ln\,(a\times b)=\ln\,(a)+\ln\,(b)}\]
            \begin{SSpartie}{Démonstration}
                Rappel : \[X=Y\iff \me^X=\me^Y\]
                \[\me^{\ln\,(ab)}=ab\]
                \[\me^{\ln\,(a)+\ln\,(b)}=\me^{\ln\,(a)}\times \me^{\ln\,(b)}=a\times b\]

                Donc : \[\me^{\ln\,(ab)}=\me^{\ln\,(a)+\ln\,(b)}\iff\ln\,(ab)=\ln\,(a)+\ln\,(b)\]
            \end{SSpartie}
        \end{Spartie}
        \begin{Spartie}{Conséquences}
            \begin{enumerate}[(1)]
                \item Quels que soient les réels $a$ et $b$, strictement positifs :
                \[\boxed{\ln\,\left(\frac{a}{b}\right)=\ln\,(a)-\ln\,(b)\text{\quad et \quad}\ln\,\left(\frac{1}{b}\right)=-\ln\,(b)}\]
                \begin{SSpartie}{Démonstration (1)} 
                    Pour tous réels $a$ et $b$ strictement positifs : \[\ln\,(a)=\ln\,\left(\frac{a}{b}\times b\right)=\ln\,\left(\frac{a}{b}\right)+\ln\,(b)\]
                    D'où : \[\ln\,\left(\frac{a}{b}\right)=\ln\,(a)-\ln\,(b)\quad\square\]

                    La deuxième égalité est le cas particulier où $a=1$.$\quad\square$
                \end{SSpartie}
                \item Quels que soient les réels $a_1, a_2,\dotsc, a_n$, strictement positifs : 
                \[\boxed{\ln\,(a_1a_2\dotsb a_n)=\ln\,(a_1)+\ln\,(a_2)+\dotsb+\ln\,(a_n)}\]
                \pagebreak
                \begin{SSpartie}{Démonstration par Récurrence (2)}
                    \begin{SSSpartie}{Initialisation}
                        \[\ln\,(a_1)=\ln\,(a_1)\]
                    \end{SSSpartie}
                    \begin{SSSpartie}{Hérédité}
                        Supposons que, pour $n$ nombres strictement positifs :
                        \[\ln\,(a_1a_2\dotsb a_n)=\ln\,(a_1)+\ln\,(a_2)+\dotsb+\ln\,(a_n)\]
                        Alors :
                        \[\begin{aligned}[t]
                            \ln\,(a_1a_2\dotsb a_na_{n+1})&=\ln\,(a_1a_2\dotsb a_n)+\ln\,(a_{n+1})\quad\text{Propriété Fond.} \\
                            &=\ln\,(a_1)+\ln\,(a_2)+\dotsb+\ln\,(a_n)+\ln\,(a_{n+1})\quad\square
                        \end{aligned}\]
                    \end{SSSpartie}
                \end{SSpartie}
                \item De plus, $\forall a\in\big]\,0~;\,+\infty~\big[,~\forall n\in\mathbb{Z}$ :
                \[\boxed{\ln\,\left(a^n\right)=n\ln\,(a)}\]
                \begin{SSpartie}{Démonstration (3)}
                    \begin{itemize}
                        \item Dans le cas où $n$ est positif, c'est le cas particulier :
                        \[\ln\,(a_1a_2\dotsb a_n)=\ln\,(a_1)+\ln\,(a_2)+\dotsb+\ln\,(a_n)\]
                        \[a_1=a_2=\dotsb=a_n=a\]
                        \[\ln\,(a^n)=\ln\,(a)+\ln\,(a)+\dotsb+\ln\,(a)=n\ln\,(a)\quad\square\]
                        \item Dans le cas où $n$ est négatif, on prend $m=-n$
                        \[\ln\,(a^n)=\ln\,(a^{-m})=\ln\,\left(\frac{1}{a^m}\right)=-\ln\,(a^m)=-m\ln\,(a)=n\ln\,(a)\quad\square\]
                    \end{itemize}
                \end{SSpartie}
                \item Finalement, $\forall a,~a>0$ :
                \[\ln\,\left(\sqrt{a}\right)=\frac{1}{2}\ln\,(a)\]
                \begin{SSpartie}{Démonstration (4)}
                    \[\ln\,(a)=\ln\,\left(\sqrt{a^2}\right)=2\ln\,\left(\sqrt{a}\right)\]
                    \[\ln\,\left(\sqrt{a}\right)=\frac{1}{2}\ln\,(a)\quad\square\]

                \end{SSpartie}
            \end{enumerate}
        \end{Spartie}
    \end{Gpartie}
    \begin{Gpartie}{Résolution d'Inéquations du type $a^n>M$}
        \vspace{-2ex}
        \begin{Spartie}{Exemple}
            On sait que : \[\lim_{n\to+\infty}3^n=+\infty\]
            On cherche le plus petit entier $n$ tel que $3^n>1\,000$ :
            \[\begin{aligned}[t]
                &\quad\ln\,\left(3^n\right)>\ln\,(1\,000) \\
                \iff&\quad n\ln\,(3)>\ln\,(1\,000) \\
                \iff&\quad n>\frac{\ln\,(1000)}{\ln\,(3)}\approx 6{,}3
            \end{aligned}\]

            On trouve $n\geq 7$, $7$ est le plus petit entier $n$ tel que $3^n>1\,000$.
        \end{Spartie}
        \vspace{-2ex}
        \begin{Spartie}{Exemple}
            On sait que : \[\lim_{n\to+\infty}\left(\frac{1}{2}\right)^n=0\text{\qquad \big(car $0<\tfrac{1}{2}<1$\big)}\]
            
            On veut résoudre $\left(\frac{1}{2}\right)^n\leq 0{,}0001$ :
            \[\begin{aligned}[t]
                \iff&\quad\ln\,\Big(\big(\tfrac{1}{2}\big)^n\Big)\leq\ln\,(0{,}0001)\text{\qquad $\ln$ strictement croissante} \\
                \iff&\quad n\ln\,\big(\tfrac{1}{2}\big)\leq\ln\,(0{,}0001) \\
                \iff&\quad n\geq\frac{\ln\,(0{,}0001)}{\ln\,\left(\frac{1}{2}\right)}\approx13{,}3
            \end{aligned}\]

            Donc, $n\geq 14$
        \end{Spartie}
    \end{Gpartie}
    \vspace{-2ex}
    \begin{Gpartie}{Logarithme Décimal} 
        \vspace{-2ex}
        \begin{Spartie}{Définition} 
            La fonction logarithme décimal, notée $\log$, est définie sur $\big]\,0~;\,+\infty~\big[$, par :
            \[\log\,(x)=\frac{\ln\,(x)}{\ln\,(10)}\]
        \end{Spartie}
        \vspace{-4ex}
        \begin{Spartie}{Propriétés} 
            De part de sa définition, la fonction $\log$ a les mêmes propriétés algébriques et analytiques que la fonction $\ln$ (dérivable, strictement croissante).
        \end{Spartie}
        \begin{Spartie}{Remarque} 
            La fonction $\log$ est la fonction réciproque de $f:x\mapsto 10^x$.
        \end{Spartie}
    \end{Gpartie}
\end{document}