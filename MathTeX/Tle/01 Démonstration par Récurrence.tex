\documentclass{cours}

\title{Suites : La Démonstration par Récurrence}
\author{Diego Van Overberghe}

\begin{document}
    \maketitle{1}

    % \setlength{\baselineskip}{1.5em}
    \begin{Gpartie}{Exemple}
        On considère la suite $\left(u_n\right)$, définie pour tout entier naturel $n$, par\quad$u_n=4^n-1$ :
            \[\left(u_n\right)=\big\{~0~;~3~;~15~;~63~;~255~;~\dotso~\big\}\]
        On remarque que tous ces nombres sont des multiples de 3. On se demande si pour tout entier naturel $n$, \quad$4^n-1$\quad est un multiple de 3.
    \end{Gpartie}
    \begin{Gpartie}{Axiome de Récurrence}
        Soit $\Pde(n)$, une propriété dépendante de l'entier naturel $n$.\\
        Si :
        \[\begin{cases}
            \Pde(0)\text{ est vraie (Initialisation)} \\ \forall k\in\mathbb{N},~\Pde(k)\text{ vraie}\implies\Pde(k+1)\text{ vraie (hérédité)}
        \end{cases}\]

        Alors : \[P(n)~\text{est vraie pour tout entier naturel.}\]
        Reprenons notre exemple. 
        
        Initialisation : La propriété : \og $4^n-1$ est un multiple de 3 \fg{}, est vraie pour $n=0$. La~propriété est donc initialisée. \\ 
        Hérédité : Supposons que la propriété est vraie à un certain rang $k$, (fixe). C'est~l'hypothèse de la récurrence, et montrons qu'alors, elle est vraie au rang $k+1$.
        
        L'Hypothèse de récurrence est donc : \og $4^n-1$ est un multiple de 3 \fg{}. C'est-à-dire qu'il existe un entier $p$ tel que $\quad4^n-1=3p\quad$ et donc $\quad4^n=3p+1$.
        \vspace*{-1ex}\[\begin{aligned}[t]
            4^{k+1}-1&=4\times 4^k-1 &\\
            &= 4\times\left(3p+1\right)-1 \quad \text{(Hypothèse de récurrence)} &\\
            &=12p+4-1 &\\
            &=12p+3 &\\
            &=3\left(4p+1\right) \quad \text{(Il s'agit bien d'un multiple de 3)}
        \end{aligned}\]
        Donc, la propriété est héréditaire et on a démontré par récurrence que \quad$\forall n\in\mathbb{N},~4^n-1$\quad est un multiple de 3.
    \end{Gpartie}
\end{document}