\documentclass{cours}

\let\oldfrac\frac
\let\frac\tfrac % lazy inline style in trig functions

\title{Complément sur la Trigonométrie}

\begin{document}
    \maketitle{19}

    \begin{Gpartie}{Rappels}
        $\forall x\in\mathbb{R},~
        \begin{aligned}[t]&\cos\left(\frac{\pi}{2}-x\right)=\sin(x) \\
            &\sin\left(\frac{\pi}{2}-x\right)=\cos(x) \\
            &\cos^2(x)+\sin^2(x)=1 \\
            &\sin(x)=-\sin(-x) \\
            &\cos(x)=\cos(-x)
        \end{aligned}$
    \end{Gpartie}

    \begin{Gpartie}{Formules d'Addition et Duplication} 
        \begin{Spartie}{Formule d'Addition} 
            \begin{center}
                % \begin{tikzpicture}
                    \includegraphics[width=5cm]{example-image}
                % \end{tikzpicture}
                \parbox{\linewidth}{\captionof{figure}{\centering Représentation de $\vec{u}$ et $\vec{b}$ dans le Cercle Trigonométrique}}
            \end{center}
            $\lvert\lvert\vec{u}\rvert\rvert=\lvert\lvert\vec{v}\rvert\rvert=1$ \\ $\left(~\overrightarrow{OI}~;~\vec{u}~\right)=a$ \\ $\left(~\overrightarrow{OI}~;~\vec{v}~\right)=b$ \\ $\vec{u}\begin{psmallmatrix}\cos a \\ \sin a\end{psmallmatrix}$ \\ $\vec{v}\begin{psmallmatrix}\cos a \\ \sin a\end{psmallmatrix}$

            Or, $\vec{u}\cdot\vec{v}=\cos(a)\cos(b)+\sin(a)\sin(b)$
            
            Mais, $\vec{u}\cdot\vec{v}=\lvert\lvert\vec{u}\rvert\rvert\times\lvert\lvert\vec{v}\rvert\rvert\times\cos(~\vec{u}~;~\vec{v}~)\quad$ avec $(~\vec{u}~;~\vec{v}~)=a-b$ ou $b-a$ 

            Ainsi : \[\cos(a-b)=\cos(a)\cos(b)+\sin(a)\sin(b)\]
            \begin{SSpartie}{Exemple} 
                $\begin{aligned}[t]
                    \cos\left(\frac{\pi}{12}\right)=\cos\left(\frac{\pi}{3}-\frac{\pi}{4}\right) &= \cos\left(\frac{\pi}{3}\right)\cos\left(\frac{\pi}{4}\right)+\sin\left(\frac{\pi}{3}\right)\sin\left(\frac{\pi}{4}\right) \\
                    &= \dfrac{1}{2}\times\dfrac{\sqrt{2}}{2}+\dfrac{\sqrt{3}}{2}\times\dfrac{\sqrt{2}}{2} \\
                    &= \dfrac{\sqrt{2}+\sqrt{6}}{4}
                \end{aligned}$
            \end{SSpartie}
            Autres Formules :
            \[\cos(a+b)=\cos\big(a-(-b)\big)=\cos(a)\cos(b)-\sin(a)\sin(b)\]
            $\begin{aligned}[t]
                \sin(a+b) &= \cos\left(\frac{\pi}{2}-(a+b)\right) \\
                &= \cos\left(\left(\frac{\pi}{2}-a\right)-b\right) \\
                &= \cos\left(\frac{\pi}{2}-a\right)\cos(b)+\sin\left(\frac{\pi}{2}-a\right)\sin(b) \\
            \end{aligned}$
            \[\sin(a+b)=\sin(a)\cos(b)+\cos(a)\sin(b)\]
            $\begin{aligned}[t]
                \sin(a-b) &= \cos\left(\frac{\pi}{2}-(a-b)\right) \\
                &= \cos\left(\left(\frac{\pi}{2}-a\right)-(-b)\right) \\
                &= \cos\left(\frac{\pi}{2}-a\right)\cos(-b)+\sin\left(\frac{\pi}{2}-a\right)\sin(-b) \\
            \end{aligned}$
            \[\sin(a-b)=\sin(a)\cos(b)-\cos(a)\sin(b)\]
        \end{Spartie}
        \begin{Spartie}{Formule de Duplication (cas $b=a$)} 
            $\begin{aligned}[t]
                \sin(2a) &= \sin(a)\cos(a)+\cos(a)\sin(a) \\
                &=2\sin(a)\cos(a)
            \end{aligned}$

            $\begin{aligned}[t]
                \cos(2a) &= \cos(a)\cos(a)-\sin(a)\sin(a) \\
                &=\cos^2(a)-\sin^2(a) \\
                &= 2\cos^2(a)-1 \\ 
                &= 1-2\sin^2(a)
            \end{aligned}$

            $\begin{aligned}[t]
                &\quad2\cos^2(a)=1-2\sin^2(a) \\
                \iff&\quad\cos^2(a)=\dfrac{\cos(2a)+1}{2} \\
                \iff&\quad\sin^2(a)=\dfrac{1-\cos(2a)}{2}
            \end{aligned}$
        \end{Spartie}
    \end{Gpartie}
    \pagebreak
    \begin{Gpartie}{Dérivabilité de cosinus et sinus} 
        \let\frac\oldfrac %  undo lazy inline
        \begin{Spartie}{Préambule} 
            \begin{center} 
                % \begin{tikzpicture}
                    \includegraphics[width=5cm]{example-image}
                % \end{tikzpicture}
                \parbox{\linewidth}{\captionof{figure}{\centering Représentation de la Situation}}
            \end{center}
            On veut encadrer l'aire entre $OHM$ et $OIT$.

            $M\in\big]~0~;~\frac{\pi}{2}~\big[$ est associé au réel $x$ sur le cercle trigonométrique.

            $H$ est le projeté orthogonal de $M$ sur $\big(OI\big)$.

            $T$ est l'intersection de $\big[OM\big)$ et la tangente à $\mathcal{C}$ en $I$.

            On encadre l'aire du secteur angulaire du disque entre les aires des triangles $OHM$ et $OIT$.

            \begin{itemize}
                \item Secteur Angulaire :
                
                \begin{center}\begin{tabular}{ | m{0.1\linewidth} || *{2}{>{\centering\arraybackslash}m{0.1\linewidth}| }} \hline
                    Angle   & $2\pi$            & $x$           \\\hline
                    Aire    & $\pi\times 1^2$   & $\frac{x}{2}$  \\\hline
                \end{tabular}\end{center}
                \parbox{\linewidth}{\captionof{figure}{\centering Tableau de l'Aire Angulaire du Disque en Fonction de $\widehat{OHM}$}}
    
                \item $\mathrm{OHM}$ : \[\mathcal{A}_\mathrm{OHM}=\frac{\cos(x)\sin(x)}{2}\]

                \item $\mathrm{OIT}$ :
                
                On utilise le théorème de Thalès dans les triangles $\mathrm{OHM}$ et $\mathrm{OIT}$ : 
                \[\begin{aligned}[t]
                    \frac{HM}{IT} &= \frac{OH}{OI} \\[1.5ex]
                    \frac{\sin(x)}{IT} &=\frac{\cos(x)}{1} \\[1.5ex]
                    IT &= \frac{\sin(x)}{\cos(x)}=\tan(x)
                \end{aligned}\]
                D'où :  \[\mathcal{A}_\mathrm{OIT}=\frac{\tan(x)}{2}=\frac{\sin(x)}{2\cos(x)}\]
            \end{itemize}
        
            \renewcommand*{\arraystretch}{2} % For upcoming array

            Donc :~
            % \[\begin{alignedat}[t]{3}
            %         &\quad\mathcal{A}_\mathrm{OHM} && \leq\mathcal{A}_\text{Secteur} && \leq\mathcal{A}_\mathrm{OIT} \\
            %     \iff&\quad\frac{\cos(x)\sin(x)}{2} && \leq\frac{x}{2} && \leq\frac{\sin(x)}{2\cos(x)} \\
            %     \iff&\quad\cos(x)\sin(x) && \leq x && \leq\frac{\sin(x)}{\cos(x)} \\
            %     \iff&\quad\cos(x) && \leq\frac{x}{\sin(x)} && \leq\frac{1}{\cos(x)} \\
            %     \iff&\quad\frac{1}{\cos(x)} && \geq\frac{\sin(x)}{x} && \geq\cos(x) 
            % \end{alignedat}\]
            \[\begin{array}{ r@{\,} c@{\,} c@{\,} c@{\,} l@{\,} }
                    &\quad\mathcal{A}_\mathrm{OHM}  & \leq & \mathcal{A}_\text{Secteur} & \leq\mathcal{A}_\mathrm{OIT} \\
                \iff&\quad\frac{\cos(x)\sin(x)}{2}  & \leq & \frac{x}{2}                & \leq\frac{\sin(x)}{2\cos(x)} \\
                \iff&\quad\cos(x)\sin(x)            & \leq & x                          & \leq\frac{\sin(x)}{\cos(x)} \\
                \iff&\quad\cos(x)                   & \leq & \frac{x}{\sin(x)}          & \leq\frac{1}{\cos(x)} \\
                \iff&\quad\frac{1}{\cos(x)}         & \geq & \frac{\sin(x)}{x}          & \geq\cos(x) 
            \end{array}\]
            Or : \[\lim\limits_{x\to0}\cos(x)=1\quad\text{et}\quad\lim\limits_{x\to0}\frac{1}{\cos(x)}=1\]

            D'après le Théorème des Gendarmes : \[\lim\limits_{x\to0}\frac{\sin(x)}{x}=1\]
            \vspace{2ex}
            Cherchons aussi \quad $\lim\limits_{x\to0}\frac{\cos(x)-1}{x}$ : 
            \[\begin{aligned}[t]
                \frac{\cos(x)-1}{x} &= \frac{\cos\left(\frac{2x}{2}\right)-1}{x} \\
                &= \frac{1-\sin^2\left(\frac{x}{2}\right)-1}{x} \\
                &= -\sin\left(\frac{x}{2}\right)\times\frac{\sin\left(\frac{x}{2}\right)}{\frac{x}{2}}
            \end{aligned}\]
            
            Or :  \[\lim\limits_{x\to0}\dfrac{\sin\left(\frac{x}{2}\right)}{x}=1\quad\text{(composition)}\] \[\lim\limits_{x\to0}-\sin\left(\frac{x}{2}\right)=0\]

            Donc, par produit : \[\lim\limits_{x\to0}\dfrac{\cos(x)-1}{x}=0\]
        \end{Spartie}
        \pagebreak
        \begin{Spartie}{Dérivabilité} 
            Soit un réel $a$ et $h$ non nul : 
            \[\begin{aligned}[t]
                \frac{\sin(a+h)-\sin(a)}{h} &= \frac{\sin(a)\cos(h)+\cos(a)\sin(h)-\sin(a)}{h} \\
                &= \frac{\sin(a)\left(\cos(h)-1\right)}{h}+\frac{\cos(a)\sin(h)}{h}
            \end{aligned}\]
            Or : \[\lim\limits_{h\to0}\frac{\cos(x)-1}{h}=0\qquad\lim\limits_{h\to0}\frac{\sin(h)}{h}=1\]
            Donc : \[\lim\limits_{h\to0}\frac{\sin(a+h)-\sin(a)}{h}=\cos(a)\iff\sin'=\cos\quad\square\]

            D'autre part : 
            \[\begin{aligned}[t]
                \frac{cos(a+h)-\cos(a)}{h} &= \frac{\cos(a)\cos(h)-\sin(a)\sin(h)-\cos(a)}{h} \\
                &= \frac{\cos(a)(\cos(h)-1)}{h}-\frac{\sin(a)\sin(h)}{h}
            \end{aligned}\]
            Or : \[\lim\limits_{h\to0}\frac{\cos(h)-1}{h}=0\qquad\lim\limits_{h\to0}\frac{\sin(h)}{h}=1\]
            Donc : \[\lim\limits_{h\to0}\frac{\cos(a+h)-\cos(a)}{h}=-\sin(a)\iff\cos'=-\sin\quad\square\]

        \end{Spartie}
    \end{Gpartie}
\end{document}