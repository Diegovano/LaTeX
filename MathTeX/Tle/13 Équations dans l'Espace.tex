\documentclass{cours}

\title{Équations dans l'Espace}

\begin{document}
    \maketitle{13}

    \begin{Gpartie}{Représentation Paramétrique d'une Droite} 
        Dans un repère $\left(O\,;\vec{\imath}\,,\vec{\jmath}\,,\vec{k}\right)$, soit $(d)$ la droite passant par un point $A\,\left(x_A\,; y_A\,; z_A\right)$ et de vecteur directeur $\vec{u}\,\big(a\,; b\,; c\big)$.

        Un point $M\,\left(x\,; y\,; z\right)$ appartient à $(d)$ si et seulement si il existe un réel $t$ tel~que~$\overrightarrow{AM}=t\vec{u}$.

        C'est-à-dire : $(S)\begin{cases} x=x_A+ta \\ y=y_A+tb \\ z=z_A+tc \end{cases}\left(t\in\mathbb{R}\right)$

        \begin{Spartie}{Définition} 
            Le système $(S)$ est une représentation paramétrique de la droite $(d)$. \\ $t$ est le paramètre de cette représentation.
            \begin{SSpartie}{Démonstration}
                Reprenons la situation précédente :
                
                $\overrightarrow{AM}\,\begin{psmallmatrix} x_M-x_A &=\ x-x_A \\ y_M-y_A &=\ y-y_A \\ z_M-z_A &=\ z-z_A \end{psmallmatrix}$

                $\begin{aligned}[t]
                    M\in(d)&\iff\overrightarrow{AM}=t\vec{u} \\
                    &\iff\begin{psmallmatrix} x-x_A \\ y-y_A \\ z-z_a \end{psmallmatrix}=t\begin{psmallmatrix} a \\ b \\ c \end{psmallmatrix} \\
                    &\iff\begin{cases}x-x_A=ta \\ y-y_A=tb \\ z-z_A= t c \end{cases} \\ %spaces to supress spellchecker
                    &\iff\begin{cases}x=x_A+ta \\ y=y_A+tb \\ z=z_A+ t c \end{cases}
                \end{aligned}$
            \end{SSpartie}
        \end{Spartie}
        \begin{Spartie}{Exemple} 
            La droite $(d)$ passant par $A\,\left(2\,;0\,;-1\right)$ et de vecteur directeur $\vec{u}~\begin{psmallmatrix}1 \\ -3 \\ 2\end{psmallmatrix}$ a pour représentation paramétrique : \[(d)\begin{cases} x=2+t \\ y=-3t \\ z=-1+2t
            \end{cases}\left(t\in\mathbb{R}\right)\]

            Les points $M\,\left(0\,;6\,;-5\right)$ et $N\,\left(1\,;3\,;2\right)$ appartiennent-ils à $(d)$ ?
            \begin{itemize}
                \item $M$ : $\begin{cases} 0=2+t \\ 6=-3t \\ -5=-1+2t \end{cases}M\in(d)$, car $-2$ est solution du système.
                \item $N$ : $\begin{cases} 1=2+t \\ 3=-3t \\ 2=-1+2t \end{cases}N\notin(d)$, car $-1$ est solution de $(1)$ et $(2)$ mais pas $(3)$.
            \end{itemize}
        \end{Spartie}
    \end{Gpartie}
    \begin{Gpartie}{Équation Cartésienne de Plan} 
        \begin{Spartie}{Rappel} 
            La plan $\mathcal{P}$ qui passe par un point $A$ et de vecteur normal $\vec{n}$ est l'ensemble des points $M$ tels que $\overrightarrow{AM}\cdot\vec{n}=0$.
        \end{Spartie}
        \begin{Spartie}{Théorème} 
            \begin{enumerate}[(1)]
                \item   L'ensemble des points $M\,\left(x\,; y\,; z\right)$ tels que $ax+by+cz+d=0$, où $a$, $b$, $c$ et $d$ sont quatre réels tels que $a$, $b$ et $c$ sont non nuls, est un plan de vecteur normal $\vec{n}\,\left(a\,, b\,, c\right)$.
                
                \item   Réciproquement : si $\vec{n}~\left(a\,, b\,, c\right)$ est un vecteur normal d'un plan $\mathcal{P}$, une équation cartésienne de ce plan est $ax+by+cz+d=0$ où $d$ est un réel.
                
                Autrement dit, pour tout point $M\,\left(x\,; y\,; z\right)$ de $\mathcal{P}$ vérifie $ax+by+cz+d=0$.
            \end{enumerate}
            \pagebreak
            \begin{SSpartie}{Démonstration} 
                \begin{enumerate}[(1)]
                    \item   Appelons $\mathcal{E}$ l'ensemble des points $M\,\left(x\,; y\,; z\right)$ tels que $ax+by+cz+d=0$. (on ne sait pas que c'est un plan) \\ Soit le vecteur $\vec{n}\,\left(a\,, b\,, c\right)$.
                    
                    Supposons $a\neq 0$, prenons le point $A\,\left(\frac{-d}{a}\,; 0\,; 0\right)$.

                    Vérifions que $A\in\mathcal{E}$ :

                    $\begin{aligned}[t]
                        A\in\mathcal{E}&\iff a\times\tfrac{-d}{a}+b\times0+c\times0+\left(-a\times\tfrac{-d}{a}-b\times0-c\times0\right)=0 \\
                        &\iff a\times\tfrac{-d}{a}-a\times\tfrac{-d}{a}=0
                    \end{aligned}$

                    C'est identique pour les points $\left(0\,; \frac{-d}{b}\,; 0\right)$ ou $\left(0\,; 0\,; \frac{-d}{c}\right)$.
                    
                    Donc, quel que soit $M~\left(x\,; y\,; z\right)\in\mathcal{E}$ :
                    \[\begin{aligned}[t]
                        \overrightarrow{AM}\cdot\vec{n}=0&\iff\begin{psmallmatrix} x-\frac{-d}{a} \\ y \\ z\end{psmallmatrix}\cdot \begin{psmallmatrix}a \\ b \\ c\end{psmallmatrix} \\ &\iff a x+by+cz+d=0 %supress spellcheck
                    \end{aligned}\]

                    Donc, tout point $M$ de $\mathcal{E}$ vérifie $\overrightarrow{AM}\cdot\vec{n}=0$, donc appartient au plan passant par $A$ et de vecteur normal $\vec{n}$. (c'est la caractérisation d'un plan) $\quad\square$

                    Ainsi, $\mathcal{E}\subset\mathcal{P}$.

                    \item   Soit $A\,\left(x_A\,; y_A\,; z_A\right)\in\mathcal{P}$. \\ Pour tout point $M\,\left(x\,; y\,; z\right)$ du plan $\mathcal{P}$ on calcule le produit scalaire $\overrightarrow{AM}\cdot\vec{n}$ qui est nul par définition (voir le rappel) d'un plan de vecteur normal $\vec{n}\,\left(a\,; b\,; c\right)$.
                    
                    On trouve $a\,\left(x-x_A\right)+b\left(y-y_A\right)+c\left(z-z_A\right)=0$ et donc $ax+by+cz+d=0$, où $d=-ax_A-by_A-cz_A.\quad\square$

                    Ainsi $\mathcal{P}\subset\mathcal{E}$.
                \end{enumerate}
                \vspace*{2ex}
                Comme $\mathcal{E}\subset\mathcal{P}$ et $\mathcal{P}\subset\mathcal{E}$, $\mathcal{P}=\mathcal{E}$.
            \end{SSpartie}
        \end{Spartie}
    \end{Gpartie}
\end{document}