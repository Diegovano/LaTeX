\documentclass[12pt, a4paper]{article}
\usepackage{exercise}
\usepackage{amsmath}
\usepackage{amsfonts}
\usepackage{mathtools}
\usepackage[shortlabels]{enumitem}
\usepackage[margin=2.5cm]{geometry}
\usepackage[french]{babel}
\usepackage[upright]{fourier}
\usepackage{tikz}
\usepackage{pgfplots}
\usepackage{caption}
\pgfplotsset{compat=1.17}
\usetikzlibrary{arrows.meta}

\renewcommand{\ExerciseHeader}{%
  \par\noindent
  \textbf{\large \ExerciseName\ \ExerciseHeaderNB\ExerciseHeaderTitle\ExerciseHeaderOrigin}%
  \par\nopagebreak\medskip
}
\renewcommand{\ExerciseName}{Exercice}

\begin{document}

    \title{Exercices Chapitre sur la Trigonométrie\footnote{Page 202 du Manuel Hatier}}
    \author{Diego Van Overberghe}
    \maketitle

\begin{Exercise}[number={45}]
    \begin{enumerate}
        \item   \begin{enumerate}[a)]
                \item $\frac{\pi}{4}\times8=2\pi$ \qquad Il faut donc multiplier par 8.
                \item $\frac{\pi}{4}\times12=3\pi$ \qquad Il faut donc multiplier par 12.
        \end{enumerate}
        \item   \begin{enumerate}[a)]
                \item $5\pi = \pi+2\times2\pi$
                \item $27\pi=\pi+13\times2\pi$
                \item $59\pi=-\pi+30\times2\pi$
        \end{enumerate}
    \end{enumerate}
\end{Exercise}

\begin{Exercise}[number={46}]
    \begin{enumerate}
        \item   \begin{enumerate}[a)]
                \item $\frac{\pi}{3}$ est associé au point A.
                \item $-\frac{\pi}{2}$ est associé au point Q.
                \item $4\pi$ est associé au point I.
                \item $-\pi$ est associé au point P.
                \item $-\frac{\pi}{4}$ est associé au point M.
                \item $\frac{13\pi}{6}$ est associé au point A.
                \item $-\frac{2\pi}{3}$ est associé au point K.
                \item $\frac{5\pi}{4}$ est associé au point H.
                \end{enumerate}
        \item   B: $\dfrac{\pi}{4}$; \ D: $\dfrac{2\pi}{3}$; \ E: $\dfrac{3\pi}{4}$; \ F: $\dfrac{5\pi}{6}$; \ G: $\dfrac{7\pi}{6}$; \ H: $\dfrac{5\pi}{4}$; \ L: $\dfrac{5\pi}{3}$; \ N: $\dfrac{11\pi}{6}$
    \end{enumerate}
\end{Exercise}

\begin{Exercise}[number={47}]
    \begin{enumerate}[a)]
        \item Faux. 0 est associé au point $(1;0)$, et $\pi$ est associé au point $(-1;0)$
        \item Vrai. $\dfrac{180}{\pi}\times\dfrac{2\pi}{5}=72^{\circ}$
        \item Vrai.
        \item Faux. $\dfrac{11\pi}{6}<2\pi$
        \item Vrai. Ajoutter $k\times2\pi,$ avec $k\in \mathbb{Z}$, ne fait que ajoutter une rotation entière, vu que le périmètre du cercle trigonométrique est $2\pi$
        \item Faux. $103^{\circ}$ est associé au nombre $\dfrac{103\pi}{180}$
    \end{enumerate}
\end{Exercise}

\begin{Exercise}[number={50}]
    \begin{enumerate}[a)]
        \item Le périmètre de la bobine est de $2\pi$ puisque son rayon est 1. Or $\dfrac{2019\pi}{2}=1009\pi+\dfrac{\pi}{2}$ \\ Soit $\frac{1014\pi}{2\pi}=513$ tours complets.
        \item La première extremité se situe du côté droite du cercle. Après les 513 tours complets, il reste $\frac{\pi}{2}$ cm de fil. Ceci correspond à un angle de $90^{\circ}$ en avant. L'autre éxtrémité du fil se situe donc du côté supérieur du cercle.
    \end{enumerate}
\end{Exercise}

\begin{Exercise}[number={52}]
    \begin{enumerate}[a)]
        \item Faux. Chaque point du cercle trigonométrique est associé à une infinité de réels. Par exemple, le point 0 est associé au meme point que le point $2\pi$
        \item Faux. Le point $\dfrac{\pi}{2}$ est associé au point de coordonnées $(0\ ;1)$, le point $-\dfrac{\pi}{2}$ est associé au point de coordonnées $(0\ ;-1)$
    \end{enumerate}
\end{Exercise}

\begin{Exercise}[number={59}]
    \begin{enumerate}[a)]
        \item $\cos{\ \dfrac{\pi}{3}}=\dfrac{1}{2}$ \quad et \quad $\sin{\ \dfrac{\pi}{3}}=\dfrac{\sqrt{3}}{2}$
        \item $\cos{\ -\dfrac{\pi}{2}}=0$ \quad et \quad $\sin{\ -\dfrac{\pi}{2}}=-1$
        \item $\cos{\ \dfrac{7\pi}{3}}=\dfrac{1}{2}$ \quad et \quad $\sin{\ \dfrac{7\pi}{3}}=\dfrac{\sqrt{3}}{2}$
        \item $\cos{\ -\pi}=-1$ \quad et \quad $\sin{\ -\pi}=0$
        \item $\cos{\ -\dfrac{\pi}{4}}=\dfrac{1}{2}$ \quad et \quad $\sin{\ -\dfrac{\pi}{4}}=-\dfrac{\sqrt{3}}{2}$
        \item $\cos{\ \dfrac{5\pi}{6}}=-\dfrac{\sqrt{3}}{2}$ \quad et \quad $\sin{\ \dfrac{5\pi}{6}}=\dfrac{1}{2}$
        \item $\cos{\ 0}=1$ \quad et \quad $\sin{\ 0}=1$
        \item $\cos{\ \dfrac{3\pi}{4}}=-\dfrac{\sqrt{2}}{2}$ \quad et \quad $\sin{\ \dfrac{3\pi}{4}}=\dfrac{\sqrt{2}}{2}$
        \item $\cos{\ \dfrac{5\pi}{3}}=\dfrac{1}{2}$ \quad et \quad $\sin{\ \dfrac{5\pi}{3}}=-\dfrac{\sqrt{3}}{2}$
        \item $\cos{\ \dfrac{3\pi}{2}}=0$ \quad et \quad $\sin{\ \dfrac{3\pi}{2}}=-1$
        \item $\cos{\ \dfrac{7\pi}{6}}=-\dfrac{\sqrt{3}}{2}$ \quad et \quad $\sin{\ \dfrac{7\pi}{6}}=-\dfrac{1}{2}$
        \item $\cos{\ -\dfrac{2\pi}{3}}=-\dfrac{1}{2}$ \quad et \quad $\sin{\ -\dfrac{2\pi}{3}}=-\dfrac{\sqrt{3}}{2}$
    \end{enumerate}
\end{Exercise}

\begin{Exercise}[number={60}]
    \begin{enumerate}[a)]
        \item Le point se situera dans le sécteur 4.
        \item Le point se situera dans le sécteur 2.
        \item Le point se situera dans le sécteur 3.
    \end{enumerate}
\end{Exercise}

\begin{Exercise}[number={61}]
    \begin{enumerate}
        \item   \begin{enumerate}[a)]
                \item Les points d'abscisse $\dfrac{1}{2}$ sont les points C et L. Ils sont associés aux réels $\dfrac{\pi}{3}$ et $-\dfrac{\pi}{3}$ réspéctivement. Leurs sinus sont égaux à $\dfrac{\sqrt{3}}{2}$ et à $-\dfrac{\sqrt{3}}{2}$ réspéctivement.
                \item Les points d'abscisse $-\dfrac{\sqrt{2}}{2}$ sont les points E et H. Ils sont associés aux réels $\dfrac{3\pi}{4}$ et $-\dfrac{3\pi}{4}$ réspéctivement. Leurs sinus sont égaux à $\dfrac{\sqrt{2}}{2}$ et à $-\dfrac{\sqrt{2}}{2}$ réspéctivement.
                \end{enumerate}
        \item   \begin{enumerate}[a)]
                \item Les points d'ordonée $\dfrac{1}{2}$ sont les points B et E. Ils sont associés aux réels $\dfrac{\pi}{4}$ et $\dfrac{3\pi}{4}$ réspéctivement. Leurs cosinus sont égaux à $\dfrac{\sqrt{3}}{2}$ et à $-\dfrac{\sqrt{3}}{2}$ réspéctivement.
                \item Les points d'ordonée $-\dfrac{\sqrt{3}}{2}$ sont les points K et L. Ils sont associés aux réels $\dfrac{3\pi}{3}$ et $\dfrac{5\pi}{3}$ réspéctivement. Leurs cosinus sont égaux à $-\dfrac{1}{2}$ et à $\dfrac{1}{2}$ réspéctivement.
                \end{enumerate}
    \end{enumerate}
\end{Exercise}

\begin{Exercise}[number={63}]
    \begin{enumerate}
        \item La réponse correcte est la b. $-\dfrac{\sqrt{2}}{2}$
        \item La réponse correcte est la a. $-1$
        \item La réponse correcte est la c. $-\dfrac{1}{2}$
        \item La réponse correcte est la a. $\dfrac{\pi}{3}$
        \item La réponse correcte est la b. Les solutions sont: $-\dfrac{\pi}{2}$ et $\dfrac{\pi}{2}$
    \end{enumerate}    
\end{Exercise}

\begin{Exercise}
    \begin{enumerate}[a)]
        \item $\cos{\ \dfrac{15\pi}{3}}=\dfrac{1}{2}$ \quad et \quad $\sin{\ \dfrac{15\pi}{3}}=-\dfrac{\sqrt{3}}{2}$
        \item $\cos{\ -\dfrac{5\pi}{2}}=-1$ \quad et \quad $\sin{\ -\dfrac{5\pi}{2}}=0$
        \item $\cos{\ -\dfrac{9\pi}{4}}=\dfrac{\sqrt{2}}{2}$ \quad et \quad $\sin{\ -\dfrac{9\pi}{4}}=-\dfrac{\sqrt{2}}{2}$
        \item $\cos{\ -\dfrac{28\pi}{3}}=\dfrac{1}{2}$ \quad et \quad $\sin{\ -\dfrac{28\pi}{3}}=-\dfrac{\sqrt{3}}{2}$
        \item $\cos{\ -\dfrac{7\pi}{6}}=\dfrac{\sqrt{3}}{2}$ \quad et \quad $\sin{\ -\dfrac{7\pi}{6}}=-\dfrac{1}{2}$
        \item $\cos{\ \dfrac{2018\pi}{4}}=0$ \quad et \quad $\sin{\ \dfrac{2018\pi}{4}}=1$
    \end{enumerate}
\end{Exercise}

\pagebreak

\begin{Exercise}[number={65}]
    \begin{enumerate}[a)]
        \item $\cos{\ \dfrac{101\pi}{6}}=-\dfrac{\sqrt{3}}{2}$ \quad et \quad $\sin{\ \dfrac{101\pi}{6}}=\dfrac{1}{2}$
        \item $\cos{\ \dfrac{43\pi}{4}}=-\dfrac{\sqrt{2}}{2}$ \quad et \quad $\sin{\ \dfrac{43\pi}{4}}=\dfrac{\sqrt{2}}{2}$
        \item $\cos{\ \dfrac{19\pi}{6}}=\dfrac{\sqrt{3}}{2}$ \quad et \quad $\sin{\ \dfrac{19\pi}{6}}=\dfrac{1}{2}$
        \item $\cos{\ -\dfrac{25\pi}{4}}=\dfrac{\sqrt{2}}{2}$ \quad et \quad $\sin{\ -\dfrac{25\pi}{4}}=-\dfrac{\sqrt{2}}{2}$
        \item $\cos{\ -\dfrac{21\pi}{2}}=0$ \quad et \quad $\sin{\ -\dfrac{21\pi}{2}}=-1$
        \item $\cos{\ -\dfrac{15\pi}{2}}=0$ \quad et \quad $\sin{\ -\dfrac{15\pi}{2}}=1$
        \item $\cos{\ \dfrac{1981\pi}{3}}=\dfrac{1}{2}$ \quad et \quad $\sin{\ \dfrac{1981\pi}{3}}=\dfrac{\sqrt{3}}{2}$
    \end{enumerate}
\end{Exercise}

\begin{Exercise}[number={65}]
    \begin{enumerate}[a)]
        \item $C=1+0+\dfrac{1}{2}+\dfrac{\sqrt{3}}{2}+\dfrac{\sqrt{2}}{2}=\dfrac{\sqrt{2}+\sqrt{3}+3}{2}$ \\ $S=0+1+\dfrac{\sqrt{3}}{2}+\dfrac{\sqrt{2}}{2}+\dfrac{1}{2}=\dfrac{\sqrt{2}+\sqrt{3}+3}{2}$
        \item Les deux nombres sont égaux. Ceci est cohérent, vu que $\cos{\ \dfrac{\pi}{2}}=\sin{\ 0}$, $\cos{\ \dfrac{\pi}{3}}=\sin{\ \dfrac{\pi}{6}}$ et $\cos{\ \dfrac{\pi}{4}}=\sin{\ \dfrac{\pi}{4}}$
    \end{enumerate}
\end{Exercise}

\begin{Exercise}[number={77}]
    Tout d'Abord, $\forall x\in[-2\pi\,;2\pi], -x\in[-2\pi\,;2\pi]$. Ceci est vrai pour chaque fonction qui suit.
    \begin{enumerate}[a)]
        \item On peut voir que la droite $x=0$ est un axe de symmétrie de la courbe $\mathcal{C}_1$. \\ Donc, $f_1(x)=f_1(-x)$, c'est-à-dire que la fonction est paire.
        \item On peut voir que l'origine est un point de symmétrie de la courbe $\mathcal{C}_2$. \\ Donc, $f_2(-x)=-f_2(x)$, c'est-à-dire que la fonction est impaire.
        \item La fonction $f_3$ est quelconque. Elle est ni paire, ni impaire.
        \item On peut voir que la droite $x=0$ est un axe de symmétrie de la courbe $\mathcal{C}_4$. \\ Donc, $f_4(x)=f_4(-x)$, c'est-à-dire que la fonction est paire.
    \end{enumerate}
\end{Exercise}

\begin{Exercise}[number={78}]
    Tout d'Abord, $\forall x\in[-2\pi\,;2\pi], -x\in[-2\pi\,;2\pi]$. Ceci est vrai pour chaque fonction qui suit.
    \begin{enumerate}[a)]
        \item On peut voir que l'origine est un point de symmétrie de la courbe $\mathcal{C}_1$. \\ Donc, $f_1(-x)=-f_1(x)$, c'est-à-dire que la fonction est impaire.
        \item On peut voir que la droite $x=0$ est un axe de symmétrie de la courbe $\mathcal{C}_2$. \\ Donc, $f_2(x)=f_2(-x)$, c'est-à-dire que la fonction est paire.
        \item On peut voir que la droite $x=0$ est un axe de symmétrie de la courbe $\mathcal{C}_3$. \\ Donc, $f_3(x)=f_3(-x)$, c'est-à-dire que la fonction est paire.
        \item On peut voir que la droite $x=0$ est un axe de symmétrie de la courbe $\mathcal{C}_4$. \\ Donc, $f_4(x)=f_4(-x)$, c'est-à-dire que la fonction est paire.
    \end{enumerate}
\end{Exercise}

\begin{Exercise}[number={80}]
    \begin{center}\begin{tikzpicture}[scale=0.8, domain=-7:7, smooth, samples=100]
        \draw[gray,very thin] (-7,-4) grid (7,4);
        \draw[thick, -{Latex[length=3mm]}] (-7,0) -- (7,0) node[anchor=north east, scale=0.75] {$x$};
        \draw[thick, -{Latex[length=3mm]}] (0,-4) -- (0,4) node[anchor=north east, scale=0.75] {$y$};
        \foreach \x in {-6,-5,-4,-3,-2,-1,0,1,2,3,4,5,6}
        \draw[line width=0.2mm] (\x, 0) -- (\x, -0.25);
        \foreach \x / \xtext in {-2*pi/-2\pi,-pi/-\pi,pi/\pi,2*pi/2\pi}
        \draw[line width=0.2mm] (\x, 0) -- (\x, 0.25) node[anchor=south, scale=0.7, fill=white] {$\xtext$};
        \foreach \x in {-3,-2,-1,1,2,3}
        \draw[line width=0.2mm] (-0.25, \x) -- (0, \x);
        \draw[darkgray, line width=0.5mm, -{Latex[length=2mm]}] (0,0) -- (0,1);
        \draw[darkgray, line width=0.5mm, -{Latex[length=2mm]}] (0,0) -- (1,0);
        
        \begin{axis}[anchor=origin,x=1cm, y=1cm,hide axis,restrict y to domain=-4:4]
            \addplot[red, very thick]{cos(deg(2*x))};
            \addplot[black!30!green, very thick]{sin(deg(3*x))};
            \addplot[blue, very thick]{2*sin(deg(x))-1};
            \addplot[purple, very thick]{x-cos(deg(x))};  
        \end{axis}
        \draw (0,0) node[anchor=north east, scale=0.9] {O};
        \draw (0.5,0) node[outer sep=10pt, anchor=north, scale=0.9] {$\vec{i}$}; 
        \draw (0,0.5) node[anchor=east, scale=0.9] {$\vec{j}$};
    \end{tikzpicture}\end{center}
    \parbox{\linewidth}{\captionof{figure}{Représentation Graphique des Fonctions $f$, $g$, $h$ et $k$}}
    \begin{enumerate}[a)]
        \item   \begin{enumerate}[1)]
                    \item Il s'agit de la courbe rouge.
                    \item On peut voir que la droite $x=0$ est un axe de symmétrie de la courbe rouge. \\ Donc, $f(-x)=f(x)$, c'est-à-dire que la fonction semble être paire.
                    \item $f(-x)=\cos(-2x) \qquad f(x)=\cos(2x)\iff f(x)=\cos(-2x)\iff f(-x)=f(x)$ \\ La fonction est donc bien paire.
                \end{enumerate}\smallbreak
        \item   \begin{enumerate}[1)]
                    \item Il s'agit de la courbe verte.
                    \item On peut voir que l'origine est un point de symmétrie de la courbe verte. \\ Donc, $g(-x)=-g(x)$, c'est-à-dire que la fonction semble être impaire.
                    \item $g(-x)=\sin(-3x) \quad -g(x)=-\sin(3x)\iff -g(x)=\sin(-3x)\iff g(-x)=-g(x)$ \\ la fonction est donc bien impaire
                \end{enumerate}\smallbreak
        \item   \begin{enumerate}[1)]
                    \item Il s'agit de la courbe bleue.
                    \item On a l'impression que la fonction est ni paire ni impaire.
                    \item $h(-x)=2\sin(-x)-1=-2\sin(x)-1\neq h(x)\neq -h(x)$ \\ La fonction est bien donc ni paire, ni impaire.
                \end{enumerate}\smallbreak
        \item   \begin{enumerate}[1)]
                    \item Il s'agit de la courbe bordeaux.
                    \item On a l'impression que la fonction est ni paire ni impaire.
                    \item $k(-x)=-x-cos(-x)=-x-cos(x)\neq k(x)\neq -k(x)$ \\ La fonction est bien donc ni paire, ni impaire.
                \end{enumerate}
    \end{enumerate}
\end{Exercise}

\pagebreak

\begin{Exercise}[number={81}]
    \begin{center}\begin{tikzpicture}[scale=0.8, domain=-7:7, smooth, samples=100]
        \draw[gray,very thin] (-7,-4) grid (7,4);
        \draw[thick, -{Latex[length=3mm]}] (-7,0) -- (7,0) node[anchor=north east, scale=0.75] {$x$};
        \draw[thick, -{Latex[length=3mm]}] (0,-4) -- (0,4) node[anchor=north east, scale=0.75] {$y$};
        \foreach \x in {-6,-5,-4,-3,-2,-1,0,1,2,3,4,5,6}
        \draw[line width=0.2mm] (\x, 0) -- (\x, -0.25);
        \foreach \x / \xtext in {-2*pi/-2\pi,-pi/-\pi,pi/\pi,2*pi/2\pi}
        \draw[line width=0.2mm] (\x, 0) -- (\x, 0.25) node[anchor=south, scale=0.7, fill=white] {$\xtext$};
        \foreach \x in {-3,-2,-1,1,2,3}
        \draw[line width=0.2mm] (-0.25, \x) -- (0, \x);
        \draw[darkgray, line width=0.5mm, -{Latex[length=2mm]}] (0,0) -- (0,1);
        \draw[darkgray, line width=0.5mm, -{Latex[length=2mm]}] (0,0) -- (1,0); 
        
        \begin{axis}[anchor=origin,x=1cm, y=1cm,hide axis,restrict y to domain=-4:4]
            \addplot[red, very thick]{cos(deg(x))*sin(deg(x))};
            \addplot[black!30!green, very thick]{(cos(deg(x))^2};
            \addplot[blue, very thick]{(sin(deg(x)))^2};
            \addplot[purple, very thick]{x+sin(deg(x))};  
        \end{axis}

        \draw (0,0) node[outer sep=12pt, anchor=north east, scale=0.9] {O};
        \draw (0.5,0) node[anchor=north, scale=0.9] {$\vec{i}$};
        \draw (0,0.5) node[anchor=east, scale=0.9] {$\vec{j}$};
    \end{tikzpicture}\end{center}
    \parbox{\linewidth}{\captionof{figure}{Représentation Graphique des Fonctions $f$, $g$, $h$ et $k$}}
    \begin{enumerate}[a)]
        \item   \begin{enumerate}[1)]
                    \item Il s'agit de la courbe rouge.
                    \item On peut voir que l'origine est un point de symmétrie de la courbe rouge. \\ Donc, $f(-x)=-f(x)$, c'est-à-dire que la fonction semble être impaire.
                    \item $f(-x)=\cos(-x)\sin(-x) \quad -f(x)=-\cos(x)\sin(x)\iff -f(x)=\cos(-x)\sin(-x)\\\iff f(-x)=-f(x)$ \quad La fonction est bien impaire.
                \end{enumerate}\smallbreak
        \item   \begin{enumerate}[1)]
                    \item Il s'agit de la courbe verte.
                    \item On peut voir que la droite $x=0$ est un axe de symmétrie de la courbe verte. \\ Donc, $g(-x)=g(x)$, c'est-à-dire que la fonction semble être paire.
                    \item $g(-x)=\left(\cos(-x)\right)^2 \quad g(x)=\left(\cos(x)\right)^2\iff g(x)=\left(\cos(-x)\right)^2\iff g(-x)=g(x)$ \\ La fonction est donc bien paire.
                \end{enumerate}\smallbreak
        \item   \begin{enumerate}[1)]
                    \item Il s'agit de la courbe bleue.
                    \item On peut voir que la droit $x=9$ est un axe de symmétrie de la courbe bleue. \\ Donc, $h(-x)=h(x)$, c'est-à-dire que la fonction semble être paire.
                    \item $h(-x)=\left(\sin(-x)\right)^2 \quad h(x)=\left(\sin(x)\right)^2\iff h(x)=\left(\sin(-x)\right)^2\iff h(-x)=h(x)$ \\ La fonction est donc bien paire.
                \end{enumerate}\smallbreak
        \item   \begin{enumerate}[1)]
                    \item Il s'agit de la droite bordeaux.
                    \item On peut voir que l'origine est un point de symmétrie de la courbe bordeaux. \\ Donc, $k(-x)=-k(x)$, c'est-à-dire que la fonction semble être impaire.
                    \item $k(-x)=-x+\sin(-x) \quad -k(x)=-x-\sin(x)\iff -k(x)=-x\sin(-x)\\\iff k(-x)=-k(x)$ \quad La fonction est bien impaire.
                \end{enumerate}
    \end{enumerate}
\end{Exercise}

\begin{Exercise}[number={82}]
    \begin{itemize}
        \item $\mathcal{C}_1\longrightarrow f(x)$
        \item $\mathcal{C}_2\longrightarrow h(x)$
        \item $\mathcal{C}_3\longrightarrow g(x)$
    \end{itemize}
\end{Exercise}

\begin{Exercise}[number={83}]
    \begin{enumerate}[1)]
        \item   \begin{enumerate}[a)]
                    \item Vrai. $f(x+2k\pi)=f(x)\quad\text{avec}\ \begin{cases}k\in\mathbb{Z}\end{cases}$
                    \item Vrai. Pour la même raison.
                \end{enumerate}
        \item Vrai. Rajoutter deux à $x$ revient à éffectuer un tour en plus du cercle trigonomérique,\\\phantom{Vrai.} et ne change donc pas la valeur de $f(x)$.
    \end{enumerate}
\end{Exercise}

\begin{Exercise}[number={88}]
    \begin{enumerate}[a)]
        \item Il semble que la période est autour de $6{,}2$. On observe par lecture graphique que $g(x)$ semble compris entre $1$ et $7$.
        \item   \begin{itemize}
                    \item   $\begin{aligned}[t]
                                &\quad g(x+y)=g(x) \\
                                \iff&\quad 4+3\cos(x+y)=4+3\cos(x) \\
                                \iff&\quad \cos(x+y)=\cos(x) \\
                                \iff&\quad y=2k\pi\quad k\in\mathbb{Z} \\
                            \end{aligned}$ \smallbreak
                            La période est donc de $2k\pi$, avec $k=1$, c'est-à-dire $2\pi\approx 6{,}3$ \medbreak
                    \item   $\begin{aligned}[t]
                                &\quad -1\leq\cos(x)\leq 1 \\
                                \iff&\quad -3\leq\ 3\cos(x)\leq 3 \\
                                \iff&\quad 1\leq\ 4+3\cos(x)\leq 70 \\
                            \end{aligned}$ \smallbreak
                            La conjecture est vérifiée. \medbreak
                \end{itemize}
    \end{enumerate}
\end{Exercise}

\end{document}