\documentclass[12pt]{book}
\usepackage{exercise}
\usepackage{amsmath}
\usepackage[shortlabels]{enumitem}
\usepackage{siunitx}
\usepackage{amsfonts}
\usepackage[margin=1in]{geometry}


\renewcommand{\ExerciseHeader}{%
  \par\noindent
  \textbf{\large \ExerciseName\ \ExerciseHeaderNB\ExerciseHeaderTitle\ExerciseHeaderOrigin}%
  \par\nopagebreak\medskip
}
\renewcommand{\ExerciseName}{Exercice}

\begin{document}

\begin{Exercise}[number={45}]
    \vskip 20px
    \begin{enumerate}
        \item   \begin{enumerate}[a)]
                \item $\frac{\pi}{4}\times8=2\pi$ \qquad Il faut donc multiplier par 8.
                \item $\frac{\pi}{4}\times12=3\pi$ \qquad Il faut donc multiplier par 12.
        \end{enumerate}
        \item   \begin{enumerate}[a)]
                \item $5\pi = \pi+2\times2\pi$
                \item $27\pi=\pi+13\times2\pi$
                \item $59\pi=-\pi+30\times2\pi$
        \end{enumerate}
    \end{enumerate}
\end{Exercise}

\begin{Exercise}[number={46}]
    \vskip 20px
    \begin{enumerate}
        \item   \begin{enumerate}[a)]
                \item $\frac{\pi}{3}$ est associé au point A.
                \item $-\frac{\pi}{2}$ est associé au point Q.
                \item $4\pi$ est associé au point I.
                \item $-\pi$ est associé au point P.
                \item $-\frac{\pi}{4}$ est associé au point M.
                \item $\frac{13\pi}{6}$ est associé au point A.
                \item $-\frac{2\pi}{3}$ est associé au point K.
                \item $\frac{5\pi}{4}$ est associé au point H.
                \end{enumerate}
        \item   B: $\dfrac{\pi}{4}$; \ D: $\dfrac{2\pi}{3}$; \ E: $\dfrac{3\pi}{4}$; \ F: $\dfrac{5\pi}{6}$; \ G: $\dfrac{7\pi}{6}$; \ H: $\dfrac{5\pi}{4}$; \ L: $\dfrac{5\pi}{3}$; \ N: $\dfrac{11\pi}{6}$
    \end{enumerate}
\end{Exercise}

\begin{Exercise}[number={47}]
    \begin{enumerate}[a)]
        \item Faux. 0 est associé au point $(1;0)$, et $\pi$ est associé au point $(-1;0)$
        \item Vrai. $\dfrac{180}{\pi}\times\dfrac{2\pi}{5}=\ang{72}$
        \item Vrai.
        \item Faux. $\dfrac{11\pi}{6}<2\pi$
        \item Vrai. Ajoutter $k\times2\pi,$ avec $k\in \mathbb{Z}$, ne fait que ajoutter une rotation entière, vu que le périmètre du cercle trigonométrique est $2\pi$
        \item Faux. \ang{103} est associé au nombre $\dfrac{103\pi}{180}$
    \end{enumerate}
\end{Exercise}

\begin{Exercise}[number={50}]
    \begin{enumerate}[a)]
        \item Le périmètre de la bobine est de $2\pi$ puisque son rayon est 1. Or $\dfrac{2019\pi}{2}=1009\pi+\dfrac{\pi}{2}$ \\ Soit $\frac{1014\pi}{2\pi}=513$ tours complets.
        \item La première extremité se situe du côté droite du cercle. Après les 513 tours complets, il reste $\frac{\pi}{2}$ cm de fil. Ceci correspond à un angle de \ang{90} en avant. L'autre éxtrémité du fil se situe donc du côté supérieur du cercle.
    \end{enumerate}
\end{Exercise}

\begin{Exercise}[number={52}]
    \begin{enumerate}[a)]
        \item Faux. Chaque point du cercle trigonométrique est associé à une infinité de réels. Par exemple, le point 0 est associé au meme point que le point $2\pi$
        \item Faux. Le point $\dfrac{\pi}{2}$ est associé au point de coordonnées $(0\ ;1)$, le point $-\dfrac{\pi}{2}$ est associé au point de coordonnées $(0\ ;-1)$
    \end{enumerate}
\end{Exercise}

\begin{Exercise}[number={59}]
    \begin{enumerate}[a)]
        \item $cos{\ \dfrac{\pi}{3}}=\dfrac{1}{2}$ \quad et \quad $sin{\ \dfrac{\pi}{3}}=\dfrac{\sqrt{3}}{2}$
        \item $cos{\ -\dfrac{\pi}{2}}=0$ \quad et \quad $sin{\ -\dfrac{\pi}{2}}=-1$
        \item $cos{\ \dfrac{7\pi}{3}}=\dfrac{1}{2}$ \quad et \quad $sin{\ \dfrac{7\pi}{3}}=\dfrac{\sqrt{3}}{2}$
        \item $cos{\ -\pi}=-1$ \quad et \quad $sin{\ -\pi}=0$
        \item $cos{\ -\dfrac{\pi}{4}}=\dfrac{1}{2}$ \quad et \quad $sin{\ -\dfrac{\pi}{4}}=-\dfrac{\sqrt{3}}{2}$
        \item $cos{\ \dfrac{5\pi}{6}}=-\dfrac{\sqrt{3}}{2}$ \quad et \quad $sin{\ \dfrac{5\pi}{6}}=\dfrac{1}{2}$
        \item $cos{\ 0}=1$ \quad et \quad $sin{\ 0}=1$
        \item $cos{\ \dfrac{3\pi}{4}}=-\dfrac{\sqrt{2}}{2}$ \quad et \quad $sin{\ \dfrac{3\pi}{4}}=\dfrac{\sqrt{2}}{2}$
        \item $cos{\ \dfrac{5\pi}{3}}=\dfrac{1}{2}$ \quad et \quad $sin{\ \dfrac{5\pi}{3}}=-\dfrac{\sqrt{3}}{2}$
        \item $cos{\ \dfrac{3\pi}{2}}=0$ \quad et \quad $sin{\ \dfrac{3\pi}{2}}=-1$
        \item $cos{\ \dfrac{7\pi}{6}}=-\dfrac{\sqrt{3}}{2}$ \quad et \quad $sin{\ \dfrac{7\pi}{6}}=-\dfrac{1}{2}$
        \item $cos{\ -\dfrac{2\pi}{3}}=-\dfrac{1}{2}$ \quad et \quad $sin{\ -\dfrac{2\pi}{3}}=-\dfrac{\sqrt{3}}{2}$
    \end{enumerate}
\end{Exercise}

\begin{Exercise}[number={60}]
    \begin{enumerate}[a)]
        \item Le point se situera dans le sécteur 4.
        \item Le point se situera dans le sécteur 2.
        \item Le point se situera dans le sécteur 3.
    \end{enumerate}
\end{Exercise}

\begin{Exercise}[number={61}]
    \begin{enumerate}
        \item   \begin{enumerate}[a)]
                \item Les points d'abscisse $\dfrac{1}{2}$ sont les points C et L. Ils sont associés aux réels $\dfrac{\pi}{3}$ et $-\dfrac{\pi}{3}$ réspéctivement. Leurs sinus sont égaux à $\dfrac{\sqrt{3}}{2}$ et à $-\dfrac{\sqrt{3}}{2}$ réspéctivement.
                \item Les points d'abscisse $-\dfrac{\sqrt{2}}{2}$ sont les points E et H. Ils sont associés aux réels $\dfrac{3\pi}{4}$ et $-\dfrac{3\pi}{4}$ réspéctivement. Leurs sinus sont égaux à $\dfrac{\sqrt{2}}{2}$ et à $-\dfrac{\sqrt{2}}{2}$ réspéctivement.
                \end{enumerate}
        \item   \begin{enumerate}[a)]
                \item Les points d'ordonée $\dfrac{1}{2}$ sont les points B et E. Ils sont associés aux réels $\dfrac{\pi}{4}$ et $\dfrac{3\pi}{4}$ réspéctivement. Leurs cosinus sont égaux à $\dfrac{\sqrt{3}}{2}$ et à $-\dfrac{\sqrt{3}}{2}$ réspéctivement.
                \item Les points d'ordonée $-\dfrac{\sqrt{3}}{2}$ sont les points K et L. Ils sont associés aux réels $\dfrac{3\pi}{3}$ et $\dfrac{5\pi}{3}$ réspéctivement. Leurs cosinus sont égaux à $-\dfrac{1}{2}$ et à $\dfrac{1}{2}$ réspéctivement.
                \end{enumerate}
    \end{enumerate}
\end{Exercise}

\begin{Exercise}[number={63}]
    \begin{enumerate}
        \item La réponse correcte est la b. $-\dfrac{\sqrt{2}}{2}$
        \item La réponse correcte est la a. $-1$
        \item La réponse correcte est la c. $-\dfrac{1}{2}$
        \item La réponse correcte est la a. $\dfrac{\pi}{3}$
        \item La réponse correcte est la b. Les solutions sont: $-\dfrac{\pi}{2}$ et $\dfrac{\pi}{2}$
    \end{enumerate}    
\end{Exercise}

\begin{Exercise}
    \begin{enumerate}[a)]
        \item $cos{\ \dfrac{15\pi}{3}}=\dfrac{1}{2}$ \quad et \quad $sin{\ \dfrac{15\pi}{3}}=-\dfrac{\sqrt{3}}{2}$
        \item $cos{\ -\dfrac{5\pi}{2}}=-1$ \quad et \quad $sin{\ -\dfrac{5\pi}{2}}=0$
        \item $cos{\ -\dfrac{9\pi}{4}}=\dfrac{\sqrt{2}}{2}$ \quad et \quad $sin{\ -\dfrac{9\pi}{4}}=-\dfrac{\sqrt{2}}{2}$
        \item $cos{\ -\dfrac{28\pi}{3}}=\dfrac{1}{2}$ \quad et \quad $sin{\ -\dfrac{28\pi}{3}}=-\dfrac{\sqrt{3}}{2}$
        \item $cos{\ -\dfrac{7\pi}{6}}=\dfrac{\sqrt{3}}{2}$ \quad et \quad $sin{\ -\dfrac{7\pi}{6}}=-\dfrac{1}{2}$
        \item $cos{\ \dfrac{2018\pi}{4}}=0$ \quad et \quad $sin{\ \dfrac{2018\pi}{4}}=1$
    \end{enumerate}
\end{Exercise}

\begin{Exercise}[number={65}]
    \begin{enumerate}[a)]
        \item $cos{\ \dfrac{101\pi}{6}}=-\dfrac{\sqrt{3}}{2}$ \quad et \quad $sin{\ \dfrac{101\pi}{6}}=\dfrac{1}{2}$
        \item $cos{\ \dfrac{43\pi}{4}}=-\dfrac{\sqrt{2}}{2}$ \quad et \quad $sin{\ \dfrac{43\pi}{4}}=\dfrac{\sqrt{2}}{2}$
        \item $cos{\ \dfrac{19\pi}{6}}=\dfrac{\sqrt{3}}{2}$ \quad et \quad $sin{\ \dfrac{19\pi}{6}}=\dfrac{1}{2}$
        \item $cos{\ -\dfrac{25\pi}{4}}=\dfrac{\sqrt{2}}{2}$ \quad et \quad $sin{\ -\dfrac{25\pi}{4}}=-\dfrac{\sqrt{2}}{2}$
        \item $cos{\ -\dfrac{21\pi}{2}}=0$ \quad et \quad $sin{\ -\dfrac{21\pi}{2}}=-1$
        \item $cos{\ -\dfrac{15\pi}{2}}=0$ \quad et \quad $sin{\ -\dfrac{15\pi}{2}}=1$
        \item $cos{\ \dfrac{1981\pi}{3}}=\dfrac{1}{2}$ \quad et \quad $sin{\ \dfrac{1981\pi}{3}}=\dfrac{\sqrt{3}}{2}$
    \end{enumerate}
\end{Exercise}

\pagebreak %for spacing may need to remove later

\begin{Exercise}
    \begin{enumerate}[a)]
        \item $C=1+0+\dfrac{1}{2}+\dfrac{\sqrt{3}}{2}+\dfrac{\sqrt{2}}{2}=\dfrac{\sqrt{2}+\sqrt{3}+3}{2}$ \\ $S=0+1+\dfrac{\sqrt{3}}{2}+\dfrac{\sqrt{2}}{2}+\dfrac{1}{2}=\dfrac{\sqrt{2}+\sqrt{3}+3}{2}$
        \item Les deux nombres sont égaux. Ceci est cohérent, vu que $cos{\ \dfrac{\pi}{2}}=sin{\ 0}$, $cos{\ \dfrac{\pi}{3}}=sin{\ \dfrac{\pi}{6}}$ et $cos{\ \dfrac{\pi}{4}}=sin{\ \dfrac{\pi}{4}}$
        
    \end{enumerate}
    
\end{Exercise}

\end{document}