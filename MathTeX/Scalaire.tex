%&exerciseformat.fmt
\documentclass[12pt, a4paper]{article}
\usepackage{exercise}
\usepackage{amsmath}
\usepackage{mathtools}
\usepackage[shortlabels]{enumitem}
\usepackage{siunitx}
\usepackage{amsfonts}
\usepackage[margin=1in]{geometry}
\usepackage{esvect}
\usepackage{enumitem}
\usepackage{minted}
\usepackage[upright]{fourier}
\usepackage[french]{babel}
%\usepackage{showframe}
%\csname endofdump \endcsname
%\endofdump

% New definition of square root:
% it renames \sqrt as \oldsqrt
\let\oldsqrt\sqrt
% it defines the new \sqrt in terms of the old one
\def\sqrt{\mathpalette\DHLhksqrt}
\def\DHLhksqrt#1#2{%
\setbox0=\hbox{$#1\oldsqrt{#2\,}$}\dimen0=\ht0
\advance\dimen0-0.2\ht0
\setbox2=\hbox{\vrule height\ht0 depth -\dimen0}%
{\box0\lower0.64pt\box2}}

\renewcommand{\ExerciseHeader}
{
    \par\noindent
    \textbf{\large \ExerciseName\ \ExerciseHeaderNB\ExerciseHeaderTitle\ExerciseHeaderOrigin}%
    \par\nopagebreak\medskip
}
\renewcommand{\ExerciseName}{Exercice}

\DeclarePairedDelimiter\norm{\lVert}{\rVert}


\begin{document}
    \title{Exercices Chapitre sur le Produit Scalaire\footnote{Page 233 du manuel Hatier}}
    \author{Diego Van Overberghe}
    \date{4 Juin 2020}
    \maketitle
    \begin{Exercise}[number={34}]
            \begin{enumerate}[a)]
                \item $\overrightarrow{AB}\cdot\overrightarrow{AC}=AB\times AC\times\cos{\ \dfrac{\pi}{6}}=AB\times AC\times \dfrac{\sqrt{3}}{2}=\dfrac{20\sqrt{3}}{2}$
                \item $\overrightarrow{AB}\cdot\overrightarrow{AC}=AB\times AH=5$ \\ H étant le projeté orthogonal de AC sur [AB].
                \item $\overrightarrow{AB}\cdot\overrightarrow{AC}=AB\times AH=20$ \\ H étant le projeté orthogonal de AC sur la demie-droite [AB)
                \item $\widehat{BAC}=60^{\circ}=\dfrac{\pi}{3}$ (angles alternes-internes) \\ $\overrightarrow{AB}\cdot\overrightarrow{AC}=AB\times AC\times\cos{\ \dfrac{\pi}{3}}=12$
                \item Non Traité
            \end{enumerate}
    \end{Exercise}

\begin{Exercise}[number={35}]
    \begin{enumerate}[a)]
        \item $-2\vec{u}\cdot(3\vec{u}+\vec{v})=-6\vec{u}^2-2\vec{u}\cdot\vec{v}=-166$
        \item $-8\vec{u}^2+14\vec{u}\cdot\vec{v}-3\vec{v}^2=-155$
        \item Demander Correction%$\norm{\vec{u}+\vec{v}}=\sqrt{\norm{\vec{u}}^2+\norm{\vec{v}}^2}=\sqrt{34}$
        \item Demander Correction
    \end{enumerate}
\end{Exercise}

\begin{Exercise}[number={44}]
    \begin{enumerate}[a)]
        \item La projéction orthogonale de $\overrightarrow{AC}$ sur $(AB)$ est $\overrightarrow{AO}$
        \item La projéction orthogonale de $\overrightarrow{BD}$ sur $(DC)$ est $\overrightarrow{DO}$
        \item La projéction orthogonale de $\overrightarrow{OC}$ sur $(BC)$ est $\overrightarrow{OC}$
        \item La projéction orthogonale de $\overrightarrow{AD}$ sur $(CD)$ est $\overrightarrow{OD}$
        \item La projéction orthogonale de $\overrightarrow{OB}$ sur $(EF)$ est $\overrightarrow{OE}$
        \item La projéction orthogonale de $\overrightarrow{AB}$ sur $(EF)$ est $\overrightarrow{FE}$
    \end{enumerate}
\end{Exercise}
        
\pagebreak %spacing might have to remove later

\begin{Exercise}[number={45}]
    \begin{enumerate}[a)]
        \item $\overrightarrow{AB}\cdot\overrightarrow{DC}=1$
        \item $\overrightarrow{AB}\cdot\overrightarrow{CD}=-1$
        \item $\overrightarrow{OB}\cdot\overrightarrow{OD}=-\dfrac{1}{2}$
        \item $\overrightarrow{DB}\cdot\overrightarrow{DC}=\dfrac{3\sqrt{2}}{2}$
        \item $\overrightarrow{DO}\cdot\overrightarrow{DC}=\dfrac{1}{2}$
        \item $\overrightarrow{AO}\cdot\overrightarrow{CB}=-\dfrac{1}{2}$
    \end{enumerate}
\end{Exercise}

\begin{Exercise}[number={46}]
    \begin{enumerate}[a)]
        \item $\overrightarrow{EF}\cdot\overrightarrow{EG}=EF\times EG\times\cos{\ \dfrac{\pi}{2}}=12\sqrt{2}$
        \item $\overrightarrow{EF}\cdot\overrightarrow{EG}=EF\times EG\times\cos{\ \dfrac{5\pi}{6}}=-\dfrac{15\sqrt{3}}{2}$
    \end{enumerate}
\end{Exercise}

\begin{Exercise}[number={48}]
    \begin{enumerate}[a)]
        \item $\vec{u}\begin{psmallmatrix}2\\-1\end{psmallmatrix},\quad\vec{v}\begin{psmallmatrix}4\\5\end{psmallmatrix}:$ \quad $\vec{u}\cdot\vec{v}=x_{\vec{u}}\times x_{\vec{v}}+y_{\vec{u}}\times y_{\vec{v}}=3$
        \item $\vec{u}\begin{psmallmatrix}\frac{1}{2}\\-3\end{psmallmatrix},\quad\vec{v}\begin{psmallmatrix}\frac{1}{4}\\\frac{1}{3}\end{psmallmatrix}:$ \quad $\vec{u}\cdot\vec{v}=x_{\vec{u}}\times x_{\vec{v}}+y_{\vec{u}}\times y_{\vec{v}}=\dfrac{-7}{8}$
        \item $\vec{u}=3\overrightarrow{\imath}+2\overrightarrow{\jmath}$, et $\vec{v}=3\overrightarrow{\imath}+\overrightarrow{\jmath}:$ \quad $\vec{u}\cdot\vec{v}=x_{\vec{u}}\times x_{\vec{v}}+y_{\vec{u}}\times y_{\vec{v}}=11$
    \end{enumerate}
\end{Exercise}

\begin{Exercise}[number={49}]
    \begin{enumerate}[a)]
        \item Faux. Si les deux vecteurs ont un sens opposé, alors le produit scalaire sera l'opposé du produit de leurs normes.
        \item Faux. Il suffit d'imaginer deux vecteurs ayant le même projeté orthogonal sur un vecteur. Ces deux derniers sont différents, et pourtant leur produit scalaire est identique.
        \item Faux. $\norm{\vec{u}}^2+\norm{\vec{v}}^2-2\vec{u}\cdot\vec{v}<0\iff\norm{\vec{u}+\vec{v}}^2<0$ \\ Or, un nombre au carré est positif.
        \item Vrai. $\vec{u}^2=\vec{u}\cdot\vec{u}=\norm{\vec{u}}\times\norm{\vec{u}}\times\cos{\ 0}\footnote{L'angle entre deux vecteurs identiques est $0^{\circ}$}=\norm{\vec{u}}^2$
    \end{enumerate}  
\end{Exercise}

\begin{Exercise}[number={50}]
    \begin{enumerate}[a)]
        \item $\overrightarrow{AB}\cdot\overrightarrow{AC}=-9$
        \item $\overrightarrow{AB}\cdot\overrightarrow{AC}=-12$
        \item $\overrightarrow{AB}\cdot\overrightarrow{AC}=8$
        \item $\overrightarrow{AB}\cdot\overrightarrow{AC}=-10$
    \end{enumerate}
\end{Exercise}

\begin{Exercise}[number={51}]
    \begin{enumerate}[a)]
        \item $\vec{u}\cdot\vec{v}=12$
        \item $\vec{t}\cdot\vec{w}=-12$
        \item $\vec{m}\cdot\vec{h}=2$
        \item $\vec{w}\cdot\vec{u}=16$
        \item $\vec{v}\cdot\vec{w}=12$
        \item $\vec{m}\cdot\vec{u}=12$
    \end{enumerate}
\end{Exercise}

\begin{Exercise}[number={52}]
    \begin{enumerate}
        \item \begin{enumerate}[a)]
                    \item $\norm{\vec{u}}=4$ \quad $\norm{\vec{v}}=3\sqrt{2}$
                    \item $\vec{u}\cdot\vec{v}=12$
                    \item $cos{\ \alpha}=\dfrac{\vec{u}\cdot\vec{v}}{\norm{\vec{u}}\times \norm{\vec{v}}}=\dfrac{\sqrt{2}}{2} \quad \alpha= 45^{\circ}$
                    \end{enumerate}
        \item \begin{enumerate}[a)]
                    \item $\norm{\vec{u}}=2\sqrt{2}$ \quad $\norm{\vec{v}}=4$
                    \item $\vec{u}\cdot\vec{v}=-8$
                    \item $cos{\ \alpha}=\dfrac{\vec{u}\cdot\vec{v}}{\norm{\vec{u}}\times \norm{\vec{v}}}=-\dfrac{\sqrt{2}}{2} \quad \alpha= 135^{\circ}$
                    \end{enumerate}
        \item \begin{enumerate}[a)]
                    \item $\norm{\vec{u}}=\sqrt{5}$ \quad $\norm{\vec{v}}=3$
                    \item $\vec{u}\cdot\vec{v}=3$
                    \item $cos{\ \alpha}=\dfrac{\vec{u}\cdot\vec{v}}{\norm{\vec{u}}\times \norm{\vec{v}}}=\dfrac{\sqrt{5}}{5} \quad \alpha\approx 63^{\circ}$
                    \end{enumerate}
        \item \begin{enumerate}[a)]
                    \item $\norm{\vec{u}}=2\sqrt{2}$ \quad $\norm{\vec{v}}=\sqrt{10}$
                    \item $\vec{u}\cdot\vec{v}=4$
                    \item $cos{\ \alpha}=\dfrac{\vec{u}\cdot\vec{v}}{\norm{\vec{u}}\times \norm{\vec{v}}}=\dfrac{\sqrt{5}}{5} \quad \alpha\approx 63^{\circ}$
                    \end{enumerate}
    \end{enumerate}
\end{Exercise}

\begin{Exercise}[number={53}]
    \begin{enumerate}
        \item \begin{enumerate}[a)]
                    \item Ce n'est pas possible. L'angle est inférieur à $90^{\circ}$ mais le produit scalaire est négatif.
                    \item $AC=\dfrac{\overrightarrow{AB}\cdot\overrightarrow{AC}}{AB\times\cos{\dfrac{3\pi}{4}}}=2\sqrt{2}$
                    \end{enumerate}
        \item \begin{enumerate}[a)]
                    \item $cos{\ \alpha}=\dfrac{\overrightarrow{AB}\cdot\overrightarrow{AC}}{AB\times AC}=-\dfrac{\sqrt{2}}{2} \quad \alpha=135^{\circ}$
                    \item Ce n'est pas possible. $AB\times AC = 10$, \quad Or, \quad $\overrightarrow{AB}\cdot\overrightarrow{AC}=15$
                    \item $AB\times AC = \overrightarrow{AB}\cdot\overrightarrow{AC}$, \quad Donc, \quad $\alpha=0^{\circ}$
                    \end{enumerate}
    \end{enumerate}
\end{Exercise}

\pagebreak

\begin{Exercise}[number={54}]
    \begin{enumerate}[a)]
        \item $5\vec{u}^2+9\vec{u}\cdot\vec{v}-2\vec{v}^2=117$
        \item $\vec{u}^2+2\vec{u}\cdot\vec{v}+\vec{v}^2=30$
    \end{enumerate}
\end{Exercise}

\begin{Exercise}[number={57}]
    \begin{enumerate}[a)]
        \item $\overrightarrow{AB}\cdot\overrightarrow{AC}=\dfrac{1}{2}(AB^2+AC^2-BC^2)=20$
        \item $\overrightarrow{AB}\cdot\overrightarrow{AC}=\dfrac{1}{2}(AB^2+AC^2-BC^2)=8$
        \item Le triangle est réctangle en B. Le projeté orthogonal de C sur [AB] est B. \smallskip \\ $\overrightarrow{AB}\cdot\overrightarrow{AC}=AB\times AB=36$ \bigskip \\ou \bigskip \\D'Après le théorème de Pythagore: $BC^2=AC^2-AB^2=64$ \smallskip \\ Alors \quad $\overrightarrow{AB}\cdot\overrightarrow{AC}=36$
        \item $BC=AD=7$ \quad Donc, $\overrightarrow{AB}\cdot\overrightarrow{AC}=\dfrac{1}{2}(AB^2+AC^2-BC^2)=-4$
    \end{enumerate}
\end{Exercise}

\begin{Exercise}[number={58}]
    \begin{enumerate}[a)]
        \item $\vec{u}\cdot\vec{v}=x_{\vec{u}}\times x_{\vec{v}}+y_{\vec{u}}\times y_{\vec{v}}=4$
        \item $-4\vec{u}\cdot\vec{v}=-16$
        \item $-2\vec{u}\cdot\vec{v}=-8$
        \item $\vec{u}+\vec{v}=\begin{psmallmatrix}2\\-4\end{psmallmatrix}$ \quad $\vec{u}-\vec{v}=\begin{psmallmatrix}0\\-2\end{psmallmatrix}$ \smallskip \\ $\begin{psmallmatrix}2\\-4\end{psmallmatrix}\cdot\begin{psmallmatrix}0\\-2\end{psmallmatrix}=8$
    \end{enumerate}
\end{Exercise}

\begin{Exercise}[number={60}]
    \begin{enumerate}[a)]
        \item $\overrightarrow{AB}\cdot\overrightarrow{AC}=-AB\times AC\implies$ A,B et C sont alignés.
        \item $\vec{u}\cdot\vec{v}=0\implies\vec{u}=0$ ou $\vec{v}=0$
        \item $\vec{u}=3\vec{v}\implies\vec{u}^2=9\vec{v}^2$
        \item $\norm{\vec{u}}=0\iff\vec{u}=\overrightarrow{0}$
    \end{enumerate}
\end{Exercise}

\begin{Exercise}[number={61}]
    \begin{enumerate}[a)]
        \item $\overrightarrow{AB}\cdot\overrightarrow{AC}=AB\times AC\times\cos{\ 40^{\circ}}\approx15{,}3$
        \item D'Après le théorème d'Al-Kashi, $a^2=b^2+c^2-2bc\times\cos{\ \widehat{A}}$. \smallskip \\ Donc, $cos{\ \widehat{A}}=\dfrac{b^2+c^2-a^2}{2bc}$ \smallskip \\ $\overrightarrow{AB}\cdot\overrightarrow{AC}=AB\times AC\times \dfrac{6^2+4^2-3^2}{2\times 6\times 4}=21{,}5$
        \item Le triangle est isocèle, donc, le projeté orthogonal de B sur [AC] se situe au centre de ce ségment. $\overrightarrow{AB}\cdot\overrightarrow{AC}=6\times 3=18$
        \item On imagine le point E qui forme le carré ADCE, avec une diagonale de 3, et donc un côté de $\frac{3}{\sqrt{2}}$ \\ $\overrightarrow{AB}\cdot\overrightarrow{AC}=5\times \dfrac{3}{\sqrt{2}}=\dfrac{15}{\sqrt{2}}$
        \item On assume que ADCB est un parallélogramme. On imagine le point E qui forme le réctangle HDCE, [EB] $=1$. E est le projeté orthogonal de C sur [AE]. $\overrightarrow{AB}\cdot\overrightarrow{AC}=30$
        \item $(\overrightarrow{AB}\ ;\overrightarrow{AC})=(\pi-\dfrac{2\pi}{3})=-\dfrac{\pi}{3})$ \quad et \quad $\overrightarrow{AB}\cdot\overrightarrow{AC}=7\times\cos{\ -\dfrac{-\pi}{3}=\dfrac{7}{2}}$
        
    \end{enumerate}
\end{Exercise}

\begin{Exercise}[number={63}]
    \begin{enumerate}[a)]
        \item \begin{itemize}[leftmargin=3cm]
                        \item[Méthode 1:] On utilise la relation de Chasles.
                            \begin{flalign*}
                                \overrightarrow{EC}\cdot\overrightarrow{ED}&=(\overrightarrow{EA}+\overrightarrow{AC})\cdot(\overrightarrow{EB}+\overrightarrow{BD}) &\\ 
                                &=\overrightarrow{EA}\cdot\overrightarrow{EB}+\overrightarrow{EA}\cdot\overrightarrow{BD}+\overrightarrow{AC}\cdot\overrightarrow{EB}+\overrightarrow{AC}\cdot\overrightarrow{BD} &\\
                                &= 9{,}25 &
                            \end{flalign*}
                        \item[Méthode 2:] Posons le repère $(A\ ;\vec{\imath}\ ;\vec{\jmath}\ )$ où $\begin{psmallmatrix}\vec{\imath}=\frac{1}{6}\vec{AB}\\\vec{\jmath}=\frac{1}{4}\overrightarrow{AC}\end{psmallmatrix}$ \medbreak
                        $\begin{cases}\overrightarrow{EC}\begin{psmallmatrix}x_C-x_E\\y_C-y_E\end{psmallmatrix}\\\overrightarrow{ED}\begin{psmallmatrix}x_D-x_E\\y_D-y_E\end{psmallmatrix}\end{cases}\iff\begin{cases}\overrightarrow{EC}\begin{psmallmatrix}-1{,}5\\4\end{psmallmatrix}\\\overrightarrow{ED}\begin{psmallmatrix}4{,}5\\4\end{psmallmatrix}\end{cases}$ \medbreak
                        $\overrightarrow{EC}\cdot\overrightarrow{ED}=x_{\overrightarrow{EC}}\times x_{\overrightarrow{ED}}+y_{\overrightarrow{EC}}\times y_{\overrightarrow{ED}}=9{,}25$ \bigbreak
                        \item[Angle $\widehat{DEC}$:] $\overrightarrow{EC}\cdot\overrightarrow{ED}=EC\times ED\times\cos{\ \widehat{DEC}}$ \smallskip \\ $EC=\sqrt{x^{2}_{\overrightarrow{EC}}+y^{2}_{\overrightarrow{EC}}}=\sqrt{18{,}25}$ \\ $ED=\sqrt{x^{2}_{\overrightarrow{ED}}+y^{2}_{\overrightarrow{ED}}}=\sqrt{36{,}25}$ \smallskip \\ $cos{\ \widehat{DEC}}=\dfrac{\overrightarrow{EC}\cdot\overrightarrow{ED}}{EC\times ED}=\dfrac{9{,}25}{\sqrt{661{,}5625}}$ \medskip \\ $\widehat{DEC}=cos^{-1}{\left(\ \dfrac{9{,}25}{\sqrt{661{,}5625}}\right)\approx 68{,}92^{\circ}}$
                    \end{itemize} \bigbreak
        \item \begin{itemize}[leftmargin=3cm]
                        \item[Méthode 1:] On voit que B est le projeté orthogonal de E sur [DB], donc, \medskip \\ $\overrightarrow{DB}\cdot\overrightarrow{DE}=DB\times DB=16$ \bigbreak
                        \item[Méthode 2:] Dans $(A;\overrightarrow{\imath};\overrightarrow{\jmath})$: $\overrightarrow{DB}(0\,;-4) \quad \overrightarrow{DE}(-4{,}5\,;-4)$ \smallskip \\ $\overrightarrow{DB}\cdot\overrightarrow{DE}=16$ \bigbreak
                        \item[Longeur de BF:] Aire DBC $=\dfrac{DB\times DE}{2}=\dfrac{DF\times DE}{2}$ donc $DB\times BE=BF\times DE$ \medskip \\ $BF=\dfrac{DB\times BE}{DE}\approx2{,}99$
                    \end{itemize}
    \end{enumerate}  
\end{Exercise}

\pagebreak

\begin{Exercise}[number={65}]
    \begin{enumerate}[a)]
        \item Faux. Les vecteurs sont colinéaires. $\vec{u}\cdot\vec{v}=x_{\vec{u}} x_{\vec{v}}+y_{\vec{u}} y_{\vec{v}}=-13$ \\ $\vec{u}\cdot\vec{v}\neq0\iff\lnot(\vec{u}\perp \vec{v})$
        \item Vrai. $\vec{u}\cdot\vec{v}=x_{\vec{u}} x_{\vec{v}}+y_{\vec{u}} y_{\vec{v}}=0$ \\ $\vec{u}\cdot\vec{v}=0\iff \vec{u}\perp \vec{v}$ \quad (Parce que $\vec{u}\neq0$ et $\vec{v}\neq0$)
        \item Faux. Pour $a=2$, on a $\vec{u}\begin{psmallmatrix}2\\-1\end{psmallmatrix}$ et $ \vec{v}\begin{psmallmatrix}0\\1\end{psmallmatrix}$. Or $x_{\vec{u}}x_{\vec{v}}+y_{\vec{u}}y_{\vec{v}}=-1$ \\ $\vec{u}\cdot\vec{v}\neq0\iff\lnot(\vec{u}\perp \vec{v})$
        \item On a $\vec{u}\begin{psmallmatrix}2\\-3\end{psmallmatrix}$ et $\vec{v}\begin{psmallmatrix}6\\4\end{psmallmatrix}$. $x_{\vec{u}}x_{\vec{v}}+y_{\vec{u}}y_{\vec{v}}=0$ \\ $\vec{u}\cdot\vec{v}=0\iff \vec{u}\perp \vec{v}$ \quad (Parce que $\vec{u}\neq0$ et $\vec{v}\neq0$)
    \end{enumerate}
\end{Exercise}

\begin{Exercise}[number={66}]
    $\vec{u}\cdot\vec{v}=0\iff x_{\vec{u}}x_{\vec{v}}+y_{\vec{u}}y_{\vec{v}}=0$
    \begin{enumerate}[a)]
        \item $\vec{u}\perp\vec{v}$ pour $\vec{v}\begin{psmallmatrix}-2\\1\end{psmallmatrix}$
        \item $\vec{u}\perp\vec{v}$ pour $\vec{v}\begin{psmallmatrix}4\\3\end{psmallmatrix}$
        \item $\vec{u}\perp\vec{v}$ pour $\vec{v}\begin{psmallmatrix}1\\-2\end{psmallmatrix}$
        \item $\vec{u}\perp\vec{v}$ pour $\vec{v}\begin{psmallmatrix}2\\\sqrt{2}\end{psmallmatrix}$
        \item $\vec{u}\perp\vec{v}$ pour $\vec{v}\begin{psmallmatrix}-a\\1\end{psmallmatrix}$
    \end{enumerate}
\end{Exercise}

\begin{Exercise}[number={68}]
    \begin{enumerate}[a)]
        \item Vrai. Si les deux vecteurs sont orthogonaux, alors, leur produit scalaire sera nul.
        \begin{flalign*}
            &\quad (\vec{u}+\vec{v})\cdot(\vec{u}-\vec{v})=0 &&\\
            \iff&\quad \vec{u}^2-\vec{v}^2=0 &&\\
            \iff&\quad \norm{\vec{u}}^2-\norm{\vec{v}}^2=0 &&\\
            \iff&\quad \norm{\vec{u}}=\norm{\vec{v}}
        \end{flalign*}
        \item Vrai. 
        \item Vrai. Il s'agit tout simplement du théorème de Pythagore.
        \item Vrai. Ici, il s'agit de la réciproque du théorème de Pythagore.
    \end{enumerate}
\end{Exercise}

\begin{Exercise}[number={69}]
        Si ABC est réctangle, alors, $\overrightarrow{AB}\cdot\overrightarrow{AC}=0$. \bigbreak 
        $\begin{cases}\overrightarrow{AB}\begin{psmallmatrix}x_B-x_A\\y_B-y_A\end{psmallmatrix}\\[0.2cm]\overrightarrow{BC}\begin{psmallmatrix}x_C-x_B\\y_C-y_B\end{psmallmatrix}\end{cases} \iff \begin{cases}\overrightarrow{AB}\begin{psmallmatrix}2\\1\end{psmallmatrix}\\[0.2cm]\overrightarrow{BC}\begin{psmallmatrix}2\\-3\end{psmallmatrix}\end{cases}$ \bigbreak
        $\overrightarrow{AB}\cdot\overrightarrow{BC}=x_{\overrightarrow{AB}}\times x_{\overrightarrow{BC}}+y_{\overrightarrow{AB}}\times y_{\overrightarrow{BC}}=1$ \qquad Donc, $\widehat{B}\neq 90^{\circ}$
\end{Exercise}

\begin{Exercise}[number={70}]
    \begin{enumerate}[a)]
        \item $\vec{u}\cdot\vec{v}=15+m$ \\ $\vec{u}\cdot\vec{v}=0\iff 15+m=0\iff m=-15$
        \item $\vec{u}\cdot\vec{v}=5m-6$ \\ $\vec{u}\cdot\vec{v}=0\iff 5m-6=0\iff m=\dfrac{6}{5}$
        \item $\vec{u}\cdot\vec{v}=8-m^2$ \\  $\vec{u}\cdot\vec{v}=0\iff 8-m^2=0\iff m=2\sqrt{2}$ \quad ou \quad $m=-2\sqrt{2}$
        \item $\vec{u}\cdot\vec{v}=0$ \\  $\vec{u}\cdot\vec{v}=0\iff m=m$
    \end{enumerate}
\end{Exercise}

\begin{Exercise}[number={72}]
    \begin{enumerate}[a)]
        \item $\begin{cases}\overrightarrow{AB}\begin{psmallmatrix}x_B-x_A\\y_B-y_A\end{psmallmatrix}\\[0.2cm] \overrightarrow{AD}\begin{psmallmatrix}x_D-x_A\\y_D-y_A\end{psmallmatrix} \\[0.2cm] \overrightarrow{BA}\begin{psmallmatrix}x_A-x_B\\y_A-y_B\end{psmallmatrix} \\[0.2cm] \overrightarrow{BC}\begin{psmallmatrix}x_C-x_B\\y_C-y_B\end{psmallmatrix}\end{cases} \iff \begin{cases}\overrightarrow{AB}\begin{psmallmatrix}-3\\1\end{psmallmatrix}\\[0.2cm] \overrightarrow{AD}\begin{psmallmatrix}-1\\-3\end{psmallmatrix} \\[0.2cm] \overrightarrow{BA}\begin{psmallmatrix}3\\-1\end{psmallmatrix} \\[0.2cm] \overrightarrow{BC}\begin{psmallmatrix}-2\\-6\end{psmallmatrix}\end{cases}$ \medbreak $\overrightarrow{AB}\cdot\overrightarrow{AD}=x_{\overrightarrow{AB}}x_{\overrightarrow{AD}}+y_{\overrightarrow{AB}}y_{\overrightarrow{AD}}=0 \\ \overrightarrow{BA}\cdot\overrightarrow{BC}=x_{\overrightarrow{BA}}x_{\overrightarrow{BC}}+y_{\overrightarrow{BA}}y_{\overrightarrow{BC}}=0$
        \item $\overrightarrow{AD}$ et $\overrightarrow{BC}$ sont colinéaires, puisque ils sont tous les deux orthogonaux avec $\overrightarrow{AB}$. Donc, [AD] et [BC] sont parallèles. Comme $AD\neq BC$ $(\overrightarrow{AD}=\frac{1}{2}\overrightarrow{BC})$, il ne s'agit pas d'un carré ou d'un rectangle, donc le quadrilatère est un trapèze.
    \end{enumerate}
\end{Exercise}

\begin{Exercise}[number={73}]
    \begin{enumerate}[a)]
        \item $\begin{cases}\overrightarrow{AB}\begin{psmallmatrix}x_B-x_A\\y_B-y_A\end{psmallmatrix}\\[0.2cm]\overrightarrow{CD}\begin{psmallmatrix}x_D-x_C\\y_D-y_C\end{psmallmatrix}\end{cases}\iff \begin{cases}\overrightarrow{AB}\begin{psmallmatrix}1\\1\end{psmallmatrix}\\[0.2cm]\overrightarrow{CD}\begin{psmallmatrix}-5\\4\end{psmallmatrix}\end{cases}$ \medbreak $\overrightarrow{AB}\cdot\overrightarrow{CD}=x_{\overrightarrow{AB}}x_{\overrightarrow{CD}}+y_{\overrightarrow{AB}}y_{\overrightarrow{CD}}=-1$ \smallbreak $\vec{u}\cdot\vec{v}\neq0\iff\lnot(\vec{u}\perp\vec{v})$ \smallbreak On peut donc conclure que (AB) et (CD) ne sont pas perpendiculaires.
        \item $\begin{cases}\overrightarrow{AB}\begin{psmallmatrix}x_B-x_A\\y_B-y_A\end{psmallmatrix}\\[0.2cm]\overrightarrow{CD}\begin{psmallmatrix}x_D-x_C\\y_D-y_C\end{psmallmatrix}\end{cases}\iff \begin{cases}\overrightarrow{AB}\begin{psmallmatrix}2\\-7\end{psmallmatrix}\\[0.2cm]\overrightarrow{CD}\begin{psmallmatrix}-7\\-2\end{psmallmatrix}\end{cases}$ \medbreak $\overrightarrow{AB}\cdot\overrightarrow{CD}=x_{\overrightarrow{AB}}x_{\overrightarrow{CD}}+y_{\overrightarrow{AB}}y_{\overrightarrow{CD}}=0$ \medbreak $\vec{u}\cdot\vec{v}=0\impliedby \vec{u}\perp \vec{v}$ \quad (Parce que $\vec{u}\neq0$ et $\vec{v}\neq0$) \smallbreak On peut donc conclure que (AB) et (CD) sont perpendiculaires.
        \item $\begin{cases}\overrightarrow{AB}\begin{psmallmatrix}x_B-x_A\\y_B-y_A\end{psmallmatrix}\\[0.2cm]\overrightarrow{CD}\begin{psmallmatrix}x_D-x_C\\y_D-y_C\end{psmallmatrix}\end{cases}\iff \begin{cases}\overrightarrow{AB}\begin{psmallmatrix}\sqrt{2}-\sqrt{3}\\-5\end{psmallmatrix}\\[0.2cm]\overrightarrow{CD}\begin{psmallmatrix}5\\\sqrt{2}-\sqrt{3}\end{psmallmatrix}\end{cases}$ \medbreak $\overrightarrow{AB}\cdot\overrightarrow{CD}=x_{\overrightarrow{AB}}x_{\overrightarrow{CD}}+y_{\overrightarrow{AB}}y_{\overrightarrow{CD}}=0$ \medbreak $\vec{u}\cdot\vec{v}=0\impliedby \vec{u}\perp \vec{v}$ \quad (Parce que $\vec{u}\neq\vec{0}$ et $\vec{v}\neq\vec{0}$) \smallbreak On peut donc conclure que (AB) et (CD) sont perpendiculaires.
    \end{enumerate}
\end{Exercise}

\begin{Exercise}[number={74}]
    \begin{enumerate}[a)]
     \item On pose le repère $(A\, ;\overrightarrow{\imath}\,;\overrightarrow{\jmath})$ où $\begin{psmallmatrix}\overrightarrow{\imath}=\frac{1}{4}\overrightarrow{AB}\\\overrightarrow{\jmath}=\frac{1}{4}\overrightarrow{AD}\end{psmallmatrix}$ \smallskip \\
        $A(0\,;0)$, $G(-2\,;4)$, $H(6\,;4)$ \medbreak
        $\begin{cases}\overrightarrow{AG}\begin{psmallmatrix}x_G-x_A\\y_G-y_A\end{psmallmatrix}\\[0.2cm]\overrightarrow{AH}\begin{psmallmatrix}x_H-x_A\\y_H-y_A\end{psmallmatrix}\end{cases} \iff\begin{cases}\overrightarrow{AG}\begin{psmallmatrix}-2\\4\end{psmallmatrix}\\[0.2cm]\overrightarrow{AH}\begin{psmallmatrix}6\\4\end{psmallmatrix}\end{cases}$ \medbreak
        $\overrightarrow{AG}\cdot\overrightarrow{AH}=x_{\overrightarrow{AG}}\times x_{\overrightarrow{AH}}+y_{\overrightarrow{AG}}\times y_{\overrightarrow{AH}}=4$ \qquad Donc, $\widehat{A}\neq 90^{\circ}$
    \end{enumerate}
\end{Exercise}

\begin{Exercise}[number={78}]
    \begin{enumerate}[1)]
        \item \begin{minted}[autogobble]{python}
            def orthogonaux(a,b,c,d):
                p=a*c+b*d
                if p==0:
                    return True
                else:
                    return False
        \end{minted}
        \item \begin{enumerate}[a)]
                    \item True
                    \item False
                    \end{enumerate}
    \end{enumerate}
\end{Exercise}

\begin{Exercise}[number={79}]
    \begin{enumerate}[1)]
        \item \begin{enumerate}[a)]
                        \item \begin{itemize}[leftmargin=3cm]
                                        \item[Méthode 1:] $\begin{cases}\overrightarrow{AH}\begin{psmallmatrix}x_H-x_A\\y_H-y_A\end{psmallmatrix}\\[0.2cm]\overrightarrow{CB}\begin{psmallmatrix}x_B-x_C\\y_B-y_C\end{psmallmatrix}\end{cases}\iff\begin{cases}\overrightarrow{AH}\begin{psmallmatrix}x_H\\y_H-3\end{psmallmatrix}\\[0.2cm]\overrightarrow{CB}\begin{psmallmatrix}4\\-4\end{psmallmatrix}\end{cases}$ \medbreak $\overrightarrow{AH}\cdot\overrightarrow{CB}=x_{\overrightarrow{AH}}x_{\overrightarrow{CB}}+y_{\overrightarrow{AH}}y_{\overrightarrow{CB}}=4x_H-4y_H+12$ \medbreak
                                        \item[Méthode 2:] La hauteur issue de BC passe par le point H et A. \\ Donc, \quad $([AH]\perp[CB])\iff\overrightarrow{AH}\cdot\overrightarrow{CB}=0$
                                    \end{itemize}
                    \item On a donc,
                                \begin{flalign*}
                                    \quad&\quad\overrightarrow{AH}\cdot\overrightarrow{CB}=0 &\\
                                    \iff&\quad 4x_H-4y_H+12=0 &\\
                                    \iff&\quad 4x_H-4y_H-12 &\\
                                    \iff&\quad x_H=y_H-3 &\\
                                    \iff&\quad y_H=x_H+3
                             \end{flalign*}
                    \end{enumerate}
        \item \begin{itemize}[leftmargin=3cm]
                        \item[Méthode 1:] $\begin{cases}\overrightarrow{BH}\begin{psmallmatrix}x_H-x_B\\y_H-y_B\end{psmallmatrix}\\[0.2cm]\overrightarrow{AC}\begin{psmallmatrix}x_C-x_A\\y_C-y_A\end{psmallmatrix}\end{cases}\iff\begin{cases}\overrightarrow{BH}\begin{psmallmatrix}x-2\\y+1\end{psmallmatrix}\\[0.2cm]\overrightarrow{AC}\begin{psmallmatrix}-2\\0\end{psmallmatrix}\end{cases}$ \medbreak $\overrightarrow{BH}\cdot\overrightarrow{AC}=x_{\overrightarrow{BH}}x_{\overrightarrow{AC}}+y_{\overrightarrow{BH}}y_{\overrightarrow{AC}}=4-2x_H$ \medbreak
                        \item[Méthode 2:] La hauteur issue de AC passe par le point H et B. \\ Donc, \quad $([BH]\perp[AC])\iff\overrightarrow{BH}\cdot\overrightarrow{AC}=0$
                    \end{itemize}
        \item $\begin{cases}4-2x_H=0\\[0.2cm]y_H=x_H+3\end{cases}\iff\begin{cases}x_H=2\\[0.2cm]y_H=5\end{cases} \quad H(2\ ;5)$
    \end{enumerate}
\end{Exercise}

\begin{Exercise}[number={81}]
    \begin{enumerate}[a)]
        \item $\overrightarrow{CD}\cdot\overrightarrow{EA}=CD\times EA\times\cos{\ 0}=ax$ \\ $\overrightarrow{DF}\cdot\overrightarrow{AD}=DF\times AD\times\cos{\ 180^{\circ}}=-ax$
        \item On pose le repère $\delimiterfactor=1200\left(D\ ;\overrightarrow{DC}\ ;\overrightarrow{DA}\right)$.\medskip \\ 
                    $\begin{cases}\overrightarrow{CF}\begin{psmallmatrix}x_F-x_C\\y_F-y_C\end{psmallmatrix}\\[0.2cm]\overrightarrow{ED}\begin{psmallmatrix}x_D-x_E\\y_D-y_E\end{psmallmatrix}\end{cases}$ 
                    $\iff \begin{cases}\overrightarrow{CF}\begin{psmallmatrix}-a\\x\end{psmallmatrix}\\[0.2cm]\overrightarrow{ED}\begin{psmallmatrix}-x\\-a\end{psmallmatrix}\end{cases}$ \medskip \\
                    $\overrightarrow{CF}\cdot\overrightarrow{ED}=-a\times-x-a\times x=0$ \smallskip \\
                    Or, \quad $\overrightarrow{CD}\cdot\overrightarrow{ED}=0\iff(CF)\perp(ED)$
    \end{enumerate}
\end{Exercise}

\begin{Exercise}[number={83}]
    \begin{minipage}{\dimexpr\textwidth-10px-\parindent\relax}
    \medbreak Tout d'Abord, $(TU)\perp(RS)\implies\overrightarrow{TU}\cdot\overrightarrow{RS}=0\iff x_{\overrightarrow{TU}}x_{\overrightarrow{RS}}+y_{\overrightarrow{TU}}y_{\overrightarrow{RS}}=0$ \smallbreak De plus, $\overrightarrow{RS}\begin{psmallmatrix}x_S-x_R\\y_S-y_R\end{psmallmatrix}\iff\overrightarrow{RS}\begin{psmallmatrix}6\\2\end{psmallmatrix}$, \quad et \quad $x_{\overrightarrow{TU}}=x_U-x_T$, \quad $y_{\overrightarrow{TU}}=y_U-y_T$ \smallbreak
        \moveright\parindent\vbox
        {%
            {\setlength{\abovedisplayskip}{0pt}
            {\setlength{\belowdisplayskip}{0pt}
                \begin{flalign*}
                    \overrightarrow{TU}\cdot\overrightarrow{RS}=0&\iff 6x_{\overrightarrow{TU}}+2y_{\overrightarrow{TU}}=0 &\\
                    &\iff 6(x_U-3)=-2(y_U+2) &\\
                    &\iff 6x_U-18=-2y_U-4 &\\
                    &\iff 6x_U+2y_U=14
                \end{flalign*}
            }
            }
        }
        Tout doublet qui satisfie cette equation représente un point qui se situera sur la droite (TU), mais n'appartiendra pas forcément à la doite (RS). \\ On définit donc cette droite. $y=\frac{1}{3}x+\frac{11}{3}$ \\ On cherche donc le doublet qui satisfait les deux équations. \medbreak
        $\begin{cases}6x_U+2y_U=14\\[0.2cm]y_U=\dfrac{1}{3}x_U+\dfrac{11}{3}\end{cases}\iff\begin{cases}6x_U+\dfrac{2}{3}x_U+\dfrac{22}{3}=14\\[0.2cm] y_U=\dfrac{1}{3}x_U+\dfrac{11}{3}\end{cases}\iff\begin{cases}x_U=1\\[0.2cm]y_U=4\end{cases} \quad U(1\ ;4)$
    \end{minipage}
\end{Exercise}

\begin{Exercise}[number={85}]
    D'Après le théorème d'Al-Kashi: \smallbreak $AC^2=BC^2+BA^2-2\times BC\times BA\times\cos{\ \widehat{ABC}}$ \smallbreak $AC=\sqrt{AB^2+BC^2-2\times AB\times BC\times\cos{\ 30^{\circ}}}\approx 4{,}6$
\end{Exercise}

\pagebreak

\begin{Exercise}[number={86}]
    D'Après le théorème d'Al-Kashi: \smallbreak $EF^2=DE^2+DF^2-2\times DE\times DF\times\cos{\ \widehat{EDF}}$ \smallbreak $EF=\sqrt{DE^2+DF^2-2DE\times DF\times\cos{\ 45^{\circ}}}\approx 3{,}6$
\end{Exercise}

\begin{Exercise}[number={87}]
    \begin{enumerate}[a)]
        \item $IJ^2+JK^2=IK^2$ \smallbreak Donc, d'après la réciproque du théorème de Pythagore, le triangle IJK est un triangle rectangle en J.
        \item $\widehat{J}=90^{\circ}$ \medbreak $\widehat{I}=\cos^{-1}\left({\dfrac{j^2+k^2-i^2}{2jk}}\right)\approx 41^{\circ}$ \medbreak $\widehat{K}=\cos^{-1}\left({\dfrac{i^2+j^2-k^2}{2ij}}\right)\approx 49^{\circ}$
    \end{enumerate}
\end{Exercise}

\begin{Exercise}[number={89}]
    \begin{enumerate}[a)]
        \item $\cos{\ \widehat{G}}=\dfrac{h^2+l^2-g^2}{2hl}=\dfrac{31}{44}$ \smallbreak $\cos{\ \widehat{H}}=\dfrac{g^2+l^2-h^2}{2gl}=-\dfrac{7}{32}$
        \item $\widehat{L}=180-\cos^{-1}\left({\ \dfrac{31}{44}}\right)-\cos^{-1}\left({-\dfrac{7}{32}}\right)\approx 32{,}2^{\circ}$
    \end{enumerate}
\end{Exercise}

\begin{Exercise}[number={90}]
    \begin{minipage}{\dimexpr\textwidth-10px-\parindent\relax}
        \begin{flalign*}
            &\quad MA^2+MB^2=2MI^2+\dfrac{AB^2}{2} &&\\
            \iff&\quad MI=\sqrt{\frac{1}{2}\left(MA^2+MB^2-\frac{AB^2}{2}\right)} &&\\
            \iff&\quad MI=5
        \end{flalign*}
    \end{minipage}
\end{Exercise}

\begin{Exercise}[number={91}]
    \begin{minipage}{\dimexpr\textwidth-10px-\parindent\relax}
        \medbreak $n^2=q^2+r^2-2pr\times\cos{\ \widehat{N}}=76$ \\ $n=2\sqrt{19}\approx 8{,}7\approx 4{,}4NP$
    \end{minipage}
\end{Exercise}

\pagebreak

\begin{Exercise}[number={92}]
    \medbreak \begin{minipage}{\dimexpr\textwidth-10px-\parindent\relax}
        D'après le théorème d'Al-Kashi: 
        \begin{flalign*}
            &\quad r^2=s^2+n^2-2sn\times\cos{\ \widehat{R}} &\\
            \iff&\quad \widehat{R}=\cos^{-1}\left({\dfrac{s^2+n^2-r^2}{2sn}}\right)\approx 123^{\circ} &\\
            \text{De même,} &\quad \widehat{N}=\cos^{-1}\left({\dfrac{r^2+s^2-n^2}{2rs}}\right)\approx 16^{\circ} &\\
            &\quad \widehat{S}=\cos^{-1}\left({\dfrac{n^2+r^2-s^2}{2nr}}\right)\approx 41^{\circ}
        \end{flalign*}
    \end{minipage}
\end{Exercise}

\begin{Exercise}[number={93}]
    \begin{enumerate}[a)]
        \item D'après le théorème d'Al-Kashi dans le trianlge ABC, et avec $a=BC$, $b=AC$ et $c=AB$:
                    \begin{flalign*}
                        &\quad a^2=b^2+c^2-2bc\times\cos{\ \widehat{A}} &\\
                        \iff&\quad \widehat{A}=\cos^{-1}\left({\dfrac{b^2+c^2-a^2}{2bc}}\right)\approx 26{,}6^{\circ} &\\
                        \text{De même,}&\quad \widehat{B}=\cos^{-1}\left({\dfrac{c^2+a^2-b^2}{2ca}}\right)\approx 137{,}9^{\circ} &\\
                        &\quad \widehat{C}=\cos^{-1}\left({\dfrac{a^2+b^2-c^2}{2ab}}\right)\approx 15{,}6^{\circ}
                    \end{flalign*}
        \item D'après le théorème d'Al-Kashi dans le triangle EDF, et avec $d=EF$, $e=DF$ et $f=DE$:
                    \begin{flalign*}
                        d&=\sqrt{e^2+f^2-2ef\times\cos{\dfrac{\pi}{3}}} &\\
                        &=\sqrt{79} &\\
                        &\approx 8{,}9
                    \end{flalign*}
        \item D'après le théorème d'Al-Kashi dans le triangle IJK, et avec $i=JK$, $j=IK$ et $k=IJ$:
                    \begin{flalign*}
                        i&=\sqrt{j^2+k^2-2jk\times\cos{\dfrac{\pi}{6}}} &\\
                        &\approx 17{,}8
                    \end{flalign*}
        \item D'après le théorème d'Al-Kashi dans le triangle GHL, et avec $g=HL$, $l=GH$ et $h=GL$:
                    \begin{flalign*}
                        &\quad h^2=g^2+l^2-2gl\times\cos{\ \widehat{H}} &\\
                        \iff&\quad \widehat{H}=\cos^{-1}\left({\dfrac{g^2+l^2-h^2}{2gl}}\right)\approx71^{\circ}
                    \end{flalign*}
    \end{enumerate}
\end{Exercise}

\pagebreak

\begin{Exercise}[number={95}]
    \begin{minipage}{\dimexpr\textwidth-10px-\parindent\relax}
        \medbreak D'après le théorème de la médiane:
            \begin{flalign*}
                &\quad MP^2+MN^2=2MM'^2+\dfrac{MP^2}{2} &\\
                \iff&\quad MM'=\sqrt{\frac{1}{2}\left(MP^2+MN^2-\dfrac{NP^2}{2}\right)} &\\
                \iff&\quad MM'\approx 6{,}2
            \end{flalign*}
            \begin{flalign*}
                &\quad MP^2+NP^2=2PP'^2+\dfrac{NM^2}{2} &\\
                \iff&\quad PP'=\sqrt{\frac{1}{2}\left(MP^2+NP^2-\dfrac{NM^2}{2}\right)} &\\
                \iff&\quad PP'\approx 4{,}5
            \end{flalign*}
            \begin{flalign*}
                &\quad NM^2+NP^2=2NN'^2+\dfrac{MP^2}{2} &\\
                \iff&\quad NN'=\sqrt{\frac{1}{2}\left(NM^2+NP^2-\dfrac{MP^2}{2}\right)} &\\
                \iff&\quad NN'\approx 7{,}9
            \end{flalign*}
    \end{minipage}
\end{Exercise}

\begin{Exercise}[number={96}]
    \begin{enumerate}[a)]
        \item D'après le théorème d'Al-Kashi dans le triangle ABC, avec $a=BC$, $b=AC$ et $c=AB$:
                    \begin{flalign*}
                        &\quad b^2=a^2+c^2-2ac\times\cos{\widehat{B}} &\\
                        \iff&\quad b=\sqrt{a^2+c^2-2ac\times\cos{\widehat{B}}} &\\
                        \iff&\quad AC=\sqrt{19}\approx 4{,}4
                    \end{flalign*}
        \item De plus,
                    \begin{flalign*}
                        &\quad \widehat{A}=\cos^{-1}\left({\dfrac{b^2+c^2-a^2}{2bc}}\right) &\\
                        \iff&\quad \widehat{A}\approx 32^{\circ} &\\
                    \end{flalign*}
                    \begin{flalign*}
                        &\quad \widehat{C}=180-30-32 &\\
                        \iff&\quad \widehat{C}=118^{\circ}
                    \end{flalign*}
    \end{enumerate}
\end{Exercise}

\pagebreak

\begin{Exercise}[number={97}]
    \begin{enumerate}[a)]
        \item D'Après le théorème d'Al-Kashi dans le triangle IMJ, \\avec $i=MJ$, $m=IJ$ et $j=IM$. $\widehat{I}=30^{\circ}$:
        \begin{flalign*}
            &i^2=m^2+j^2-2mj\times\cos{\ \widehat{I}}&\\
            \iff\quad &i=\sqrt{m^2+j^2-2mj\times\cos{\ \widehat{I}}}&\\
            \iff\quad &MJ\approx 3{,}4
        \end{flalign*}
        De même, dans le triangle IML, \\avec $i=[ML]$, $m=[IL]$ et $l=[IM]$. $\widehat{I}=60^{\circ}$:
        \begin{flalign*}
            &i^2=m^2+l^2-2ml\times\cos{\ \widehat{I}}&\\
            \iff\quad &i=\sqrt{m^2+l^2-2ml\times\cos{\ \widehat{I}}}&\\
            \iff\quad &ML\approx 3{,}5
        \end{flalign*}
        \item On cherche d'Abord $\widehat{MLK}$. \quad $\widehat{MLK}=90-\widehat{ILM}$ \\ D'après le théorème d'Al-Kashi dans le triangle IML, avec $l=IM$, $i=ML$ et $m=IL$:
        \item \setlength{\abovedisplayskip}{0pt}
        \begin{flalign*} 
            &\quad l^2=i^2+m^2-2im\times\cos{\ \widehat{ILM}}&\\
            \iff&\quad\widehat{ILM}=\cos^{-1}\left({\dfrac{i^2+m^2-l^2}{2im}}\right)&\\
            \iff&\quad\widehat{ILM}=30^{\circ}=\frac{\pi}{6}.
        \end{flalign*}
        On conclut que $\widehat{MLK}=60^{\circ}=\frac{\pi}{3}$ \\ D'après le théorème d'Al-Kashi, dans le triangle MLK, avec $l=MK$, $m=LK$ et $k=ML$:
        \begin{flalign*}
            &\quad l^2=m^2+k^2-2mk\times\cos{\ \widehat{L}}&\\
            \iff&\quad l=\sqrt{m^2+k^2-2mk\times\cos{\ \widehat{L}}}&\\
            \iff&\quad MK\approx 4{,}4
        \end{flalign*}
    \end{enumerate}
\end{Exercise}

\begin{Exercise}[number={99}]
    \begin{minipage}{\dimexpr\textwidth-10px-\parindent\relax}
        \medbreak $[AB]=\cos{\ \dfrac{\pi}{6}}\times [AE]=7$ \medbreak D'Après la contraposée du théorème de Pythagore, dans le triangle ABE, rectangle en E: \medbreak $[BE]=\sqrt{AE^2-AB^2}=\dfrac{7\sqrt{3}}{3}\approx 4$ \medbreak D'après le théorème d'Al-Kashi, dans le triangle ADB, avec $a=DB$, $d=AB$ et $b=AD$: \medbreak $a=\sqrt{d^2+b^2-2db\times\cos{\dfrac{\pi}{6}}}\approx 3{,}5$ \medbreak Donc, le chemin a une longeur d'à peu près $7{,}5$
    \end{minipage}
\end{Exercise}

\pagebreak

\begin{Exercise}[number={101}]
    \begin{minipage}{\dimexpr\textwidth-10px-\parindent\relax}
        \medbreak D'Après le théorème d'Al-Kashi, dans le triangle ABC, avec $a=BC$, $b=AC$ et $c=AB$: \medbreak $\widehat{I}=\cos^{-1}\left({\dfrac{b^2+c^2-a^2}{2bc}}\right)\approx 39^{\circ}$ \medbreak $[CH]=\sin{\ 39^{\circ}}\times [AC]\approx 3{,}7$ \medbreak $\text{Aire}_{ABC}=\dfrac{bh}{2}\approx 15$
        \end{minipage}
\end{Exercise}

\begin{Exercise}[number={108}]
    \begin{minipage}{\dimexpr\textwidth-10px-\parindent\relax}
        \medbreak On cherche d'abord la longuer de $ST$. \\ D'Après le théorème d'Al-Kashi, dans le triangle RST, avec $r=ST$, $s=RT$ et $t=RS$: \begin{flalign*}
            &\quad r^2=s^2+t^2-2st\times\cos{\widehat{TRS}} &\\
            \iff&\quad r=\sqrt{s^2+t^2-2st\times\cos{45^{\circ}}} &\\
            \iff&\quad ST\approx 3{,}6
        \end{flalign*}
        On définit le point $I$ tel que $TI=IS$
        \begin{flalign*}
            &\quad RT^2+RS^2=2RI^2+\dfrac{TS^2}{2} &\\
            \iff&\quad RI=\sqrt{\dfrac{1}{2}\left(RT^2+RS^2-\dfrac{TS^2}{2}\right)} &\\
            \iff&\quad RI\approx 4{,}2
        \end{flalign*}
    \end{minipage}
\end{Exercise}

\begin{Exercise}[number={109}]
    \begin{enumerate}[a)]
        \item Pour BP: \medbreak $\widehat{BAP}=\widehat{BAC}-15$ \medbreak D'après le théorème de Pythagore dans le triangle ABC, rectangle en B: \\ $AC=\sqrt{AB^2+BC^2}\approx 10{,}8$ \medbreak D'après le théorème d'Al-Kashi dans le triangle ABC, avec $a=BC$, $b=AC$ et $c=AB$:
        \begin{flalign*}
            &\quad a^2=b^2+c^2-2bc\times\cos{\widehat{BAC}} &\\
            \iff&\quad \widehat{BAC}=\cos^{-1}\left({\dfrac{b^2+c^2-a^2}{2bc}}\right) &\\
            \iff&\quad \widehat{BAC}\approx 21{,}7^{\circ}
        \end{flalign*}
        Donc, $\widehat{BAP}\approx6{,}8^{\circ}$ \medbreak D'après le théorème d'Al-Kashi dans le triangle ABP, avec $a=BP$, $b=AP$ et $p=AB$:
        \begin{flalign*}
            &\quad a=\sqrt{b^2+p^2-2bp\times\cos{\ 6{,}8^{\circ}}} &\\
            \iff&\quad BP\approx 7{,}0
        \end{flalign*} 
        
        \pagebreak

        Pour DP: \medbreak $\widehat{PAD}=90-6{,}8=84{,}2^{\circ}$ \medbreak D'après le théorème d'Al-Kashi dans le triangle ADP, avec $a=DP$, $d=AP$ et $p=AD$:
        \begin{flalign*}
            &\quad a=\sqrt{d^2+p^2-2dp\times\cos{\ 83{,}2^{\circ}}} &\\
            \iff&\quad DP\approx 4{,}7
        \end{flalign*}
        \item Pour CP: \medbreak D'après le théorème d'Al-Kashi dans le triangle ACP, avec $a=CP$, $c=AP$ et $p=AC$:
        \begin{flalign*}
            &\quad a=\sqrt{c^2+p^2-2cp\times\cos{\ 15^{\circ}}} &\\
            \iff&\quad CP\approx 7{,}9
        \end{flalign*}
        \end{enumerate}
\end{Exercise}

\begin{Exercise}[number={112}]
    \begin{minipage}{\dimexpr\textwidth-10px-\parindent\relax}
        \medbreak D'après le théorème d'Al-Kashi dans le triangle IJK, avec $i=JK$, $j=IK$ et $k=IJ$:
        \begin{flalign*}
            &\quad i^2=j^2+k^2-2jk\times\cos{\widehat{I}} &\\
            \iff&\quad i=\sqrt{j^2+k^2-2jk\times\cos{\widehat{I}}} &\\
            \iff&\quad JK\approx15{,}7
        \end{flalign*}
        \begin{flalign*}
            &\quad\widehat{J}=cos^{-1}\left({\dfrac{i^2+k^2-j^2}{2ik}}\right) &\\
            \iff&\quad \widehat{J}\approx9{,}6^{\circ}
        \end{flalign*}
        Donc, $\widehat{K}=50{,}5^{\circ}$
    \end{minipage}
\end{Exercise}

\begin{Exercise}[number={114}]
    \begin{enumerate}[a)]
        \item Le triangle $PRR'$ est isocèle si et seulement si $PR=PR'$.
        \begin{flalign*}
            &\quad PR^2+QR^2=2RR'+\dfrac{PQ^2}{2} &\\
            \iff&\quad RR'=\sqrt{\dfrac{1}{2}\left(PR^2+QR^2-\dfrac{PQ^2}{2}\right)}&\\
            \iff&\quad RR'=5
        \end{flalign*}
        $[RR']=[PR]$, donc $PRR'$ est bien isocèle.
        \item $P'$ appartient au cercle si et seulement si $[QP']=[QP]$. \\ Or, \quad $[QP']=\dfrac{\sqrt{97}}{2}\approx 4{,}9$ \quad et \quad $[QP]=12$
    \end{enumerate}
\end{Exercise}

\pagebreak

\begin{Exercise}[number={117}]
    \begin{enumerate}[a)]
        \item	$\overrightarrow{AP}\cdot\overrightarrow{AB}=1$ \quad alors \quad $\overrightarrow{AP}=\frac{1}{AB^2}\times AB=\frac{1}{AB^2}\overrightarrow{AB}$
        \item   $\overrightarrow{AB}\cdot\overrightarrow{AP}=-4$ \quad alors \quad $\overrightarrow{AP}=-\frac{4}{AB^2}\overrightarrow{AB}$
        \item   $\overrightarrow{AB}\cdot\overrightarrow{AP}=2{,}5$ \quad alors \quad $\overrightarrow{AP}=-\frac{2{,}5}{AB^2}\times \overrightarrow{AB}$
        \item   $\overrightarrow{AP}\cdot\overrightarrow{AB}=\sqrt{2}$ \quad alors \quad $\overrightarrow{AP}=\frac{\sqrt{2}}{AB^2}\overrightarrow{AB}$
        \item   $\overrightarrow{BP}\cdot\overrightarrow{AB}=\left(\overrightarrow{BA}+\overrightarrow{AP}\right)\cdot\overrightarrow{AB}=\overrightarrow{AP}\cdot\overrightarrow{AB}-AB^2$ \quad d'où \quad $\overrightarrow{AP}=\frac{AB^2-10}{AB^2}\overrightarrow{AB}$
    \end{enumerate}
\end{Exercise}

\begin{Exercise}[number={118}]
    \begin{enumerate}[a)]
        \item	Soit $H\in(AB)$ tel que $\overrightarrow{AH}\cdot\overrightarrow{AB}=1$, \quad A, H et B étant trois points alignés. $AH=\frac{1}{5}$ \\ $\mathcal{E}:$ La droite perpendiculaire à $(AB)$, passant par le point $H$.
        \item   Soit $H\in[BA)\setminus[AB]$ tel que $\overrightarrow{AB}\cdot\overrightarrow{AH}=-4$, \quad les deux vecteurs ont un sens opposé. \\ $AH=\frac{4}{5}$ \\ $\mathcal{E}:$ La droite perpendiculaires à $(AB)$, passant par le point $H$.
        \item   Soit $H\in[BA)\setminus[AB]$ tel que $\overrightarrow{AH}\cdot\overrightarrow{AB}=-2{,}5$. \quad $AH=\frac{1}{2}$ \\ $\mathcal{E}:$ La droite perpendiculaire à $(AB)$, passant par le point $H$.
        \item   Soit $H\in(AB)$ tel que $\overrightarrow{AH}\cdot\overrightarrow{AB}=\sqrt{2}$, \quad $AH=\frac{\sqrt{2}}{5}$ \\ $\mathcal{E}:$ La droite perpendiculaire à $(AB)$, passant par le point $H$.
        \item   Soit $H\in(AB)$ tel que $\overrightarrow{BH}\cdot\overrightarrow{AB}=-10$, \quad $\overrightarrow{BH}=-\frac{10}{25}\overrightarrow{AB}$ \quad $BH=2$ \\ $\mathcal{E}:$ La droite perpendiculaire à $(AB)$, passant par le point $H$.

    \end{enumerate}
\end{Exercise}

\begin{Exercise}[number={119}]
    \begin{enumerate}[a)]
        \item	$\begin{aligned}[t]
                    \overrightarrow{MA}\cdot\overrightarrow{MB}=MI^2-\frac{AB^2}{4}&\iff MI^2-\frac{AB^2}{4}=3 &\\
                    &\iff MI^2=\frac{21}{4}
                \end{aligned}$
        \item   $\begin{aligned}[t]
                    MI^2-\frac{AB^2}{4}=-3\iff MI^2=-\frac{3}{4} \quad \text{C'est impossible}
                \end{aligned}$
        \item   $\begin{aligned}[t]
                    \overrightarrow{AM}\cdot\overrightarrow{MB}=-10&\iff\overrightarrow{MA}\cdot\overrightarrow{MB}=10 &\\
                    &\iff MI^2-\frac{AB^2}{4}=10 &\\
                    &\iff MI^2=26
                \end{aligned}$
        \item   $\begin{aligned}[t]
                    \overrightarrow{AM}\cdot\overrightarrow{BM}=1&\iff\overrightarrow{MA}\cdot\overrightarrow{MB}=1 &\\
                    &\iff MI^2-\frac{AB^2}{4}=1 &\\
                    &\iff MI^2=\frac{5}{4}
                \end{aligned}$
    \end{enumerate}
\end{Exercise}

\pagebreak

\begin{Exercise}[number={120}]
    \begin{enumerate}[a)]
        \item	$\begin{aligned}[t]
                    \overrightarrow{MA}\cdot\overrightarrow{MB}=-4&\iff MI^2-\frac{AB^2}{4}=-4 &\\
                    &\iff MI^2=0
                \end{aligned}$ \medbreak $\mathcal{E}$: Le point I.
        \item   $\begin{aligned}[t]
                    \overrightarrow{MA}\cdot\overrightarrow{MB}=-1&\iff MI^2-\frac{AB^2}{4}=-1 &\\
                    &\iff MI^2=3
                \end{aligned}$ \medbreak $\mathcal{E}$: Le cercle de centre I et de rayon $\sqrt{3}$.
        \item   $\begin{aligned}[t]
                    \overrightarrow{MA}\cdot\overrightarrow{MB}=2&\iff MI^2-\frac{AB^2}{4}=2 &\\
                    &\iff MI^2=6
                \end{aligned}$ \medbreak $\mathcal{E}$: Le cercle de centre I et de rayon $\sqrt{6}$.
    \end{enumerate}
\end{Exercise}

\begin{Exercise}[number={121}]
    \begin{enumerate}[1)]
        \item	\begin{enumerate}[a)]
                    \item	Les deux vecteurs sont colinéaires, donc $\overrightarrow{DA}\cdot\overrightarrow{DE}=DA\times DE=3$, d'où $DA=\frac{3}{DE}=0{,}6$ et donc $\overrightarrow{DA}=\frac{0{,}6}{5}\overrightarrow{DE}=0{,}12\overrightarrow{DE}$ 
                    \item   Si $M\in\mathcal{D}_1$, \quad alors \quad   $\begin{aligned}[t]
                                                                            \overrightarrow{AM}\cdot\overrightarrow{DE}=0&\iff\left(\overrightarrow{AD}+\overrightarrow{DM}\right)\cdot\overrightarrow{DE}=0 &\\
                                                                            &\iff\overrightarrow{AD}\cdot\overrightarrow{DE}+\overrightarrow{DM}\cdot\overrightarrow{DE}=0 &\\
                                                                            &\iff\overrightarrow{DM}\cdot\overrightarrow{DE}=AD\times DE &\\
                                                                            &\iff\overrightarrow{DM}\cdot\overrightarrow{DE}=3
                                                                    \end{aligned}$ \medbreak Donc, $M\in\mathcal{D}_1\implies\overrightarrow{AM}\cdot\overrightarrow{DE}=0$
                    \item   $\mathcal{D}_1\subset\text{\ perpendiculaire\ }(d)\text{\ de\ }(DE)$ \quad $(d)$ passe par le point A. \\ Soit K un point de la droite $(d)$, \\ 
                    $\begin{aligned}[t]
                        \overrightarrow{AK}\cdot\overrightarrow{DE}=0&\iff\left(\overrightarrow{AD}+\overrightarrow{DK}\right)\cdot\overrightarrow{DE}=0 &\\
                        &\iff -3+\overrightarrow{DK}\cdot\overrightarrow{DE}=0 &\\
                        &\iff\overrightarrow{DK}\cdot\overrightarrow{DE}=3 &\\
                    \end{aligned}$ \medbreak Donc $\mathcal{D}_1$ est la droite $(d)$.
                \end{enumerate}
        \item   \begin{enumerate}[a)]
                    \item   Soit le point B de $(DE)$ tel que $\overrightarrow{DB}\cdot\overrightarrow{DE}=10$. \\ $\overrightarrow{DB}$ et $\overrightarrow{DE}$ sont de même sens et $DB\times DE=10$, $DB=2$, et $\overrightarrow{DB}=\frac{2}{5}\overrightarrow{DE}$. \\
                    \item   $\begin{aligned}[t]
                                P\in\mathcal{D}_2\iff\overrightarrow{DP}\cdot\overrightarrow{DE}=10&\iff\left(\overrightarrow{DB}+\overrightarrow{BP}\right)\cdot\overrightarrow{DE}=10 &\\
                                &\iff 10+\overrightarrow{BP}\cdot\overrightarrow{DE}=10 &\\
                                &\iff \overrightarrow{BP}\cdot\overrightarrow{DE}=0 &\\
                                &\iff\text{P est sur la perpendiculaire à }(DE)\text{ passant par B.}
                            \end{aligned}$ 
                    \item   Donc, $\mathcal{D}_2$ est la perpendiculaire à $(DE)$ passant par B.
                \end{enumerate}
    \end{enumerate}
\end{Exercise}

\begin{Exercise}[number={122}]
    \begin{enumerate}[1)]
        \item	\begin{enumerate}[a)]
                    \item	$\begin{aligned}[t]
                                \overrightarrow{SM}\cdot\overrightarrow{UM}=3&\iff MT^2-\frac{SU^2}{4}=3 &\\
                                &\iff MT^2=19
                            \end{aligned}$
                    \item   Il s'agit d'un cercle de centre T et de rayon $\sqrt{19}$.
                \end{enumerate}
        \item   \begin{enumerate}[a)]
                    \item	$\begin{aligned}[t]
                                \overrightarrow{SP}\cdot\overrightarrow{UP}=-3&\iff PT^2-\frac{SU^2}{4}=-3 &\\
                                &\iff MT^2=13
                            \end{aligned}$
                    \item   Il s'agit d'un cercle de centre T et de rayon $\sqrt{13}$.
                \end{enumerate}
    \end{enumerate}
\end{Exercise}

\begin{Exercise}[number={142}]
    \begin{enumerate}
        \item D'Après le théorème d'Al-Kashi dans le triangle QNZ, avec $q=NZ$, $n=QZ$ et $z=QN$.
        \begin{flalign*}
            &\quad q^2=n^2+z^2-2nz\times\cos{\widehat{NQZ}} &\\
            \iff&\quad q=\sqrt{n^2+z^2-2nz\times\cos{\widehat{NQZ}}} &\\
            \iff&\quad NZ=\dfrac{\sqrt{6}-\sqrt{2}}{2}
        \end{flalign*}
        \item \begin{enumerate}[a)]
                        \item Dans un triangle isocèle, la hauteur coupe la base en son milieu, de plus,
                        \begin{flalign*}
                            &\quad\widehat{HQZ}=\sin^{-1}\left({\dfrac{HZ}{QZ}}\right) &\\
                            \iff&\quad\widehat{HQZ}=\sin^{-1}\left({\dfrac{\sqrt{6}-\sqrt{2}}{4}}\right) &\\
                            \iff&\quad\widehat{HQZ}=15^{\circ}
                        \end{flalign*}
                        \item $\sin{15^{\circ}}=\dfrac{\sqrt{6}-\sqrt{2}}{4}$
                        \item $HQ=\cos{15^{\circ}}=\dfrac{\sqrt{6}+\sqrt{2}}{4}$
                    \end{enumerate}
    \end{enumerate}
\end{Exercise}

\begin{Exercise}[number={144}]
    \begin{enumerate}[a)]
        \item D'après la loi des Sinus dans le triangle DEF, \\ \quad $\dfrac{\sin{\widehat{D}}}{d}=\dfrac{\sin{\widehat{E}}}{e}=\dfrac{\sin{\widehat{F}}}{f}$ \qquad De plus, \quad $\widehat{F}=97^{\circ}$ \medbreak D'où \quad $e=\dfrac{\sin{\widehat{E}}\times f}{\sin{\widehat{F}}}\approx 3{,}0$ \qquad $d=\dfrac{\sin{\widehat{D}}\times f}{\sin{\widehat{F}}}\approx 2{,}3$ \medbreak
        \item D'après la loi des Sinus dans le triangle GHK, \\ \quad $\dfrac{\sin{\widehat{K}}}{k}=\dfrac{\sin{\widehat{G}}}{g}=\dfrac{\sin{\widehat{H}}}{h}$ \qquad De plus, \quad $\widehat{G}=105^{\circ}$ \medbreak D'où \quad $k=\dfrac{\sin{\widehat{K}}\times g}{\sin{\widehat{G}}}\approx 7{,}3$ \qquad $h=\dfrac{\sin{\widehat{H}}\times g}{\sin{\widehat{G}}}\approx 5{,}2$
    \end{enumerate}
\end{Exercise}

\begin{Exercise}[number={146}]
    \begin{minipage}{\dimexpr\textwidth-10px-\parindent\relax}
    \medbreak Je ne sais pas résoudre le problème sans faire recours à la loi des Sinus, vue en Enseignement Scientifique.\footnote{je viens de voir que l'on peut utiliser la loi des sinus, démontrée dans l'exercice 143} \\
    On trouve $\widehat{R}=180-30-45=105^{\circ}$ \bigbreak
    D'après la loi des Sinus: \qquad $\dfrac{\sin{\widehat{A}}}{a}=\dfrac{\sin{\widehat{R}}}{r}=\dfrac{\sin{\widehat{B}}}{b}$ \medbreak
    D'où: \quad $a=\dfrac{\sin{\widehat{A}\times r}}{\sin{\widehat{R}}}\approx 2{,}6$ \quad et \quad $b=\dfrac{\sin{B}\times r}{\sin{R}}\approx 3{,}7$
    \end{minipage}
\end{Exercise}

\end{document}