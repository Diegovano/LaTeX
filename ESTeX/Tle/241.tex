\documentclass{scrartcl}
\usepackage[french]{babel}
\usepackage[T1]{fontenc}
\usepackage{newpxtext}
\usepackage{newpxmath}

\KOMAoptions{fontsize=10pt}

\title{Vers une Étude Systématique de la Terre}
\author{Diego Van Overberghe}

\begin{document}
    \maketitle
    \begin{enumerate}
        \item   La Terre est constituée de noyau et manteau, une lithosphère, recouverte par une hydrosphère (océans par exemple), biosphère (ensemble des êtres vivants) ainsi qu'une atmosphère (les gaz qui enveloppent la planète)
        
                Nos choix énergétiques peuvent avoir un impacte sur la lithosphère (épuisement des ressources fossiles), l'hydrosphère (construction de barrages), l'atmosphère (émission de gaz à effet de serre) ainsi que la biosphère (destructions d'habitats naturels en construisant des fermes solaires)
        \item   D'après le document 2, les études de l'histoire de la vie et l'histoire de la Terre doivent se faire ensemble car tous les éléments du système Terre sont en interaction. Cette science se nomme science du système Terre (SST). L'établissement de cette science (à partir des années '80) constitue une avancée dans l'étude du réchauffement climatique car la SST permet à l'humanité de prévoir les conséquences de l'activité humaine sur l'évolution de la Terre et ainsi pouvoir faire évoluer les choix de l'humanité, notamment énergétiques (d'après le document 2) en ayant conscience des effets de ses choix sur le système Terre.
        \item   A travers le document 3, nous avons trois exemples de transitions énergétiques. Des exemples de transitions qui ont provoquées des ruptures importantes sont la transition de l'utilisation du bois comme combustible à l'utilisation du charbon. Ceci a été possible grâce au développement de la machine à vapeur par James~Watt en 1769.
        
        Une autre transition rapide à été la transition vers l'énergie électrique, possible grâce aux développements de Volta, Tesla et Edison. L'énergie électrique a pu être distribuée a grande échelle.

        Mais il y a également eu des transitions plus lentes, dont notamment celle de la transition de l'utilisation du charbon au pétrole. En effet, de nos jours, il existe encore des centrales à charbon.
        \item   La transition énergétique d'aujourd'hui n'est pas comme ceux d'hier. Alors que les transitions énergétiques précédentes se faisaient parce que le développement avait un véritable avantage industriel, c'est-à-dire que la société avait un intérêt économique à se modifier, notre transition économique a un cout. Nous devons arrêter d'utiliser des techniques ancrées dans le processus productif depuis longtemps. 
        
        Il est difficile de motiver certaines personnes à s'adapter pour faire face à un ennemi invisible. Les avantages du renouvelables ne sont pas tangibles, ne peuvent pas être représentés par une courbe de bénéfice croissante. On doit faire confiance à des scientifiques quelque part, que l'on ne connait pas. Certains ne sont pas prêts a accepter ceci.
    \end{enumerate}
\end{document}