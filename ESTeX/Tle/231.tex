\documentclass{scrartcl}
\usepackage{siunitx}
\sisetup{locale = FR}
\usepackage[french]{babel}
\usepackage[T1]{fontenc}
\usepackage{newpxtext}
\usepackage{newpxmath}
\usepackage[siunitx,european,cuteinductors,straightvoltages,RPvoltages]{circuitikz}
\usetikzlibrary{babel}

\setlength{\parskip}{1em}

\title{Activité 2.3.1. Pourquoi utilise-t-on des lignes à haute tension ?}
\author{Diego Van Overberghe}
\begin{document}
    \maketitle

    \section*{Question 1}

    D'après le document 1, il semble que l'on perd moins de puissance par effet Joule lorsque l'on transporte l'électricité à haute tension. L'effet Joule transforme l'énergie électrique en énergie thermique qui est ensuite dissipée et donc perdue.

    La question qui se pose est, comment augmenter la tension sans changer la puissance ?

    On a la formule de la puissance : \[P=U\times V\]
    On voit qu'il est possible de modifier la tension sans changer la puissance. On augmente la tension en réduisant l'intensité. C'est d'ailleurs la réduction de l'intensité qui permet de réduire les pertes énergétiques puisque l'effet Joule se modélise par $P_{f}=R\times I^2$

    Le changement de tension est aisé avec le régime de courant alternatif, promu par Nikola Tesla et Georges Westinghouse. Il suffit d'utiliser deux bobines qui utilisent le phénomène d'induction électromagnétique, afin de changer la tension avec un rendement quasiment parfait (proche de 100\%)

    Finalement, l'utilisation de câbles à résistance très faible est la dernière partie de notre quête à limiter les pertes lors du transport électrique.

    En synthétisant, voici les choix technologiques à effectuer :
    \begin{itemize}
        \item Utiliser un réseau à haute tension (il faudra réduire la tension pour assurer le fonctionnement des machines domestiques, ainsi que protéger les utilisateurs)
        \item Utiliser un réseau au courant alternatif afin de pouvoir aisément changer la tension.
        \item Utiliser des matériaux à résistance faible pour transporter l'électricité.
    \end{itemize}

    \section*{Question 2}

    \begin{center}\begin{circuitikz}
        \draw   (-2,0) node [transformer](B){}
                (B.base) node{B};
        \draw   (2,0) node [transformer](A){}
                (A.base) node{A};
        \draw   (B.A2) -- (-5, |-B.A2) to[V,v>=$U_1$,l=Barrage,i=$i$] (-5, |-B.A1) -- (B.A1) ;
        \draw   (B.B1) to[R,l=$\SI{4}{\ohm}$,v<=$U$,i=$i$] (A.A1) (A.A2) to[R,l=$\SI{4}{\ohm}$,v<=$U$] (B.B2);
        \draw   (A.B2) -- (5, |-A.B2) to[V,v_<=$\SI{600}{V}$,l=Usine,i=$i$] (5, |-A.B1) -- (A.B1);
    \end{circuitikz}\end{center}
    \parbox{\linewidth}{\captionof{figure}{Schéma du Réseau Électrique}}

    \section*{Question 3}

    Pour un dipôle ohmique on a : \[U=R\times I\] On sait que : \[P_{f}=U\times I\] D'où \[P_f=R\times I^2\]
    Entre A et B nous avons une résistance de $\SI{8}{\ohm}$ pour faire le trajet complet. 

    On a : \[P=U\times I=\text{constante}\iff I=\frac{\text{constante}}{U}\]
    La puissance développée par le barrage (la constante) vaut \\ $P=\num{200000}\times 0{,}15=\SI{3,0e4}{W}$ 
    On va calculer pour chaque tension : \[P_f=R\times\left(\frac{\text{constante}}{U}\right)^2\]

    \begin{itemize}
        \setlength{\itemsep}{1em}

        \item Pour $U_2=\SI{400000}{V}$, \quad $P_f=8\times\left(\frac{\num{3.0e4}}{\num{4e5}}\right)^2=\SI{4.5e-2}{W}$
        \item Pour $U_2=\SI{10000}{V}$, \quad $P_f=8\times\left(\frac{\num{3.0e4}}{\num{1.0e4}}\right)^2=\SI{7.2e1}{W}$
        \item Pour $U_2=\SI{230}{V}$, \quad $P_f=8\times\left(\frac{\num{3.0e4}}{\num{2.3e2}}\right)^2=\SI{1.4e5}{W}$
    \end{itemize}

    \section*{Question 4}

    On observe donc que si on utilise une tension élevée, on peut transporter la même quantité d'énergie, en minimisant les pertes dues à l'effet Joule. Ceci est cohérent avec la formule $P_f=R\times I^2$, car on observe non seulement que c'est uniquement la résistance et l'intensité qui sont responsables de la puissance perdue, mais que l'énergie perdue croît avec le carré de l'intensité !

    \section*{Question 5}

    On a $P=U\times I$ \quad et \quad $U=R\times I$, \quad d'où \quad $P_J=R\times I\times I=R\times I^2$

    \section*{Question 6}

    $E_{diss}=P_f\times\Delta t=R\times I^2\times\Delta t$
\end{document}