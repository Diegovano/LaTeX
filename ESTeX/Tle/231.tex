\documentclass{scrartcl}
\usepackage{siunitx}
\sisetup{locale = FR}
\usepackage[french]{babel}
\usepackage[T1]{fontenc}
\usepackage{newpxtext}
\usepackage{newpxmath}
\usepackage[siunitx,european,cuteinductors,straightvoltages,RPvoltages]{circuitikz}
\usetikzlibrary{babel}

\setlength{\parskip}{1em}

\title{Activité 2.3.1. Pourquoi utilise-t-on des lignes à haute tension ?}
\author{Diego Van Overberghe}
\begin{document}
    \maketitle

    \section*{Question 1}

    Afin de réduire les pertes de puissance électrique lors du transport, on utilise un réseau de transportation à tension variées.

    Pour le transport à longue distance, il est avantageux d'utiliser une tension élevée. Ceci reduit la perte de puissance par l'effet Joule, effet qui transforme l'énergie électrique en énergie thermique. Cette transformation est due à l'intensité du courrant électrique, couplé avec la résistance du dipole ohmique, dans ce cas, le cable.

    La question qui se pose est, comment augmenter la tension sans augmenter la puissance ?

    On a la formule de la puissance : \[P=U\times V\]
    On voit qu'il est possible de conserver la meme puissance, tout en elevant la tension en réduisant l'intensité. C'est d'ailleurs la réduction de l'intensité qui permet de reduire les pertes énergétiques pusique l'effet Joule se modélise par $P_{perdue}=R\times I^2$

    Donc, on se demande, quels choix technologiques doivent etre faits pour facilement changer la tension d'un courrant ? (On arrive finalement à la question posée !)

    Le changement de tension est aisé avec le régime de courrant alternatif, promu par Nikola Tesla et Georges Westinghouse. Il suffit d'utiliser deux bobines qui utilisent le phénomène d'induction électromagnétique, afin de changer la tension avec un rendement quasiment parfait (proche de 100\%)

    Finalement, l'utilisation de cables à résistance très faible est la dernière partie de notre quete à limiter les pertes lors du transport électrique.

    En synthétisant, voici les choix technologiques à éffectuer :
    \begin{itemize}
        \item Utiliser un réseau à haute tension (il faudra réduire la tension pour assurer le fonctionnement des machines domestiques, ainsi que protéger les utilisateurs)
        \item Utiliser un réseau au courrant altérnatif.
        \item Utiliser des materiaux à résistance faible pour transporter l'électricité.
    \end{itemize}

    \section*{Question 2}

    \begin{center}\begin{circuitikz}
        \draw   (-2,0) node [transformer](B){}
                (B.base) node{B};
        \draw   (2,0) node [transformer](A){}
                (A.base) node{A};
        \draw   (B.A2) -- (-5, |-B.A2) to[V,v>=$\SI{200000}{\volt}$,i=$i$] (-5, |-B.A1) -- (B.A1) ;
        \draw   (B.B1) to[R,v<=$\SI{4}{\ohm}$,i=$i$] (A.A1) (A.A2) to[R,v<=$\SI{4}{\ohm}$] (B.B2);
        \draw   (A.B2) -- (5, |-A.B2) to[V,v_<=Usine,i=$i$] (5, |-A.B1) -- (A.B1);
    \end{circuitikz}\end{center}
    \parbox{\linewidth}{\captionof{figure}{Schéma du Réseau Électrique}}

    \section*{Question 3}

    Enoncons l'expression de la puissance thermique développée par un dipole ohmique : \[P_{Th}=R\times I^2\]
    Entre A et B nous avons une résistance de $\SI{8}{\ohm}$ pour faire le trajet complet. 

    On a : \[P=U\times I=\text{constante}\iff I=\frac{\text{constante}}{U}\]
    La puissance développée par le barrage (la constante) vaut \\ $P=\num{200000}\times 0{,}15=\SI{3,0e4}{W}$

    \begin{itemize}
        \item Pour $U_2=\SI{400000}{V}$, \quad $I=\frac{\SI{3,0e4}{}}{\num{400000}}=\SI{7.5e-2}{A}$ \\ $P_{Th}=\SI{4.5e-2}{W}$
        \item Pour $U_2=\SI{10000}{V}$, \quad $I=\frac{\SI{3,0e4}{}}{\num{10000}}=\SI{4.0}{A}$ \\ $P_{Th}=\SI{1.3e1}{W}$
        \item Pour $U_2=\SI{230}{V}$, \quad $I=\frac{\SI{3,0e4}{}}{\num{230}}=\SI{1.3e1}{A}$ \\ $P_{Th}=\SI{1.4e5}{W}$
    \end{itemize}

    \section*{Question 4}

    On observe donc que si on utilise une tension elevée, on peut transporter la meme quantité d'énergie, en minisant les pertes dues à l'effeet joule.

    \section*{Question 5}

    On a $P=U\times I$ \quad et \quad $U=R\times I$, \quad d'où \quad $P_J=R\times I\times I=R\times I^2$

    \section*{Question 6}

    $E=P\times\Delta t=R\times I^2\times\Delta t$
\end{document}