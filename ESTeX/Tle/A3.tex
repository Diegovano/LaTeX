\documentclass[12pt]{article}
\usepackage[francais]{babel}
\usepackage[T1]{fontenc}
\usepackage{siunitx}
\usepackage{chemformula}
\usepackage{lmodern}
% \usepackage{amsmath}
% \usepackage{amssymb}
% \usepackage[shortlabels]{enumitem}
% \usepackage[upright]{fourier}

\sisetup{inter-unit-product=\ensuremath{{}\cdot{}}}

\title{Les Causes du Réchauffement Climatique}
\author{Diego Van Overberghe}
\date{1\textsuperscript{er} Octobre 2020}
\begin{document}
    \maketitle
    De nos jours, on entend très souvent parler de réchauffement climatique, l'augmentation continuelle de la température moyenne à la surface de la Terre, depuis l'ère industrielle. Ce phénomène est tout à fait anthropique, c'est-à-dire resultat directe de l'Homme. Certes, la Terre à connue dans le passé une variation cyclique de la température, mais l'augmentation que l'on connaît actuellement est plus élevée que tout précédent que l'on ait pu mesurer. \\[1ex]

    \indent Les jeunes entre nous auront sûrement appris en 1\textsuperscript{ère} que le réchauffement climatique est lié à l'augmentation de forçage radiatif. Pour les autres, le concept est simple: le forçage radiatif est la différence entre l'énergie reçue et l'énergie émise par la Terre. Quand cette différence croît, il y a de l'énergie en \guillemotleft{} trop \guillemotright{} qui reste coincée dans l'atmosphère. Cette différence croît dû à la présence d'une quantité croissante de gaz à éffet de serre, dont nous ferons l'affaire plus tard. En effet, on peut voir, avec le document 2, que en 2011, le forçage radiatif avait augmenté de preque $2\ \si{W.m^{-2}}$ depuis 1750. Ceci paraît petit, mais correspond à une augmentation de puissance reçue de $1\times 10^{15}\ \si{W}$ au niveau de la Terre entière, soit cent-mille milliards de watts! Cette augmentation d'énergie se manifeste donc par une augmentation de température. \\[1ex]

    \indent Il est donc important de comprendre ce qu'est un gaz à éffet de serre. Ce sont des gaz qui absorbent une type d'onde infrarouge, empêchant cette énergie de sortir de l'atmosphère. En observant le document 3, on voit que les quatre gaz absorbent une partie de l'infrarouge. Certains gaz on plus de potentiel d'absorbance d'infrarouge, par exemple, la vapeur d'eau. Le gaz le plus \guillemotleft{} potent \guillemotright{} selon ce document est le protoxyde d'azode \ch{N2O}, qui a un pouvoir de réchauffement global, ainsi qu'une durée moyenne de séjour dans l'atmosphère très élevée. On peut donc tirer de ce document que les gaz \ch{CO2}, \ch{CH4} et \ch{N2O} ainsi que la vapeur d'eau \ch{H2O_{(gazeux)}}, sont tous des gaz à éffet de serre. \\[1ex]

    \indent En observant les documents 1 et 4, il semble évident qu'il existe une correlation entre l'anomalie de température et la teneur de gaz à éffet de serre dans l'atmosphère. Or, on s'apperçoit que l'augmentation des émissions de ces gaz correspond directemment à une augmentation du produit mondial brut. C'est simple; les gaz à éffet de serre proviennent d'activités productriçes, qui sont à l'origine de notre vie facile, douce, légère. Mais sommes-nous prêts à sacrifier ce luxe pour réduire nos émissions?\\[1ex]

    \indent Je vous entend dire, cher lecteur: Quelle est l'influence de cette science sur \textit{ma} vie? La réponse est simple. Les premières manifestations de ce réchauffement se sont déjà manifestées. Pensez au incendies de forêt en Californie ou en Australie. Voire même le fait que l'air chaud peut contenir plus d'eau que l'air froid, c'est pour cela que l'intensité de la pluie semble augmenter, entraînant des innondations catastrophiques. Bien sur, nous connaissons aussi une montée spéctaculaire du niveaux des mers (environ 200mm depuis 1900 selon le document 6). Vous voyez bien, ce phénomène est bien trop réel, et y trouver des solutions ne sera pas facile.
\end{document}