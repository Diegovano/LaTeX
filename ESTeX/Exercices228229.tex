\documentclass[12pt, a4paper]{article}
\usepackage{amsmath}
\usepackage{amssymb}
\usepackage[shortlabels]{enumitem}
\usepackage[upright]{fourier}
\usepackage[french]{babel}
\usepackage{geometry}
\usepackage{siunitx}
\geometry
{
    a4paper,
    total={170mm, 257mm},
    left=20mm,
    top=20mm,
}

\sisetup{inter-unit-product=\ensuremath{{}\cdot{}}}

\begin{document}
    \title{Enseignement Scientifique \\ Exercices Pages 228 à 229}
    \author{Diego Van Overberghe}
    \date{19 Juin 2020}
    \maketitle

    \setcounter{section}{3}

    \section{De la Sensation à la Perception du Son}
    \begin{enumerate}[1.]
        \item   Les aires auditives se situent au cœur du cerveaux. L'aire dite ``primaire'', est chagée de traiter les sons à travers la distinction des fréquences et des inténsités des sons. La zone ``secondaire'', quand à elle, traite les sons plus fins, tels que les mots individuels.
        \item   L'université d'Indiana a realisé une experience qui a prouvé que l'audition joue un role central dans l'apprentissage du langage. Les chercheurs ont découvert que les sujets sourds jeunes apprennent à reconnaître le langage beaucoup plus facilement, s'il ont un implant. Les chercheurs ont donc pu conclure que l'audition est extrêmement important pour le intérioriser le langage.
        \item   L'audition de musique ne fait pas que intervenir les aires auditives, mais aussi des aires supplémentaires, telles que l'Hippocampe, le Cortex Temporal, et l'Amygdale Cérébrale. Chaque personne connaîtra un traitement différent d'une même mélodie, selon leur entraînement, familiarité avec le genre, parmi d'autres. De plus, d'après le document {\Large\textcircled{\small{4}}}, l'Amygdale Cérébrale est stimulée plus ou moins selon les émotions qu'ils ressentent par rapport à cette musique. Nous pouvons donc conclure que l'experience qu'est la musique est bel et bien différent pour chaque personne, et à chaque moment. 
    \end{enumerate}
\end{document}