\documentclass[12pt, a4paper]{article}
\usepackage{amsmath}
\usepackage{amssymb}
\usepackage[shortlabels]{enumitem}
\usepackage[upright]{fourier}
\usepackage[french]{babel}
\usepackage{geometry}
\usepackage{siunitx}
\usepackage{tikz}
\usepackage{caption}
\geometry
{
    a4paper,
    total={170mm,257mm},
    left=20mm,
    top=20mm,
}

\sisetup{inter-unit-product=\ensuremath{{}\cdot{}}}

\begin{document}
    \title{Enseignement Scientifique \\ Exercices Pages 224 à 227}
    \author{Diego Van Overberghe}
    \date{15 Mai 2020}
    \maketitle

    \section{L'Oreille, Organe Sensoriel de l'Audition}

    \begin{enumerate}[1.]
        \item	Quand on parle de ``tendre l'oreille'', on parle de la partie externe de l'oreille. Cette partie joue le rôle de canaliser les sons vers le tympan.
        \item   Les sons graves sont perçus dans la zone C, c'est-à-dire la zone la plus profonde. Les sons aïgus, quand à eux sont perçus dans la zone A.
        \item   Les cellules ciliées transforment les stimulus physiques (les vibrations) en signaux nerveux électriques.
        \item   \parbox[t][][c]{\linewidth}{\centering\begin{tikzpicture}[scale=1.5, every node/.style={scale=1.5}]]
                    \draw[fill=lightgray] (-1, -1) rectangle (1, 1) (0,0) node[rotate=45, scale=0.75]{Oreille Externe} (1, -1) rectangle (3, 1) (2, 0) node[rotate=45, scale=0.75]{Oreille Moyenne} (3, -1) rectangle (5, 1) (4, 0) node[rotate=45, scale=0.75]{Oreille Interne};
                    \draw[thick, color=red] (1,-1) -- (1, 1) (3, -1) -- (3, 1);
                    \draw (1, 0) node[rotate=90, scale=0.5, fill=lightgray]{Tympan} (3,0) node[rotate=90, scale=0.5, fill=lightgray]{Fenêtre Ovale} (0, -1.1) node[scale=0.5]{Milieu Aérien} (2, -1.1) node[scale=0.5]{Milieu Materiel} (4, -1.12) node[scale=0.5]{Milieu Liquide};
                \end{tikzpicture}} \medbreak
                \parbox{\linewidth}{\captionof{figure}{Schéma Résumant le Trajet d'une Onde Sonore (de Gauche à Droite)}}
    \end{enumerate}

    \section{La Capacité de Réception du Système Auditif}

    \begin{enumerate}[1.]
        \item	Entre $20\ \si{Hz}$ et $20\ \si{kHz}$.
        \item   La hauteur des sons est perçu de manière differente pour les sons moyens. C'est pour cela qu'on nomme aussi cette zone de fréquence.
        \item   La perception de la hauteur d'un son peut être differente selon la gamme vocale d'une personne. C'est la référence que chaque personne a pour juger la hauteur d'un son.
    \end{enumerate}

    \section{La Fragilité du Système Auditif}

    \begin{enumerate}[1.]
        \item	Deux heures passées à la cantine n'ont pas d'effet néfaste sur la santé long-terme de l'oreille, cependant, à partir de $4\ \si{min.jour^{-1}}$ de musique au volume maximal, un endommagement permanant peut être infligé.
        \item   Les cellules ciliées de Grégory sont endommagées de manière irréversible. Ceci reduira la sensitivité auditive de Grégory.
        \item   Minimiser le temps passé dans des environments bruyants, si ceci n'est pas possible, utiliser de la protection auditive.
        \item   Il est possible de rupturer son tympan, ceci augmente le risque d'inféction de l'oreille.
    \end{enumerate}

\end{document}